\chapter{为什么说“发展才是硬道理”,但发展的道路不止一条?}

在上一章中,我们揭示了“资源诅咒”的奥秘,理解了自然资源这把“双刃剑”如何既能带来财富,也可能带来停滞与腐败。我们看到,健全的制度和高瞻远瞩的治理,是避免“富饶陷阱”的关键。然而,对于世界上大多数在二战后获得独立、既没有雄厚工业基础也缺乏自然资源优势的新生国家而言,它们面临的挑战更为严峻。它们如同站在一片广袤而陌生的十字路口,眼前是迷雾重重的未来,耳边回响着一个响亮却又空泛的口号——“发展才是硬道理”。

发展,这个词汇在20世纪下半叶充满了魔力。它意味着告别贫穷与饥饿,意味着建设现代化的工厂与城市,意味着提升国民的教育与健康水平,更意味着一个国家能够在国际舞台上赢得尊严与实力。然而,这个“硬道理”的背后,却隐藏着一个更为艰难的问题:\textbf{路在何方?}

想象一下1960年代两位截然不同的国家领导人的思考:

\begin{quote}
一位是拉丁美洲某国的总统,他看着自己国家的港口,满载着咖啡豆和香蕉的货船离港,换回的却是美国生产的汽车、收音机和各种昂贵的工业品。他感到一种深深的屈辱和不甘。他认为,只要自己的国家还不能生产这些东西,就永远是发达国家的“经济殖民地”,受人剥削。他下定决心:我们必须拥有自己的工业!我们要关起国门,用自己的双手,造出自己的汽车!
\end{quote}

\begin{quote}
另一位是东亚某个刚刚摆脱战争废墟的领导人,他环顾四周,国家一穷二白,资源匮乏,国内市场狭小。他知道,关起门来自己搞,无异于坐以待毙。他望向波涛汹涌的大海,做出了一个截然相反的决定:我们没有市场,就去抢占别人的市场!我们没有技术,就用我们廉价但勤劳的双手,为别人代工,从模仿开始,把产品卖到全世界去!
\end{quote}

这两位领导人的抉择,代表了二战后发展中国家所面临的两种截然不同、甚至相互对立的发展哲学。它们最终演变成了两条影响了数亿人命运的工业化战略——前者是“\textbf{进口替代工业化}”(Import Substitution Industrialization, ISI),后者则是“\textbf{出口导向工业化}”(Export-Oriented Industrialization, EOI)。

这两条道路,没有哪一条在起点时就注定了成败。它们都是在特定的历史背景和理论思潮下,被认为是通往繁荣的希望之路。然而,几十年过去,它们的结果却天差地别。曾经选择第一条道路的许多拉美国家,陷入了通货膨胀、债务危机和经济停滞的“中等收入陷阱”;而选择了第二条道路的东亚“四小龙”等经济体,却创造了举世瞩目的“经济奇迹”。

本章的核心任务,就是要深入解剖这两条发展路径。我们将不仅仅满足于描述它们的特征,更要挖掘其背后的理论根源、运作逻辑,并通过翔实的案例,去理解它们为何在不同的土壤上开出了不同的花、结出了不同的果。这趟旅程将告诉我们,发展的道路从来都不是唯一的,每一个选择的背后,都充满了智慧的闪光、无奈的妥协,以及深刻的历史教训。

\section{发展的十字路口:为何要“造自己的汽车”?}

二战的硝烟散尽,殖民帝国的瓦解催生了一大批亚非拉的新独立国家。这些国家的领袖们在政治上赢得了主权,但在经济上却发现自己依然深陷泥潭。他们普遍面临着一系列令人沮丧的共同挑战:

\begin{itemize}
    \item \textbf{经济结构的“依附性”}:殖民时代留下的遗产,是单一、畸形的经济结构。许多国家的经济命脉,就是出口一两种初级产品——古巴的糖、加纳的可可、智利的铜、马来西亚的橡胶。这些产品的价格在国际市场上波动剧烈,且由发达国家的“核心”市场决定。他们辛勤劳动,却只能换回微薄的利润。
    \item \textbf{工业基础的“一张白纸”}:国内几乎没有任何像样的现代工业。从一颗螺丝钉到一台拖拉机,几乎所有工业制成品都需要从前宗主国或发达国家进口,价格昂贵。
    \item \textbf{技术的“代差”鸿沟}:缺乏现代科学技术和管理经验,也缺乏足够数量的工程师、科学家和熟练工人。
    \item \textbf{资本的极度匮乏}:国内储蓄率低,缺乏工业化所必需的巨额启动资金。
\end{itemize}

面对这种困境,一个深刻的理论思潮在发展中国家的知识分子和决策者中产生了巨大影响——\textbf{依附理论(Dependency Theory)}。

依附理论由普雷维什(Raúl Prebisch)等拉美经济学家提出,它像一剂猛药,辛辣地指出了当时国际经济秩序的“不公”。其核心观点可以通俗地理解为:

\begin{quote}
世界经济体系并非一个公平的竞技场,而是一个等级森严的金字塔。塔尖是少数发达的“\textbf{中心国家}”(如美国、西欧),塔基则是广大的“\textbf{边缘国家}”(即发展中国家)。中心国家通过控制技术、资本和国际贸易规则,从边缘国家购买廉价的原材料和农产品,然后加工成昂贵的工业品再卖回给边缘国家。这种“剪刀差”使得财富源源不断地从边缘流向中心。边缘国家越是努力参与这种不平等的国际分工,就越是深陷于“依附”地位,永远无法实现真正的独立和富强。
\end{quote}

依附理论为许多发展中国家的领导人提供了一个极具说服力的解释框架,它将本国的贫困归咎于外部的不平等结构,而非内部问题。这种理论直接催生了一个看似顺理-成章的政策结论:\textbf{要想打破依附,就必须摆脱这种不平等的国际分工,走一条自主发展的道路。}

而这条“自主发展”之路,最直接的体现,就是“进口替代工业化”(ISI)。

\section{进口替代工业化:关起门来搞建设}

进口替代工业化(ISI)是一种雄心勃勃的“内向型”发展战略。它的核心逻辑,就像一个家庭决定不再去昂贵的商场买衣服,而是自己买来缝纫机和布料,在家里自己做。其终极目标,是通过在国内生产原本需要从国外进口的工业品,一步步替代进口,最终建立起一套完整的、自给自足的本国工业体系,实现真正的经济独立。

\subsection{核心理念:}

\begin{itemize}
    \item \textbf{保护幼稚产业(Infant Industry Protection):} 这是ISI最重要的理论基石。它认为,本国的新兴工业就像一个刚出生的婴儿,如果直接将它扔到世界市场,与那些身强力壮的“成年”跨国公司竞争,无异于以卵击石。因此,国家必须像一位慈爱的母亲,为这个“婴儿”建立一个温暖的“保护性摇篮”,让它在国内市场这个安全的环境里,先学会走路、长大,直到有朝一日能够独立面对风雨。
\end{itemize}

\subsection{主要特征:}

这个“保护性摇篮”是通过一整套精心设计的政策组合来构建的:

\begin{itemize}
    \item \textbf{高关税与贸易壁垒:} 这是最核心的保护手段。政府对几乎所有进口的消费品(如汽车、家电、纺织品)征收高得惊人的关税,或者设置严格的进口数量配额。
    \begin{itemize}
        \item \textbf{举个例子}:假设一辆从德国进口的大众汽车成本是2000美元,政府对其征收150\%的关税,其在国内的售价就变成了5000美元。这时,一家本国新建的汽车厂,哪怕它生产的汽车质量稍差,成本高达3000美元,但在国内市场上依然具有巨大的价格优势。这道“关税墙”有效地将外国竞争者挡在了国门之外。
    \end{itemize}
    \item \textbf{政府主导与国有企业(State-Owned Enterprises, SOEs):} 在资本和技术都极度匮乏的背景下,私人企业无力也无意投资于那些规模巨大、回报周期长的重工业。因此,国家必须亲自下场,扮演“企业家”和“投资人”的角色。政府通过成立大量的国有企业,直接控制钢铁、石化、电力、交通等国民经济命脉。
    \item \textbf{汇率高估与信贷优惠:} 为了支持工业化,政府常常人为地维持本国货币的高汇率。这看似矛盾,实则用心良苦:高估的本币,意味着可以用更少的本国货币换取更多的外汇,从而降低进口工业化所必需的机器设备、关键零部件和技术的成本。同时,国有银行会向那些被选中的进口替代企业提供利率极低的“政策性贷款”和各种财政补贴,进一步降低其生产成本。
    \item \textbf{严格的国内市场导向:} 整个战略的重心完全放在满足国内市场上。企业生产什么、生产多少,主要取决于国内的需求,而很少去考虑产品是否能在国际市场上竞争。
\end{itemize}

\subsection{优点:}

在推行的初期,尤其是在20世纪50-60年代,ISI战略确实为许多发展中国家带来了令人鼓舞的“蜜月期”。

\begin{itemize}
    \item \textbf{保护民族工业:} 在高墙的保护下,许多国家确实从无到有地建立起了一批自己的工业企业,实现了工业化的“零的突破”。一个国家能够生产自己的汽车、电视机,这在当时是巨大的民族自豪感的来源。
    \item \textbf{创造国内就业:} 工业发展带动了大规模的城市化进程,吸引了大量农村劳动力进入城市工厂,形成了一个庞大的产业工人阶级,也催生了新兴的中产阶级。
    \item \textbf{初步的技术积累:} 通过组装、模仿和学习,国内企业在一定程度上掌握了部分工业技术和管理经验。
\end{itemize}

\subsection{缺点:}

然而,“蜜月期”过后,ISI模式的深层次弊病开始逐渐暴露,最终将许多国家拖入了泥潭。这个曾经被寄予厚望的“保护性摇篮”,最终却变成了阻碍成长的“温室”。

\begin{itemize}
    \item \textbf{效率低下与缺乏竞争力:} 这是ISI最致命的弱点。由于长期缺乏来自外部的竞争压力,国内企业养成了“皇帝女儿不愁嫁”的惰性。它们无需努力提高产品质量、降低生产成本或进行技术创新,因为无论产品多差、价格多高,在国内市场上总能卖出去。这些“幼稚产业”最终变成了永远长不大的“巨婴”。
    \item \textbf{寻租与腐败:} 政府对经济的过度干预,意味着它掌握了分配进口配额、发放低息贷款、审批项目许可等巨大权力。这些权力成为了寻租和腐败的温床。企业主不再专注于提高生产效率,而是将更多精力用于搞好与政府官员的关系,通过贿赂来获取特权。这导致了严重的资源错配和效率损失。
    \item \textbf{外汇短缺与债务危机:} ISI战略虽然旨在替代进口消费品,但工业化本身却更加依赖进口资本品(机器设备)和中间产品。与此同时,高估的汇率和对农业的忽视(为了补贴城市工业,常常压低农产品价格),严重打击了传统农产品的出口创汇能力。这就形成了一个致命的循环:进口需求巨大,出口能力孱弱,导致国家长期面临外汇短缺。为了弥补缺口,这些国家不得不大量借入外债。当20世纪80年代初全球利率飙升、大宗商品价格暴跌时,这条资金链应声断裂,引发了席卷整个拉美的“债务危机”。
    \item \textbf{市场规模的天然瓶颈:} 对于大多数发展中国家而言,其国内市场规模相对有限,远不足以支撑起汽车、钢铁等产业实现规模经济,从而无法有效降低单位成本。
\end{itemize}

\subsection{案例分析:从富裕到衰落的阿根廷与“许可证印度”}

20世纪中叶,拉丁美洲是ISI战略最忠实的实践者。而其中,阿根廷的经历最具悲剧色彩。

\textbf{阿根廷的“庇隆主义”与持久的衰落:}

阿根廷在20世纪初曾是世界上最富裕的国家之一,依靠广袤肥沃的潘帕斯草原,出口的农产品和牛肉让它富甲一方,首都布宜诺斯艾利斯被誉为“南美巴黎”。但在二战后,尤其是在胡安·庇隆(Juan Perón)上台后,阿根廷开始全面推行以ISI为核心的“庇隆主义”经济政策。庇隆是一位极具民粹魅力的领袖,他向城市中的工人阶级承诺了一个工业化和高福利的梦想。

\begin{itemize}
    \item \textbf{实践}:政府大力发展国有企业,将铁路、电力等关键部门收归国有;通过高关税将进口商品拒之门外;同时,为了讨好作为其政治基础的城市工会,政府大幅提高工人工资和福利,并严格限制解雇工人。为了给工业化提供资金,政府通过国家贸易机构,以低价强制收购农民的谷物和牛肉,再以国际价格出口,将剪刀差收入用于补贴亏损的国企和城市福利。
    \item \textbf{后果}:初期,这确实带来了工业增长和民众支持。但长期来看,受保护的工业部门效率低下,产品质量差、价格高。而作为国家经济支柱的农业,却因长期受到政策的“惩罚”而严重萎缩,出口创汇能力大减。政府为了维持难以为继的高福利和补贴亏损的国企,不得不大量印钞,导致了长期的、恶性的通货膨胀,货币几度沦为废纸。最终,这个曾经的富国,在经历了数十年的经济停滞、恶性通胀和反复的政治动荡(军事政变与民粹政府交替)后,彻底跌落,至今仍在ISI模式留下的后遗症中挣扎。
\end{itemize}

\textbf{印度的“许可证拉吉”(License Raj):}

独立后的印度,在首任总理尼赫鲁的领导下,也走上了一条类似ISI的“自主发展”道路。印度希望建立一个“社会主义类型”的社会,对苏联的计划经济模式颇为欣赏。这套体系在印度被形象地称为“许可证拉吉”,意为“由许可证支配的统治”。

\begin{itemize}
    \item \textbf{实践}:政府对经济实行了无孔不入的管制。任何企业,无论是开办新厂、扩大生产规模、还是开发新产品,都必须向政府申请内容繁杂、审批流程漫长的许可证。政府的初衷是希望通过计划来合理配置资源,防止资本家垄断。但实际上,这套体系扼杀了企业的活力和创新精神。
    \item \textbf{后果}:企业家们发现,与其花费时间和金钱去研发新技术、提高生产效率,远不如把精力用在首都新德里,贿赂官僚、疏通关系,以更快地拿到许可证来得重要。这导致了普遍的腐败和效率低下。印度的工业产品长期质次价高,在国际上毫无竞争力。直到1991年印度遭遇严重的外汇危机,才被迫开启市场化改革,逐步废除“许可证拉吉”,印度经济才开始真正起飞。这段历史,成为了发展经济学中一个关于政府过度干预如何扼杀经济活力的深刻教训。
\end{itemize}

\section{出口导向工业化:面向世界求发展}

当拉美和南亚国家在ISI的道路上步履蹒跚时,地球另一端的东亚,一些资源匮乏、满目疮痍的国家和地区,却选择了另一条截然不同的道路。出口导向工业化(EOI)是一种“外向型”发展战略,它的核心逻辑是:\textbf{既然国内市场狭小,那就把全世界当作我们的市场;既然我们没有技术,那就从我们唯一拥有的优势——廉价而勤劳的劳动力——开始,通过为世界生产,来学习和积累。}

\subsection{核心理念:}

\begin{itemize}
    \item \textbf{融入全球市场:} 它不把国际市场看作是剥削的陷阱,而是看作是学习和成长的“竞技场”和“大学校”。
    \item \textbf{“发展型国家”(Developmental State):} EOI的成功,绝非简单的自由放任。其背后,是一个强大、高效、有远见的“发展型国家”在运筹帷幄。这个政府不是直接下场当运动员(像ISI那样),而是扮演一个严格的“总教练”角色。它为选定的“国家队选手”(重点企业)提供各种支持,但唯一的目标,就是让它们在国际比赛中赢得金牌。
\end{itemize}

\subsection{主要特征:}

\begin{itemize}
    \item \textbf{以出口为导向的产业政策:} 政府通过一个由精英技术官僚组成的强大机构(如日本的通产省、韩国的经济企划院),精心挑选和扶持具有出口潜力的“战略产业”。扶持手段包括低息贷款、税收减免、研发补贴等。
    \item \textbf{出口纪律(Export Discipline):} 这是EOI与ISI最根本的区别。政府的扶持并非无条件的,而是与企业的\textbf{出口绩效}严格挂钩。一家企业只有在国际市场上成功地卖出了产品,证明了自己有竞争力,才能继续获得政府的支持。如果出口失败,补贴和优惠就会被取消。这种“不成功,便成仁”的压力,迫使企业必须不断地学习、创新、提高效率。
    \item \textbf{汇率低估与金融控制:} 与ISI相反,EOI国家通常会刻意维持本币的低汇率,这使得它们的出口产品在国际市场上显得格外便宜,极具价格优势。同时,政府牢牢控制着银行系统,确保储蓄能够被引导到政府希望发展的出口产业中去。
    \item \textbf{对人力资本的疯狂投资:} EOI国家普遍认识到,在自然资源匮乏的情况下,人才是唯一的宝藏。因此,它们将巨额财政收入投入到教育领域,特别是基础教育和职业技术教育,培养了大量高素质、有纪律的劳动力和工程师队伍。
    \item \textbf{选择性地吸引外资:} 它们欢迎外国直接投资(FDI),但目的不仅仅是钱,更看重的是其带来的先进技术、管理经验和全球销售网络。政府常常会要求外资企业与本国企业合资,并强制进行技术转让。
\end{itemize}

\subsection{优点:}

\begin{itemize}
    \item \textbf{高效率与国际竞争力:} 国际市场这个残酷的“竞技场”,是最好的“磨刀石”。它迫使企业必须在质量、成本和技术上达到世界一流水平,否则就会被淘汰。
    \item \textbf{获取外汇与技术:} 大规模出口带来了源源不断的外汇收入,为进一步进口先进技术和设备提供了保障。与跨国公司的合作与竞争,也加速了技术的学习和消化。
    \item \textbf{规模经济效应:} 面向全球数十亿人的市场,企业可以进行大规模生产,极大地降低了单位成本,形成了强大的规模优势。
    \item \textbf{动态的产业升级:} EOI国家通常会遵循一个清晰的产业升级路径,从最初的劳动密集型产业(如纺织、玩具),逐步升级到资本密集型产业(如钢铁、造船),最终迈向技术密集型产业(如汽车、半导体、电子产品)。
\end{itemize}

\subsection{缺点:}

EOI的成功之路也并非铺满鲜花,而是充满了荆棘和代价。

\begin{itemize}
    \item \textbf{易受外部市场波动影响:} 经济高度依赖出口,意味着国家的命运与全球经济的荣衰紧密相连。一旦主要出口市场(如美国、欧洲)发生经济衰退,或遭遇贸易保护主义,本国经济就会受到沉重打击。1997年的亚洲金融危机就是一次惨痛的教训。
    \item \textbf{威权主义的阴影:} 在经济起飞的初期,大多数EOI国家(如韩国、台湾、新加坡)都处于威权统治之下。政府为了推行发展战略,常常会压制工会运动,限制言论自由,牺牲民众的政治权利,以换取社会“稳定”和低廉的劳动力成本。
    \item \textbf{社会代价高昂:} “经济奇迹”的背后,是几代人付出的巨大牺牲——长时间的工作、高强度的竞争压力、滞后的社会福利体系以及日益扩大的贫富差距。
    \item \textbf{对政府能力要求极高:} EOI的成功,高度依赖于一个廉洁、高效、有远见且不受特殊利益集团俘获的精英官僚团队。这种“发展型国家”的治理能力,是许多发展中国家难以复制的。
\end{itemize}

\subsection{案例分析:韩国的“汉江奇迹”——从废墟到巅峰}

20世纪后半叶,东亚的“四小龙”——韩国、台湾、新加坡和香港——通过EOI战略,上演了人类经济史上最壮观的“逆袭”。其中,韩国的经历尤为典型和震撼。

1960年的韩国,人均GDP不足100美元,比许多非洲国家还要贫穷。这个刚刚经历过残酷战争的国家,资源贫瘠,满目疮痍。然而,在强人总统朴正熙的铁腕领导下,韩国从60年代开始,毅然决然地转向了EOI战略。

\begin{itemize}
    \item \textbf{“国家队”的打造}:政府精心挑选了三星、现代、LG等一批企业,将它们打造成“国家冠军”,即我们熟知的“财阀”(Chaebol)。政府通过控制银行,向这些财阀倾注了海量的廉价贷款,支持它们进入政府指定的战略产业。
    \item \textbf{出口的血路}:政府官员像严厉的教练一样,每天紧盯着这些财阀的出口数据。总统亲自主持“月度出口扩大会议”,对出口成绩优异的企业家给予重奖,对不达标的则予以惩罚甚至撤换。在这种巨大的压力下,韩国企业爆发出惊人的能量。
    \item \textbf{产业升级之路}:韩国的出口产品,清晰地画出了一条从低到高的升级曲线:
    \begin{itemize}
        \item 60年代:假发、胶合板、纺织品。
        \item 70年代:钢铁(浦项制铁)、船舶(现代重工)。
        \item 80年代:汽车(现代汽车)、家电(三星、LG)。
        \item 90年代至今:半导体、智能手机、显示面板。
    \end{itemize}
\end{itemize}

短短三十年间,韩国从一个贫穷的农业国,一跃成为世界级的工业强国和OECD(经济合作与发展组织)成员,创造了震惊世界的“汉江奇迹”。中国大陆在1978年改革开放后,也借鉴了东亚的经验,通过建立经济特区、吸引外资、大力发展出口加工业,成功融入全球产业链,实现了经济的腾飞。

\section{战略的权衡与转型:没有永远正确的道路}

通过对进口替代工业化和出口导向工业化这两种发展战略的比较,我们可以看到,没有单一的“最佳”发展道路,每种策略都有其特定的优缺点和适用条件。它们是两种关于国家如何在世界经济体系中定位自己的根本性哲学。

\begin{itemize}
    \item \textbf{ISI} 是一条\textbf{内向、防御性}的道路。它试图通过“关起门来”建立一个安全的“堡垒”,保护本国工业免受外部冲击,以实现经济自主。但这个堡垒,最终却常常因为与世隔绝而变得僵化和低效,沦为一座“发展的囚笼”。
    \item \textbf{EOI} 是一条\textbf{外向、进攻性}的道路。它选择勇敢地“冲向世界”,在国际市场的惊涛骇浪中学习游泳,通过竞争来锻造自身的实力。这条路风险更高,代价也可能更大,但一旦成功,其回报也是巨大的。
\end{itemize}

历史的经验告诉我们,成功的关键,往往在于能否根据自身国情、发展阶段和国际环境的变化,\textbf{灵活地进行战略调整}。例如,日本和韩国在发展重化工业的初期,也曾采取过高度的贸易保护措施来扶持本国市场,这带有明显的ISI色彩。但它们的根本不同在于,这种保护是\textbf{暂时性的、有选择性的,并且始终以最终的出口为导向}。它们懂得何时该“保护”,何时该“竞争”。

\textbf{中国的转型之路:}

中国的案例尤其具有启发性。在1978年改革开放之前,中国实行的是一种高度集中的计划经济,其自给自足、重工业优先的特点,在某种程度上与ISI有相似之处。然而,改革开放之后,中国开启了向EOI的宏大转型。

\begin{itemize}
    \item \textbf{渐进式开放}:与拉美国家在危机后被迫采取的“休克疗法”不同,中国的转型是渐进的、分阶段的。从建立沿海经济特区(如深圳)作为对外开放的“窗口”和“试验田”开始,逐步吸引外资,发展加工贸易。
    \item \textbf{强大的政府引导}:中国政府在转型中始终保持着强大的引导能力。通过基础设施建设(“要想富,先修路”)、提供优惠政策、维持相对稳定的宏观环境,为出口导向型经济的发展创造了有利条件。
    \item \textbf{利用巨大国内市场}:与东亚小经济体不同,中国拥有巨大的国内市场。这使得中国在融入全球化的同时,也保留了相当的战略纵深。国内市场既是消费市场,也是新技术、新产品的试验场,为企业提供了独特的优势。
\end{itemize}

中国的成功,可以看作是一种混合模式的胜利,它既借鉴了EOI的成功经验,又结合了自身的独特国情,走出了一条独一无二的发展道路。

\section{结论:发展的智慧在于适应与选择}

最终,无论选择哪条道路,一个国家的命运,最终还是取决于其\textbf{制度的质量和治理的能力}。一个高效、廉洁、有远见、能够有效动员和配置社会资源的政府,是任何成功发展故事中不可或缺的主角。相反,一个腐败、低效、被特殊利益集团俘获的政府,即使选择了正确的战略方向,也可能在执行中步步走偏,最终一事无成。

为什么东亚能够诞生出高效的“发展型国家”,而许多拉美国家却在民粹主义和腐败的泥潭中挣扎?这背后涉及到深刻的历史、文化和政治制度差异。例如,东亚儒家文化中对教育的重视、对秩序的尊重以及强烈的集体主义精神,可能为其发展提供了文化土壤。而冷战的地缘政治格局,也使得美国出于遏制共产主义的目的,对东亚盟友开放市场并提供了大量援助。这些都是难以简单复制的独特条件。

而当这些不同的发展道路与一个更宏大的历史进程——全球化——相遇时,又会产生怎样复杂的化学反应呢?为什么曾经被全世界精英和民众普遍看好的“全球化”,在今天却似乎遭遇了强劲的“逆流”?谁是全球化的赢家,谁又是输家?更重要的是,在逆全球化浪潮和地缘政治竞争日益激烈的今天,传统的EOI模式是否还能走得通?发展中国家未来的道路,又在何方?

这正是我们第十三章将要深入探讨的,关乎我们每个人当下生活的核心议题。