\chapter{附录A:核心概念词汇表}


本词汇表收录了本书中出现的重要政治学概念,按字母顺序排列,并标注了相关章节。

\section{A}
\begin{description}
    \item[爱国主义 (Patriotism)] 对祖国、家园、文化和同胞的深厚情感和忠诚,侧重于对国家所代表的价值、制度和生活方式的认同与热爱。与民族主义相比,更具包容性,不必然排斥其他国家或民族。 \par\textit{相关章节:第3章}
\end{description}

\section{B}
\begin{description}
    \item[暴力垄断 (Monopoly on the Legitimate Use of Force)] 马克斯·韦伯提出的现代国家核心特征,指国家在其领土范围内拥有合法使用暴力的排他性权力。只有国家才能合法地使用武力,任何个人或团体未经授权使用暴力都是非法的。 \par\textit{相关章节:第1章、第2章}
    \item[比例代表制 (Proportional Representation System)] 一种选举制度,政党在议会中获得的席位比例与其获得的选票比例大致相当。强调公平反映民意,有利于多党制发展,但可能导致政府不稳定。 \par\textit{相关章节:第6章}
    \item[半总统制 (Semi-Presidentialism)] 一种混合型政府体制,既有由全民直接选举产生的总统(国家元首),又有由总统任命但需对议会负责的总理(政府首脑)。法国是典型代表。 \par\textit{相关章节:第4章}
\end{description}

\section{C}
\begin{description}
    \item[参与型文化 (Participant Culture)] 阿尔蒙德和维巴提出的政治文化类型之一,公民积极参与政治,认为自己能够影响政治决策,并对政治体系抱有积极情感。 \par\textit{相关章节:第5章}
    \item[臣民型文化 (Subject Culture)] 政治文化类型之一,公民对政府的输出(如公共服务)有认知和情感,但较少主动参与政治输入(如政策制定),倾向于服从权威。 \par\textit{相关章节:第5章}
    \item[出口导向工业化 (Export-Oriented Industrialization, EOI)] 一种"外向型"发展战略,通过生产具有国际竞争力的产品并积极出口,融入全球市场,从而带动经济增长和工业化。东亚"经济奇迹"的主要策略。 \par\textit{相关章节:第11章}
\end{description}

\section{D}
\begin{description}
    \item[地方型文化 (Parochial Culture)] 政治文化类型之一,公民对政治体系的认知和参与度都非常低,主要关注地方或部族事务,对国家层面政治漠不关心。 \par\textit{相关章节:第5章}
    \item[多数制 (Plurality/Majority System)] 一种选举制度,在单一选区内,获得最多票数的候选人获胜,体现"赢者通吃"逻辑。有利于形成稳定政府和两党制,但可能导致代表性不足。 \par\textit{相关章节:第6章}
\end{description}

\section{F}
\begin{description}
    \item[发展型国家 (Developmental State)] 一种国家主导的经济发展模式,政府通过产业政策、出口导向战略和与企业的密切合作,推动国家经济增长和国际竞争力提升。日韩是典型代表。 \par\textit{相关章节:第9章、第11章}
\end{description}

\section{G}
\begin{description}
    \item[国家 (State)] 拥有明确主权、固定领土,并对其领土内合法暴力拥有垄断权的政治实体。现代国家的核心特征包括主权、领土、暴力垄断和合法性。 \par\textit{相关章节:第1章、第3章}
    \item[国家能力 (State Capacity)] 国家有效执行政策、提供公共服务、维持秩序和实现目标的能力。包括汲取能力、规管能力、提供公共服务能力和强制能力四个方面。 \par\textit{相关章节:第2章}
    \item[国家自主性 (State Autonomy)] 国家机构在多大程度上能够独立于社会各利益集团的压力,制定和执行符合国家整体利益的政策的能力。 \par\textit{相关章节:第2章}
    \item[公民社会 (Civil Society)] 介于国家和家庭之间,由各种非政府、非营利、自愿性组织构成的领域。为公民提供政治参与渠道,监督政府,表达多元利益。 \par\textit{相关章节:第5章}
    \item[规管能力 (Regulatory Capacity)] 国家制定、执行和强制实施法律法规,以规范社会经济活动的能力。包括维护市场秩序、保护产权、执行合同等。 \par\textit{相关章节:第2章}
\end{description}

\section{H}
\begin{description}
    \item[汲取能力 (Extractive Capacity)] 国家从社会中合法地获取资源(主要是税收)的能力。反映国家对经济活动的监管能力和对公民的动员能力。 \par\textit{相关章节:第2章}
    \item[荷兰病 (Dutch Disease)] 资源诅咒的一种表现,指一国发现大量自然资源后,资源出口导致本币升值,使非资源性出口产品失去竞争力,造成经济结构单一化。 \par\textit{相关章节:第10章}
    \item[混合成员比例代表制 (Mixed-Member Proportional Representation)] 一种混合选举制度,选民投两票:一票给选区候选人(多数制),一票给政党名单(比例代表制)。德国是典型代表。 \par\textit{相关章节:第6章}
\end{description}

\section{I}
\begin{description}
    \item[进口替代工业化 (Import Substitution Industrialization, ISI)] 一种"内向型"发展战略,通过在国内生产原本需要进口的工业品,减少对外国制成品的依赖,从而促进本国工业发展。 \par\textit{相关章节:第11章}
    \item[竞争性威权主义 (Competitive Authoritarianism)] 冷战后出现的威权政体形态,形式上存在民主制度,但执政者通过各种手段系统性地压制反对派,确保不公平的政治竞争。 \par\textit{相关章节:第7章}
\end{description}

\section{K}
\begin{description}
    \item[宽容 (Tolerance)] 对不同意见、信仰、生活方式的接受和尊重,尤其是在政治领域对异见的容忍。是多元社会共存和民主运作的基础。 \par\textit{相关章节:第5章}
\end{description}

\section{L}
\begin{description}
    \item[劣质民主/非自由民主 (Illiberal Democracy)] 扎卡利亚提出的概念,指那些虽然举行选举,但却限制公民自由、压制反对派、破坏法治的政体。 \par\textit{相关章节:第5章}
\end{description}

\section{M}
\begin{description}
    \item[民族 (Nation)] 基于共同文化、语言、历史、血缘、宗教或共同命运认同而形成的"想象的共同体",通过共同的叙事、符号和记忆被建构起来。 \par\textit{相关章节:第3章}
    \item[民族主义 (Nationalism)] 强调民族独特性、团结和忠诚,主张民族应该拥有自决权的意识形态和政治运动。既能构建共同体,也可能导致排外与冲突。 \par\textit{相关章节:第3章}
    \item[民主化 (Democratization)] 一个国家从威权统治向民主统治转变的过程,通常涉及引入自由公正的选举、保障公民自由、建立法治等。 \par\textit{相关章节:第8章}
    \item[民主衰退 (Democratic Backsliding)] 一个国家从民主政体向威权政体倒退,或民主质量显著下降的过程,通常通过合法但渐进的方式侵蚀民主制度。 \par\textit{相关章节:第8章}
\end{description}

\section{Q}
\begin{description}
    \item[强制能力 (Coercive Capacity)] 国家通过其军队、警察和司法系统,有效维护社会治安、打击犯罪、应对外部威胁的能力。 \par\textit{相关章节:第2章}
    \item[权变性 (Contingency)] 政治现象的根本特性,指政治结果并非预先注定,而是多种因素在特定时空下相互作用的产物,具有偶然性和不确定性。 \par\textit{相关章节:结语}
\end{description}

\section*{S}
\begin{description}
    \item[社会资本 (Social Capital)] 通过社会网络、互惠规范和信任所形成的集体资源,能够促进合作和集体行动。包括社会网络、互惠规范和信任三个要素。 \par\textit{相关章节:第5章}
    \item[社会运动 (Social Movements)] 一群具有共同目标和价值观的个体或组织,通过集体行动来推动、阻止或抵制社会变革的努力。 \par\textit{相关章节:第16章}
    \item[身份政治 (Identity Politics)] 基于共同的群体身份(如种族、民族、宗教、性别等)而形成的政治动员和政治诉求,强调特定群体的独特经验和权利。 \par\textit{相关章节:第14章}
\end{description}

\section{T}
\begin{description}
    \item[提供公共服务能力 (Provisioning Capacity)] 国家向公民提供教育、医疗、基础设施、社会保障等公共产品和服务的能力。 \par\textit{相关章节:第2章}
\end{description}

\section{W}
\begin{description}
    \item[威权政体 (Authoritarianism)] 政治权力集中于少数人手中,公民政治参与受到严格限制,缺乏自由公正的选举和有效问责机制的政体。 \par\textit{相关章节:第7章}
    \item[威斯特伐利亚体系 (Westphalian System)] 1648年《威斯特伐利亚和约》确立的国际体系,基于主权原则、领土原则和不干涉内政原则,奠定了现代国际关系的基础。 \par\textit{相关章节:第1章}
\end{description}

\section{X}
\begin{description}
    \item[寻租 (Rent-seeking)] 通过政治影响力或特殊关系,不创造实际财富而获取经济利益的行为。常见于资源丰富或政府干预较多的国家。 \par\textit{相关章节:第7章、第10章、第11章}
    \item[选举民主 (Electoral Democracy)] 最基础的民主形式,主要特征是定期举行自由而公正的选举,允许公民选择领导人,但可能在公民自由和法治方面存在不足。 \par\textit{相关章节:第5章}
\end{description}

\section{Y}
\begin{description}
    \item[议会制 (Parliamentary System)] 一种政府体制,行政权与立法权紧密结合,政府从议会中产生并对议会负责。总理是政府首脑,国家元首通常是虚位。 \par\textit{相关章节:第4章}
\end{description}

\section{Z}
\begin{description}
    \item[政治文化 (Political Culture)] 一个社会中普遍存在的,对政治体系、政治角色和政治行为的认知、情感和评价模式。包括认知、情感和评价三个层面。 \par\textit{相关章节:第5章}
    \item[政治机会结构 (Political Opportunity Structure)] 外部政治环境对社会运动兴起和发展的影响,包括政治体系的开放程度、精英分裂情况、政府镇压能力等因素。 \par\textit{相关章节:第16章}
    \item[政治经济模式 (Political Economy Model)] 处理市场与政府、效率与公平关系的不同制度安排。主要包括自由主义、社会民主主义和发展型国家三种模式。 \par\textit{相关章节:第9章}
    \item[主权 (Sovereignty)] 国家在其领土范围内拥有至高无上的权力,不受任何外部或内部势力干涉。是现代国家最根本的属性。 \par\textit{相关章节:第1章}
    \item[主权财富基金 (Sovereign Wealth Fund)] 由国家控制的投资基金,通常用于管理自然资源收入或外汇储备,进行长期投资以实现财富保值增值。挪威是成功典范。 \par\textit{相关章节:第10章}
    \item[总统制 (Presidential System)] 一种政府体制,行政权与立法权严格分离,总统既是国家元首又是政府首脑,由全民选举产生,任期固定。 \par\textit{相关章节:第4章}
    \item[资源诅咒 (Resource Curse)] 一个国家拥有丰富的自然资源,但这些资源反而阻碍了其经济多元化发展、民主化进程和良善治理的现象。 \par\textit{相关章节:第10章}
    \item[自由民主 (Liberal Democracy)] 在选举民主基础上,进一步强调对公民自由和权利的充分保障,以及健全的法治、权力制衡机制。 \par\textit{相关章节:第5章}
\end{description}

\hrulefill

\begin{quote}
\textbf{使用说明:}
\begin{itemize}
    \item 概念后的相关章节标注了该概念首次出现或重点讨论的章节
    \item 如需深入理解,请参阅相关章节的详细论述
    \item 本词汇表仅提供基础定义,政治学概念往往具有争议性,不同学者可能有不同理解
\end{itemize}
\end{quote}
