\chapter{附录B:延伸阅读建议}

本阅读清单按照本书各个主题进行分类,为有兴趣深入学习的读者提供进一步的阅读指导。书目包括经典著作、重要研究和优秀的中文译作。

\section{国家理论与国家建设}

\subsection{经典著作}
\begin{itemize}
    \item \textbf{马克斯·韦伯}:《经济与社会》(Economy and Society)
    \quad - 现代国家理论的奠基之作,提出国家暴力垄断的经典定义
    \item \textbf{查尔斯·蒂利}:《强制、资本和欧洲国家(公元990-1992年)》(Coercion, Capital and European States)
    \quad - 从历史角度分析欧洲国家形成过程
    \item \textbf{西达·斯考切波}:《找回国家》(Bringing the State Back In)
    \quad - 重新强调国家在政治分析中的中心地位
\end{itemize}

\subsection{当代重要研究}
\begin{itemize}
    \item \textbf{弗朗西斯·福山}:《政治秩序的起源》(The Origins of Political Order)
    \quad - 从人类学角度探讨政治制度的演化
    \item \textbf{达龙·阿西莫格鲁 \& 詹姆斯·罗宾逊}:《国家为什么会失败》(Why Nations Fail)
    \quad - 从制度经济学角度分析国家兴衰
    \item \textbf{詹姆斯·斯科特}:《国家的视角》(Seeing Like a State)
    \quad - 批判性分析现代国家的治理逻辑
\end{itemize}

\subsection{中文译作推荐}
\begin{itemize}
    \item 《国家与社会革命》(西达·斯考切波)
    \item 《暴力与社会秩序》(诺斯、沃利斯、韦因加斯特)
    \item 《政治发展的经济分析》(巴罗)
\end{itemize}

\section{民主理论与威权研究}

\subsection{民主理论经典}
\begin{itemize}
    \item \textbf{罗伯特·达尔}:《论民主》(On Democracy)
    \quad - 当代民主理论的集大成之作
    \item \textbf{约瑟夫·熊彼特}:《资本主义、社会主义与民主》(Capitalism, Socialism and Democracy)
    \quad - 提出程序性民主定义的开创性著作
    \item \textbf{阿历克西·德·托克维尔}:《论美国的民主》(Democracy in America)
    \quad - 观察民主社会运作的经典文本
\end{itemize}

\subsection{威权研究}
\begin{itemize}
    \item \textbf{胡安·林茨}:《威权政体》(Authoritarian Brazil)
    \quad - 威权政体类型学的奠基之作
    \item \textbf{莱维茨基 \& 齐布拉特}:《民主如何死亡》(How Democracies Die)
    \quad - 分析当代民主衰退现象
    \item \textbf{拉里·德比}:《恋恋红尘:当代中国的滋生政治》
    \quad - 分析当代威权政体的适应性
\end{itemize}

\subsection{中文译作推荐}
\begin{itemize}
    \item 《第三波:20世纪后期民主化浪潮》(塞缪尔·亨廷顿)
    \item 《民主的细节》(刘瑜)
    \item 《历史的终结与最后的人》(弗朗西斯·福山)
\end{itemize}

\section{选举制度与政党政治}

\subsection{选举制度研究}
\begin{itemize}
    \item \textbf{阿伦·利帕特}:《民主的模式》(Patterns of Democracy)
    \quad - 比较不同民主制度设计的经典著作
    \item \textbf{莫里斯·杜弗尔热}:《政党论》(Political Parties)
    \quad - 提出著名的"杜弗尔热定律"
    \item \textbf{派琵亚·诺里斯}:《选举制度与政党制度》(Electoral Systems and Party Systems)
    \quad - 系统分析选举制度的政治后果
\end{itemize}

\subsection{政党研究}
\begin{itemize}
    \item \textbf{罗贝托·米歇尔斯}:《寡头统治铁律》(Political Parties)
    \quad - 分析政党组织的寡头化趋势
    \item \textbf{卡茨 \& 梅尔}:《党派政治的变化性质》(The Changing Nature of Party Politics)
    \quad - 分析当代政党政治的新特征
\end{itemize}

\subsection{中文译作推荐}
\begin{itemize}
    \item 《政党与政党制度》(萨托里)
    \item 《代议制政府》(密尔)
    \item 《多数人的暴政》(托克维尔)
\end{itemize}

\section{政治文化与公民社会}

\subsection{政治文化经典}
\begin{itemize}
    \item \textbf{阿尔蒙德 \& 维巴}:《公民文化》(The Civic Culture)
    \quad - 政治文化研究的开创性著作
    \item \textbf{罗纳德·英格尔哈特}:《文化转变》(Culture Shift)
    \quad - 分析后工业社会的价值观变迁
    \item \textbf{罗伯特·帕特南}:《使民主运转起来》(Making Democracy Work)
    \quad - 社会资本理论的重要贡献
\end{itemize}

\subsection{公民社会研究}
\begin{itemize}
    \item \textbf{科恩 \& 阿拉托}:《公民社会与政治理论》(Civil Society and Political Theory)
    \quad - 公民社会概念的系统阐述
    \item \textbf{迈克尔·埃德华兹}:《公民社会》(Civil Society)
    \quad - 公民社会理论的批判性综述
\end{itemize}

\subsection{中文译作推荐}
\begin{itemize}
    \item 《独自打保龄球》(罗伯特·帕特南)
    \item 《民主的习惯》(帕特南)
    \item 《社会资本》(詹姆斯·科尔曼)
\end{itemize}

\section{政治经济学}

\subsection{比较政治经济学经典}
\begin{itemize}
    \item \textbf{霍尔 \& 索斯基奇}:《资本主义的多样性》(Varieties of Capitalism)
    \quad - 当代比较政治经济学的里程碑
    \item \textbf{亚当·普泽沃斯基}:《民主与市场》(Democracy and the Market)
    \quad - 分析政治转型中的经济因素
    \item \textbf{彼得·埃文斯}:《嵌入式自主性》(Embedded Autonomy)
    \quad - 发展型国家理论的重要贡献
\end{itemize}

\subsection{发展政治学}
\begin{itemize}
    \item \textbf{道格拉斯·诺斯}:《制度、制度变迁与经济绩效》(Institutions, Institutional Change and Economic Performance)
    \quad - 新制度主义经济学的代表作
    \item \textbf{沃尔特·罗斯托}:《经济增长的阶段》(The Stages of Economic Growth)
    \quad - 现代化理论的经典文本
\end{itemize}

\subsection{中文译作推荐}
\begin{itemize}
    \item 《国富论》(亚当·斯密)
    \item 《经济发展理论》(熊彼特)
    \item 《制度经济学》(康芒斯)
\end{itemize}

\section{民主化与政治转型}

\subsection{转型研究经典}
\begin{itemize}
    \item \textbf{奥唐奈 \& 施密特}:《从威权统治的转型》(Transitions from Authoritarian Rule)
    \quad - 转型学研究的奠基之作
    \item \textbf{林茨 \& 斯泰潘}:《民主转型与巩固的问题》(Problems of Democratic Transition and Consolidation)
    \quad - 系统分析民主巩固的条件
    \item \textbf{詹姆斯·鲁宾逊}:《经济起源的独裁与民主》(Economic Origins of Dictatorship and Democracy)
    \quad - 从经济角度分析政治转型
\end{itemize}

\subsection{当代转型研究}
\begin{itemize}
    \item \textbf{托马斯·卡罗瑟斯}:《转型范式的终结》(The End of the Transition Paradigm)
    \quad - 对转型理论的批判性反思
    \item \textbf{列维茨基 \& 韦伊}:《竞争性威权主义》(Competitive Authoritarianism)
    \quad - 分析冷战后的混合政体
\end{itemize}

\subsection{中文译作推荐}
\begin{itemize}
    \item 《第三波》(塞缪尔·亨廷顿)
    \item 《民主转型与巩固》(林茨、斯泰潘)
    \item 《威权韧性》(安德鲁·内森)
\end{itemize}

\section{身份政治与社会运动}

\subsection{身份政治研究}
\begin{itemize}
    \item \textbf{本尼迪克特·安德森}:《想象的共同体》(Imagined Communities)
    \quad - 民族主义研究的经典之作
    \item \textbf{埃里克·霍布斯鲍姆}:《民族与民族主义》(Nations and Nationalism Since 1780)
    \quad - 历史学视角的民族主义分析
    \item \textbf{弗朗西斯·福山}:《身份》(Identity)
    \quad - 当代身份政治的系统分析
\end{itemize}

\subsection{社会运动研究}
\begin{itemize}
    \item \textbf{西德尼·塔罗}:《运动中的力量》(Power in Movement)
    \quad - 社会运动理论的集大成之作
    \item \textbf{麦卡锡 \& 扎尔德}:《资源动员与社会运动》(Resource Mobilization and Social Movements)
    \quad - 资源动员理论的重要文献
    \item \textbf{查尔斯·蒂利}:《社会运动》(Social Movements)
    \quad - 社会运动的历史社会学分析
\end{itemize}

\subsection{中文译作推荐}
\begin{itemize}
    \item 《民族主义》(盖尔纳)
    \item 《社会运动概论》(德拉波特、迈耶)
    \item 《抗争政治》(查尔斯·蒂利)
\end{itemize}

\section{比较方法与研究设计}

\subsection{方法论经典}
\begin{itemize}
    \item \textbf{阿伦·利帕特}:《比较政治学的研究方法》(Comparative Politics and the Comparative Method)
    \quad - 比较政治学方法论的重要文献
    \item \textbf{基恩·金}:《社会科学中的研究设计》(Designing Social Inquiry)
    \quad - 定量与定性方法的结合
    \item \textbf{戴维·拉金}:《比较方法》(The Comparative Method)
    \quad - 小样本比较研究的方法指南
\end{itemize}

\subsection{案例研究方法}
\begin{itemize}
    \item \textbf{约翰·杰林}:《案例研究研究》(Case Study Research)
    \quad - 案例研究方法的系统阐述
    \item \textbf{乔治 \& 贝内特}:《社会科学的案例研究》(Case Studies and Theory Development)
    \quad - 案例研究在理论建构中的作用
\end{itemize}

\subsection{中文译作推荐}
\begin{itemize}
    \item 《社会科学研究方法》(艾尔·巴比)
    \item 《比较政治学:理论与方法》(燕继荣)
    \item 《政治学研究方法》(约翰逊、雷诺兹)
\end{itemize}

\section{区域研究专题}

\subsection{欧洲政治}
\begin{itemize}
    \item \textbf{阿伦·利帕特}:《多数民主与共识民主》(Democracies)
    \quad - 欧洲民主制度的比较分析
    \item \textbf{赫伯特·基钦斯凯尔特}:《欧洲政党制度的转变》(The Transformation of European Social Democracy)
    \quad - 欧洲社会民主主义的演变
\end{itemize}

\subsection{发展中国家政治}
\begin{itemize}
    \item \textbf{詹姆斯·斯科特}:《农民的道义经济》(The Moral Economy of the Peasant)
    \quad - 发展中国家农村政治的经典分析
    \item \textbf{布鲁斯·海顿}:《比较政治中的发展问题》(Comparative Politics in Developing Countries)
    \quad - 发展中国家政治的系统分析
\end{itemize}

\subsection{中文译作推荐}
\begin{itemize}
    \item 《发展政治学》(阿尔蒙德)
    \item 《第三世界政治学》(查德威克)
    \item 《转型中的政治》(亨廷顿)
\end{itemize}

\section{当代热点专题}

\subsection{数字时代政治}
\begin{itemize}
    \item \textbf{曼纽尔·卡斯特}:《网络社会的崛起》(The Rise of the Network Society)
    \quad - 信息时代的政治社会变迁
    \item \textbf{叶夫根尼·莫罗佐夫}:《网络幻象》(The Net Delusion)
    \quad - 对数字乐观主义的批判
\end{itemize}

\subsection{全球化与治理}
\begin{itemize}
    \item \textbf{大卫·赫尔德}:《全球化与政治》(Global Transformations)
    \quad - 全球化对政治的系统性影响
    \item \textbf{詹姆斯·罗森瑙}:《没有政府的治理》(Governance without Government)
    \quad - 全球治理理论的重要贡献
\end{itemize}

\subsection{中文译作推荐}
\begin{itemize}
    \item 《全球化与政治》(大卫·赫尔德)
    \item 《数字化生存》(尼古拉斯·尼葛洛庞帝)
    \item 《网络社会》(曼纽尔·卡斯特)
\end{itemize}

\hrulefill

\section{阅读建议}

\subsection{对于初学者}
\begin{itemize}
    \item 先阅读综合性教科书,建立整体框架
    \item 选择感兴趣的专题深入阅读
    \item 结合现实案例验证理论
\end{itemize}

\subsection{对于进阶学习者}
\begin{itemize}
    \item 关注经典著作,理解理论发展脉络
    \item 跟踪前沿研究,了解学科动态
    \item 培养批判性思维,比较不同观点
\end{itemize}

\subsection{对于研究者}
\begin{itemize}
    \item 掌握方法论基础,提高研究能力
    \item 关注跨学科对话,拓宽视野
    \item 重视实证研究,检验理论假设
\end{itemize}

\hrulefill

\textbf{注:}
\begin{itemize}
    \item 本书目不求穷尽,重在精选具有代表性的重要著作
    \item 中英文书目并重,便于不同语言背景的读者选择
    \item 建议读者根据自己的兴趣和需求有选择性地阅读
    \item 部分著作可能有多个中文译本,建议选择权威出版社的版本
\end{itemize}