\part{舞台与演员——国家、民族与政权}

\chapter{为什么说我们今天熟悉的“国家”,其实是个“现代”发明?}

\section{清晨的“国家”:一次日常生活的政治学解剖}

闹钟响起,你从睡梦中醒来,开始新的一天。你可能从未想过,在你睁开眼的那一刻,你就已经沉浸在一个巨大而无形的网络之中。你拧开水龙头,清澈的自来水流出——这背后是一个庞大的公共事业系统在规划、净化和输送,它的运作标准由一个名为“国家”的机构设定。你走进厨房,打开冰箱,里面的食物都贴着标签,标明了成分、产地和保质期,这是国家食品安全法规在保护你的健康。你一边吃着早餐,一边打开电视或手机浏览新闻,无论是报道国内政策还是国际冲突,新闻的框架、主持人的措辞,甚至哪些新闻能被你看到,都或多或少受到了国家广播电视管理条例和信息政策的影响。

吃完早餐,你出门上班。你驾驶的汽车,其生产标准、安全配置、牌照发放,无一不受到国家交通部门的严格监管。你行驶在平坦的公路上,这是国家税收投资建设的公共基础设施。路口的红绿灯,是国家强制力在微观层面的体现,它以一种和平的方式协调着成千上万陌生人的行动,避免混乱。你用来加油的货币,其价值由国家中央银行的信誉所担保。你抵达办公室,开始一天的工作,你与雇主签订的劳动合同,你的工作时长、最低工资、休假权利,都受到国家劳动法的保护。

在这一连串看似寻常的日常活动中,“国家”如影随形。它像空气一样,无处不在,却又常常被我们忽略。它似乎是天经地义、自古以来就存在的政治单元。我们每天接触的新闻、我们持有的护照、我们缴纳的税款,无不与“国家”紧密相连。然而,作为政治科学家,我们必须对这种“常识”保持审慎和好奇。如果我们将历史的镜头拉远,穿越回几百甚至上千年前,我们会惊讶地发现,我们今天所理解的这种拥有明确地理边界、统一法律体系、至高无上权威和排他性权力的“国家”,在人类漫长的政治历史中,其实是一个相当“年轻”、相当“现代”的产物。

那么,在现代国家登上历史舞台之前,人类社会是如何组织起来的呢?那时的政治舞台又是怎样一番景象?理解这一点,是解构我们习以为常的政治世界、开启比较政治学之旅的第一步。本章的目的,就是要带领大家进行一次思想上的“时空穿越”,去探索现代国家诞生前那片广阔而陌生的政治图景,并见证我们今天熟悉的这个“利维坦”是如何在特定的历史条件下被“发明”出来的。

\hrulefill

\section{前现代的政治剧场:帝国、城邦与封建网络}

想象一下,让我们将时间拨回到公元1000年左右。此时,现代国家尚未诞生,世界各地的政治组织形态五花八门,与我们今天的认知截然不同。主要的组织形式是帝国、城邦以及欧洲独特的封建体系。它们共同构成了一个权力分散、边界模糊、忠诚多元的前现代政治剧场。

\subsection{帝国:广袤但模糊的权力梯度}

帝国是前现代世界最常见的宏大政治体,比如我们熟悉的罗马帝国、中华帝国、奥斯曼帝国或神圣罗马帝国。它们幅员辽阔,统治着多个民族、语言和文化群体。但帝国的统治逻辑与现代国家有着根本性的区别。

\begin{itemize}
    \item \textbf{模糊流动的边界:} 现代国家的边界是清晰的、划在地图上的“线”,而帝国的边界更像是一个模糊的“区域”或“地带”。以罗马帝国为例,其在欧洲的北部边界——莱茵河与多瑙河防线,并非一条密不透风的墙,而是一个由堡垒、哨所和巡逻队构成的军事化区域。边界内外的居民、商旅、部落时常往来,帝国的控制力随着远离中心而逐渐减弱,形成一个权力梯度。你很难精确地说出,在某一天的某一刻,帝国的权威究竟在哪里戛然而止。
    \item \textbf{多元重叠的法律与治理:} 帝国通常不会试图将一套统一的法律和行政体系强加给所有被征服的地区。相反,它们常常采取更为务实的“间接统治”。罗马帝国允许地方保留原有的习俗、宗教和部分法律,只要他们效忠皇帝、缴纳税款并维持秩序。在奥斯曼帝国,著名的“米利特”制度允许非穆斯林社群(如希腊东正教徒、亚美尼亚基督徒、犹太教徒)在各自的宗教领袖管理下,享有高度的司法和文化自治。这意味着,一个生活在伊斯坦布尔的希腊人,其婚姻、继承等民事纠纷是由东正教法庭而非帝国法庭裁决的。这与现代国家法律面前人人平等、司法管辖权统一的原则大相径庭。
    \item \textbf{个人化、多层次的忠诚:} 在帝国,忠诚更多是针对具体的统治者(皇帝、苏丹、国王)或王朝,而非一个抽象的、法理上的“国家”概念。人们效忠于凯撒,而非“罗马国”;效忠于李氏王朝,而非“大唐国”。这种忠诚是个人化的、垂直的。同时,一个人的忠诚对象可以是多层次的,他既效忠于遥远的皇帝,也更直接地效忠于本地的总督、族长或宗教领袖。
    \item \textbf{权力网络的中心与边缘:} 帝国的统治更像是一个以首都为中心的、庞大的、等级化的网络,而非一个拥有统一、排他性主权的实体。权力从中心向边缘层层递减,中央政府对边远省份的控制力往往十分有限,地方精英拥有相当大的自治权。
\end{itemize}

\subsection{城邦:紧密但脆弱的政治共同体}

与帝国形成鲜明对比的是城邦,它们规模小巧,以城市为中心,是前现代世界另一种重要的政治形态。

\begin{itemize}
    \item \textbf{古希腊的城邦:} 比如雅典和斯巴达。雅典以其“民主”而闻名,公民(仅限成年男性公民)可以直接参与城邦大会,对战争、法律、人事等重大议题进行投票。政治生活是紧密的、面对面的,公民身份既是权利也是责任。然而,这种紧密的共同体也是高度排外的。妇女、奴隶、外邦人被排除在政治生活之外,他们虽然生活在雅典,却不是城邦的一份子。此外,城邦之间各自为政,拥有自己的军队、法律和神明,彼此间冲突不断,难以形成跨区域的稳定秩序。希腊世界最终在伯罗奔尼撒战争中两败俱伤,并被北方的马其顿王国所征服,正暴露了城邦体系在应对外部挑战时的脆弱性。
    \item \textbf{文艺复兴时期的意大利城邦:} 比如威尼斯、佛罗伦萨和热那亚。这些城邦是当时欧洲的商业和金融中心。威尼斯凭借其强大的海军和贸易网络,建立了一个地中海商业帝国,由一个封闭的商人贵族阶层统治。佛罗伦萨在美第奇家族的领导下,成为了文艺复兴的摇篮,其政治权力与银行家的财富紧密相连。这些城邦拥有复杂的官僚机构、外交使节和税收系统,在某些方面已经具备了现代国家的雏形。但它们的本质仍然是城市共同体,其权力基础是商业网络和财富,而非对一片广阔连续领土的绝对主权。它们的身份认同是“威尼斯人”或“佛罗伦萨人”,而非“意大利人”。
\end{itemize}

\subsection{欧洲的封建体系:碎片化的权力网络}

中世纪欧洲的封建体系,是理解现代国家起源时一个至关重要的参照物。它是一种高度碎片化、个人化的权力结构。

\begin{itemize}
    \item \textbf{权力碎片化:} 在封建体系下,权力并非集中于国王一人之手,而是像金字塔一样层层分封下去。国王将土地(封地)授予大贵族(公爵、伯爵),大贵族再将土地分封给小贵族(男爵、骑士)。每一级领主在其封地内都拥有独立的行政、司法和税收权力。一个农民首先效忠的是其直接的庄园领主,而非遥远的国王。国王的权力实际上非常有限,他更像是一个“领主中的领主”,而非一个国家的最高统治者。
    \item \textbf{权力个人化:} 权力关系建立在领主与附庸之间一对一的、个人化的效忠契约之上。附庸向领主宣誓效忠,承诺为其提供军事服务(如每年服役40天);领主则向附庸提供土地和保护。这种关系是契约性的、私人的,而非制度化的、公共的。
    \item \textbf{司法与军事的分散:} 法律和正义是地方性的。每个庄园都有自己的法庭,审理领地内的案件,领主即法官。军事力量也是分散的,没有统一的国家常备军,国王需要打仗时,必须召集其附庸,附庸再带来自己的士兵,形成一支临时拼凑的封建军队。
\end{itemize}

无论是帝国、城邦还是封建体系,它们都缺乏现代国家最核心、最决定性的几个特征:\textbf{明确且固定的领土、不容挑战的最高主权、以及对合法暴力的垄断}。在前现代世界,权力是分散的、重叠的,忠诚是多层次的,界限是模糊的。一个中世纪的农民,可能既要向本地领主纳税,又要向教会缴纳什一税,还要在国王需要时服兵役,他的生活中存在着多个权力中心。这正是现代国家所要彻底颠覆的政治图景。

\hrulefill

\section{现代国家的登场:威斯特伐利亚体系(1648)}

历史的车轮滚滚向前,欧洲在16世纪的宗教改革后,陷入了长达一个多世纪的血腥冲突。天主教与新教之间的对立,与王室、贵族之间的权力斗争交织在一起,最终在17世纪酿成了一场空前惨烈的浩劫——“三十年战争”(1618-1648)。

这场战争的残酷性难以想象。它始于神圣罗马帝国内部的宗教冲突,却迅速演变为一场全欧洲范围的大混战。瑞典、法国、西班牙、丹麦等国纷纷卷入,战争的目的早已超越了宗教,变成了赤裸裸的领土和霸权争夺。战争的主力并非各国统一的军队,而是由战争承包商(如著名的华伦斯坦)招募的雇佣兵。这些军队没有固定的后勤,以战养战,所到之处烧杀抢掠,给中欧地区带来了毁灭性的打击。据估计,德意志地区的人口在战争中减少了近三分之一。

战争的惨痛教训,促使精疲力竭的欧洲统治者们开始寻求一种新的、能够避免类似灾难重演的政治秩序。《威斯特伐利亚和约》正是在这样的背景下,于1648年在威斯特伐利亚地区的明斯特和奥斯纳布吕克两个城市签订的。这份和约并非一份单一文件,而是一系列条约的总称。它通常被视为现代国际体系和现代国家诞生的标志性事件。尽管它并非一夜之间创造了现代国家,但它确立了一些颠覆性的关键原则,为现代国家的兴起奠定了法理基础:

\begin{enumerate}
    \item \textbf{主权原则:} 这是和约最核心、最革命性的贡献。它确立了“国家主权”的概念,承认了各签约方(包括神圣罗马帝国内的数百个邦国)在其领土内的最高权威。这意味着,外部势力——特别是罗马教皇和神圣罗马帝国皇帝——无权干涉一个国家的内部事务,尤其是宗教事务(重申了“教随国定”原则)。这在法理上终结了教权高于王权的时代,也瓦解了神圣罗马帝国作为普世帝国的权威。\textbf{主权,意味着国家对内拥有至高无上的、不容分割的权力,对外拥有独立自主、不受干涉的地位。}
    \item \textbf{领土原则(Territory):} 和约通过一系列复杂的领土划分,强调了国家是建立在明确划定、固定不变的地理区域之上的。这与前现代政治实体模糊、变动的边界形成了鲜明对比。固定的领土边界,为现代地图的绘制、边境的控制、关税的征收以及统一的行政管理提供了基础。权力不再是弥散的,而是被清晰地限定在地理空间之内。
    \item \textbf{平等与不干涉内政原则:} 基于主权和领土原则,和约隐含了承认各国在法理上的平等地位,以及在自身事务上的自主权,不应相互干涉。这为后来的现代国际法、外交关系和集体安全理念奠定了理论基石。一个国家无论大小强弱,在主权上都是平等的。
\end{enumerate}

威斯特伐利亚和约就像一声发令枪,标志着一种新的政治游戏规则开始形成。它宣告了中世纪那种权力交错、忠诚多元的普世主义秩序的终结,开启了一个由主权独立、领土明确、地位平等的国家所构成的全新时代。我们今天熟悉的“现代国家”,正是在这个舞台上,作为主角正式登场的。

\hrulefill

\section{现代国家的核心特征:一套全新的“政治操作系统”}

在威斯特伐利亚体系的影响下,经过之后几个世纪的演化和巩固(尤其是在法国大革命和19世纪民族主义浪潮的推动下),现代国家逐渐发展出以下几个彼此关联、相互支撑的核心特征。它们共同构成了一个区别于以往所有政治形态的、全新的“政治操作系统”。

\subsection{主权:最高且唯一的“管理员权限”}

这是现代国家最最根本的属性,是其一切权力的合法性来源。你可以将主权理解为国家这台电脑的“最高管理员权限”,它拥有对系统内所有文件和程序的最终控制权,且这个权限是唯一的、不容分享的。

\begin{itemize}
    \item \textbf{对内最高性:} 在其领土范围内,国家的权力是至高无上的。任何其他组织或个人——无论是教会、公司、工会还是地方领主——都不能挑战其权威。国家制定法律,所有人都必须遵守。这种最高性,确保了国家能够建立统一的法律和行政秩序。
    \item \textbf{对外独立性:} 在国际关系中,主权意味着国家是独立的、平等的行为者,不受任何外部势力的支配。每个国家都有权自主决定其内外政策,其他国家无权干涉。
\end{itemize}

\textbf{挑战与现实:} 当然,主权在现实中是一个不断受到挑战和重新定义的概念。在全球化时代,跨国公司、国际组织(如欧盟、联合国)、国际法和全球性问题(如气候变化、网络攻击)都在一定程度上“侵蚀”着传统的威斯特伐利亚主权。一个国家加入世界贸易组织(WTO),就意味着它愿意接受该组织的规则约束,其贸易政策的自主性会受到限制。但这并不意味着主权消失了,而是国家为了获取更大利益(如进入全球市场)而自愿让渡了部分主权权力。然而,其作为主权国家的根本地位并未改变。

\subsection{领土:权力行使的“硬盘空间”}

现代国家是领土性的政治实体,其权力有效行使的范围由固定且可识别的地理边界清晰限定。这就像你的电脑有一个明确的硬盘空间,所有数据都储存在这个空间之内,并由你全权管理。

\begin{itemize}
    \item \textbf{固定性与排他性:} 领土是固定的,其边界通过条约和勘界被精确地确定下来。国家对其领土拥有排他性的管辖权,能够对这片土地上的人口、资源和活动进行统一管理、征税和规管。
    \item \textbf{功能:} 明确的领土是国家能力的基础(我们将在下一章深入探讨)。它使得国家能够进行人口普查、资源勘探、征兵、建设基础设施,并建立起有效的边境控制和海关系统。
\end{itemize}

\subsection{暴力垄断:秩序的最终保障}

德国伟大的社会学家马克斯·韦伯曾给现代国家下过一个经典定义:“\textbf{国家是那个在特定疆域之内,成功地垄断了对正当武力之使用权的属人组织。}” 这个定义极其深刻,抓住了现代国家区别于其他所有权力组织(如黑手党、公司、教会)的关键所在。

\begin{itemize}
    \item \textbf{关键词一:“垄断”(Monopoly):} 在一个健康的现代国家里,只有国家及其授权的机构(如军队、警察、法院)才能合法地使用或威胁使用武力。任何未经国家授权的暴力行为——无论是个人斗殴、黑帮火并还是私人武装——都是非法的,将受到国家机器的打击。这种垄断,终结了中世纪封建贵族拥有私人军队、可以合法地相互攻伐的局面。
    \item \textbf{关键词二:“合法地”(Legitimate):} 这点至关重要。国家并非垄断了所有暴力,而是垄断了\textbf{合法的}暴力。一个警察在执法过程中使用必要的武力被视为合法,而一个匪徒使用同样的武力则被视为犯罪。这种合法性来自于法律的授权和民众的普遍接受。国家对暴力的垄断,是其维持社会秩序、提供公共安全、执行法律判决的最终保障。试想一下,如果社会中存在多个可以合法使用暴力的团体,那将是何等混乱?
\end{itemize}

\textbf{现实的挑战:} 在许多“弱国家”或“失败国家”,政府恰恰是失去了对暴力的垄断。例如,在某些地区的哥伦比亚或墨西哥,强大的贩毒集团拥有自己的武装力量,公然挑战甚至在局部地区压倒了政府军警,形成了事实上的“双重权力”结构。在索马里或也门,各种军阀、部族武装和恐怖组织控制着不同地区,国家对暴力的垄断完全崩溃。这些案例反过来证明了暴力垄断对于维持一个有效国家是何等重要。

\subsection{合法性:发自内心的认同}

一个稳固的现代国家,其统治绝不仅仅依靠暴力强制。如果一个政权只能靠枪杆子来让民众服从,那么它的统治成本会极高,且极其脆弱。更重要的是,它需要其统治被民众普遍接受和认可,即拥有\textbf{合法性}。合法性是一种心理现象,是民众发自内心地认为“这个政府有权统治我们,我们应该服从它”。

韦伯同样提出了三种经典的合法性来源:
\begin{itemize}
    \item \textbf{传统型合法性:} 基于对古老传统和习俗的神圣性的信仰。人们服从统治者,是因为“自古以来就是如此”。例如,世袭君主制国家的合法性就来源于此。
    \item \textbf{魅力型合法性:} 基于对某个领导人非凡的、神圣的或英雄般个人品质的追随和信赖。这种合法性高度依赖于领袖个人的魅力,如革命领袖或民族英雄。
    \item \textbf{法理型合法性:} 基于对法律、规则和程序的公正性的信仰。人们服从的不是某个具体的人,而是法律和职位本身。人们相信统治者是通过合法的程序上台的,其权力在法律框架内行使。这是现代国家最主要、最稳定的合法性来源。通过定期选举、宪法约束和法治运作的民主国家,其合法性正是建立在法理基础之上。
\end{itemize}

此外,当代政治学还常常讨论\textbf{绩效合法性},即政府通过提供良好的公共服务、实现经济高速增长、提升国民生活水平来赢得民众的支持和认同。

\subsection{官僚机构:高效运转的“发动机”}

如果说主权是操作系统,领土是硬盘,那么官僚机构就是运行这一切的硬件和软件。现代国家事务庞杂,从征税、教育、卫生到国防、外交,无法依靠个人或松散的组织来管理。它需要一套理性、高效、专业化的行政管理体系,即\textbf{官僚机构}。

根据韦伯的理想模型,现代官僚机构具有以下特征:
\begin{itemize}
    \item \textbf{层级分明(Hierarchy):} 清晰的指挥链和权力等级。
    \item \textbf{专业分工(Specialization):} 每个部门和官员都有明确的职责范围。
    \item \textbf{规则导向(Rule-based):} 依照成文的法律和规章办事,而非个人好恶。
    \item \textbf{非人格化(Impersonality):} 对事不对人,公平对待所有公民。
    \item \textbf{专家治国(Expertise):} 官员根据其专业知识和技能被录用和晋升。
\end{itemize}

一个高效、廉洁、专业的官僚机构,是现代国家能够有效运转的“发动机”,是国家能力的核心体现。相反,一个腐败、臃肿、低效的官僚体系,则会严重侵蚀国家能力,导致政策无法执行,公共服务瘫痪。

\hrulefill

\section{结论与展望:从“是什么”到“强不强”}

通过本章的探索,我们完成了一次关键的“祛魅”过程。我们看到,今天我们习以为常的“国家”,并非永恒不变的自然产物,而是人类社会在特定的历史条件下,为了更有效地组织社会、管理资源、维护秩序而逐步演化出的一个“现代”发明。它以主权、领土、暴力垄断、合法性和官僚机构这五大特征,与古代的帝国、城邦以及其他前现代政治实体彻底区分开来。

理解这一点,是理解整个比较政治学的起点。因为它提醒我们,政治的形态并非一成不变,而是历史、文化和权力博弈的产物。我们所比较的,正是这些在不同时空背景下演化出的、形态各异的“现代国家”。

\textbf{教授的小贴士:}
在你生活的国家,这五个核心特征是如何体现的?主权是否完整?领土是否明确?国家对暴力的垄断是否牢固?政府的合法性主要来源于什么?官僚机构的运作效率如何?有没有哪些方面似乎比较“弱”?为什么?这正是我们接下来要深入探讨“国家能力”的基础。

然而,一个现代国家仅仅是“存在”还不够,它还需要具备“能力”才能有效运作。为什么有些国家能够有效治理,提供世界一流的公共服务,维持令人羡慕的社会秩序,而另一些国家却深陷混乱,政府职能瘫痪,甚至连公民最基本的安全都无法保障?这已经不再是“国家是什么”的问题,而是“国家强不强”的问题。这正是我们下一章将要深入探讨的核心议题——国家能力。
