

\chapter{为什么有些国家“强”,有些国家“弱”?——解剖国家的心脏与手臂}

在上一章中,我们进行了一次思想上的“时空穿越”,见证了我们今天所熟悉的“现代国家”并非自古就有,而是在特定的历史条件下,为了应对混乱、组织社会而诞生的一个“现代发明”。我们理解了它的核心构造:主权、领土、暴力垄断、合法性和官僚机构。这就像我们拿到了一份国家的设计蓝图,知道了它“是什么”。

然而,仅仅拥有一份蓝图,并不意味着就能建造出一座坚固的大厦。在国际新闻的万花筒中,我们每天都会看到截然不同的国家命运:

\begin{quote}
在德国斯图加特,一位名叫克劳斯的工程师,他缴纳的近40\%的个人所得税,被一个高效的官僚系统转化为覆盖全民的医疗保险、子女免费的优质教育、以及准点到达的高速列车。他相信,当他失业或年老时,一个强大的社会保障网络会接住他。他生活的世界是可预测的、有序的。
\end{quote}

\begin{quote}
与此同时,在刚果民主共和国(DRC)的东部,一位名叫雅克的矿工,他冒着生命危险在手工作坊式的矿井里挖掘钴矿——这是制造我们手机电池的关键原料。然而,他所在的地区被武装民兵所控制,而非政府。他不知道自己挖出的矿产财富流向了何方,只知道自己和家人随时可能面临暴力和疾病的威胁。他从未见过像样的公路,他的孩子上不起学,生病了只能求助于巫医或昂贵的、由国际援助组织运营的诊所。他生活的世界是混乱的、朝不保夕的。
\end{quote}

克劳斯和雅克生活在同一个地球,但他们仿佛置身于两个平行的宇宙。他们命运的巨大差异,根源并不在于他们个人的努力,而在于他们背后那个名为“国家”的实体的天壤之别。德国是一个典型的“强国家”,而刚果(金)则是一个典型的“弱国家”。

一个国家强大与否,似乎直观地体现在其经济总量(GDP)、军事实力或国际影响力上。然而,在比较政治学中,“强”与“弱”的衡量标准远不止于此。一个更深层次、更关乎普通人命运的问题是:\textbf{为什么有些国家能够有效治理,将社会资源转化为公共福祉,提供安全、教育和健康,而另一些国家却深陷混乱,政府职能瘫痪,甚至连最基本的秩序都无法保障?}

要解剖这两种截然不同的国家命运,我们需要引入两个诊断国家健康状况的核心概念:“国家能力”(State Capacity)和“国家自主性”(State Autonomy)。它们就像国家这台复杂机器的“手臂”与“心脏”,共同决定了一个国家是充满活力的巨人,还是步履蹒跚的泥足巨人。

\hrulefill

\section{国家能力:政府的“手臂”有多长、多有力?}

“国家能力”这个概念,听起来很学术,但其实非常直观。它指的就是一个国家\textbf{有效执行其自己制定的政策、提供公共服务、维持社会秩序和实现其目标的能力}。它衡量的是政府的“手臂”能伸多远,以及这只手有多么强壮和灵巧。一个“强”的国家,其政府通常具备以下几种彼此关联、相互支撑的关键能力。

\subsection{汲取能力:国家的“输血系统”}

汲取能力,是国家从社会中合法地获取资源(主要是税收)的能力。这是一个国家所有其他能力的基础,就像人体的“输血系统”,为国家机器的运转提供源源不断的能量。一个没有钱的政府,一切雄心壮志都只是空谈。

\textbf{为什么税收如此重要?}
一个高效、公平的税收系统,其意义远超“收钱”本身:
\begin{itemize}
    \item \textbf{财政基础:} 税收是公共服务(教育、医疗、基建、国防)的资金来源。
    \item \textbf{治理的体现:} 能够建立起复杂的税收系统,本身就说明国家拥有强大的信息收集、处理和强制执行能力,是对社会经济活动的有效穿透。
    \item \textbf{社会契约的纽带:} 正如一句古老的政治格言所说:“无代表,不纳税”(No taxation without representation)。反过来也同样深刻:\textbf{“不纳税,无代表”}。当公民普遍纳税时,他们会更关心政府如何花钱,更有动力去监督政府、要求问责,从而形成政府与公民之间的良性互动。
\end{itemize}

\textbf{衡量指标:}
\begin{itemize}
    \item \textbf{税收占GDP比重:} 这是最直观的指标。发达的“强国家”(如多数OECD国家)这一比重通常在30-50\%之间。例如,法国约为45\%,丹麦接近47\%。而许多“弱国家”可能不足15\%。例如,尼日利亚约为8\%,阿富汗在塔利班接管前仅约9\%。
    \item \textbf{税收征收效率:} 实际征收额与理论上应征额的比率。一个国家可能有名义上很高的税率,但如果征收能力低下,大量税款流失,也是枉然。
    \item \textbf{税基覆盖率:} 纳税人口或企业占总体的比例。税基越广,税收越稳定,也越公平。
    \item \textbf{非正规经济(Informal Economy)比重:} 这是一个反向指标。非正规经济(如街头小贩、未注册的小作坊)游离于国家监管和税收体系之外。其比重越高,说明国家的汲取能力越弱。在许多发展中国家,非正规经济可能占到GDP的40\%甚至更高。
\end{itemize}

\textbf{案例对比:强弱分明的“输血系统”}

\begin{itemize}
    \item \textbf{强国家典范:法国的中央集权税收}
    法国拥有世界上最强大、最成熟的税收体系之一,这与其悠久的中央集权历史密切相关。从路易十四的“太阳王”时代开始,建立一个能从全国汲取资源的强大中央政府就是法国国家建设的核心。今天的法国,其公共财政总局(DGFiP)是一个高度专业化、信息化的官僚机构,能够有效地对个人和企业征收所得税、增值税(VAT)、社会保障缴款等多种税项。这种强大的汲取能力,支撑起了法国闻名于世的高质量公共服务,包括全民医疗体系、公共教育和完善的基础设施。
    \item \textbf{弱国家困境:刚果(金)的“资源诅咒”式汲取}
    刚果(金)拥有价值惊人的矿产资源(钴、铜、钻石等),但其政府的税收汲取能力却极其低下。其财政收入严重依赖于对矿产出口征收的少量关税和向外国矿业公司收取的特许权使用费。这种“点状”的资源汲取方式,而非对整个社会经济活动进行普遍征税,带来了灾难性后果:
    \begin{enumerate}
        \item \textbf{腐败的温床:} 资源收入高度集中,易被少数政治精英和军阀通过不透明的合同和回扣所侵吞。
        \item \textbf{治理惰性:} 政府无需费力建立复杂的全国性税收网络,也就不需要去了解和服务广大民众,导致国家治理能力长期停滞。
        \item \textbf{社会契约缺失:} 大多数民众不直接向国家纳税,他们感觉不到国家的存在,除非是以压迫者的面目出现。国家与社会之间严重脱节。
    \end{enumerate}
    \item \textbf{历史的视角:战争与税收}
    政治学家查尔斯·蒂利(Charles Tilly)曾提出一个著名的论断:“\textbf{战争制造国家,国家制造战争。}” 在近代早期的欧洲,连绵不断的战争是塑造强大汲取能力的催化剂。为了赢得战争,君主们必须以前所未有的规模筹集资金,购买火炮、建造军舰、支付雇佣兵。这迫使他们建立常设的官僚机构,进行人口普查和土地丈量,设计新的税种,并用强制力确保税款上缴。那些无法有效征税的国家,在战争中被淘汰出局。可以说,现代高效的税收国家,是在战火的熔炉中锻造出来的。
\end{itemize}

\subsection{规管能力:市场的“裁判员”与“交通警察”}

如果说汲取能力是为国家提供血液,那么规管能力就是国家伸出的“手”,为社会经济活动制定规则、维护秩序。它指国家制定、执行和强制实施法律法规,以规范社会经济活动的能力。这包括维护市场秩序、保护产权、执行合同、监管金融、保护环境和公共卫生等。一个没有规管能力的国家,市场将沦为野蛮的丛林。

\textbf{衡量指标:}
\begin{itemize}
    \item \textbf{法治指数(World Bank's Rule of Law Index):} 衡量社会成员和政府遵守法律规则的程度。
    \item \textbf{合同执行便利度(Doing Business Report):} 衡量解决商业纠纷所需的时间和成本。
    \textbf{产权保护指数(Property Rights Index):} 衡量私人财产权受法律保护的程度。
    \item \textbf{监管质量评分(Regulatory Quality Index):} 衡量政府制定和实施良好政策和法规的能力。
    \item \textbf{腐败控制指数(Control of Corruption Index):} 衡量公共权力被用于谋取私利的程度。
\end{itemize}

\textbf{案例对比:规则的确立与失效}

\begin{itemize}
    \item \textbf{强国家典范:新加坡的“法治立国”}
    新加坡是规管能力的全球典范。这个城市国家资源匮乏,但通过建立一个高效、透明、廉洁的法律和监管体系,成功吸引了全球资本,成为国际金融、贸易和航运中心。
    \begin{itemize}
        \item \textbf{产权保护:} 在新加坡,你的财产(无论是房产还是知识产权)受到极其严格的法律保护,这给了投资者巨大的信心。
        \item \textbf{合同执行:} 商业合同被严格执行,解决商业纠纷的司法程序高效且可预测。根据世界银行报告,新加坡的合同执行效率长期位居世界前列。
        \item \textbf{反腐败:} 新加坡的贪污调查局(CPIB)独立运作,拥有极大权力,使得新加坡成为全球最廉洁的国家之一。
    \end{itemize}
    强有力的规管能力,是新加坡从一个落后的殖民港口,一跃成为发达经济体的核心密码。
    \item \textbf{弱国家困境:印度的“纸上法规”}
    印度是一个拥有健全成文法律和独立司法体系的民主国家。然而,在实践中,其规管能力却面临巨大挑战。法律法规的执行往往缓慢、低效且充满不确定性。
    \begin{itemize}
        \item \textbf{合同执行难:} 尽管有法院,但在印度打一场商业官司平均需要近四年时间(约1445天),诉讼成本高昂,使得许多中小企业对商业合作望而却步。
        \item \textbf{“许可证拉吉”(License Raj):} 这是指印度独立后长期存在的、极其繁琐的行政审批和许可制度。虽然近年来有所改革,但官僚程序的拖沓和腐败(所谓的“速度钱”)仍然是企业面临的主要障碍。
        \item \textbf{环境监管不力:} 1984年发生的博帕尔(Bhopal)毒气泄漏事件是规管能力缺失的惨痛教训。尽管存在安全法规,但由于监管松懈和执行不力,美国联合碳化物公司的农药厂发生泄漏,导致数千人死亡,数十万人受害,成为人类历史上最严重的工业灾难之一。
    \end{itemize}
\end{itemize}

\subsection{提供公共服务能力:国家的“园丁”与“建筑师”}

这是国家能力最直接惠及民众的一面,指国家向公民提供教育、医疗、基础设施(如道路、电力、供水、网络)、社会保障等公共产品和服务的能力。它衡量的是一个国家能否为其公民创造一个有尊严、有保障、有机会的生活环境。

\textbf{衡量指标:}
\begin{itemize}
    \item \textbf{人类发展指数(HDI):} 综合衡量预期寿命、教育水平和人均收入的指标。
    \item \textbf{基础设施质量指数:} 道路、铁路、港口、电力、通信的覆盖率和质量。
    \textbf{教育普及率和质量:} 识字率、各级教育入学率、PISA(国际学生评估项目)成绩等。
    \item \textbf{医疗可及性:} 每千人医生/床位数、婴儿死亡率、人均预期寿命、疫苗接种率。
    \item \textbf{社会保障覆盖率:} 养老金、失业救济、医疗保险等覆盖的人口比例。
\end{itemize}

\textbf{案例对比:从无到有的建设与持续的失灵}

\begin{itemize}
    \item \textbf{强国家典范:韩国的“新村运动”与全民医保}
    韩国在二战后曾是世界上最贫穷的国家之一,满目疮痍。但通过强大的国家动员,其公共服务能力实现了惊天逆转。
    \begin{itemize}
        \item \textbf{“新村运动”(Saemaul Undong):} 20世纪70年代,朴正熙政府发起了旨在改造农村面貌的“新村运动”。政府提供水泥、钢筋等基本物资,并派出指导员,但核心是激发村民的“勤勉、自助、协作”精神,由村民自己规划和建设村庄的道路、桥梁、灌溉系统和公共会馆。这场自上而下推动、自下而上参与的运动,极大地改善了韩国农村的基础设施和生活水平,是国家提供公共服务能力的经典案例。
        \item \textbf{全民医保:} 韩国在短短12年内(1977-1989),就建立起了覆盖全体国民的医疗保险体系,创造了世界纪录。这背后是国家强大的组织、协调和财政投入能力。
    \end{itemize}
    \item \textbf{弱国家悲剧:海地的持续崩溃}
    海地是西半球最贫穷的国家,其公共服务能力的缺失是全方位的。2010年的毁灭性大地震,以及之后连绵不绝的政治动荡、飓风和疫情,使其本就脆弱的国家机器彻底瘫痪。
    \begin{itemize}
        \item \textbf{教育与医疗:} 大部分学校和医院由国际非政府组织(NGOs)、教会或私人运营,质量参差不齐且收费高昂,政府几乎无法提供统一、免费的教育和医疗服务。
        \item \textbf{基础设施:} 道路、电力、供水系统常年失修,大部分人口无法获得清洁饮用水和稳定供电。首都太子港的许多社区,连基本的垃圾处理系统都没有。
        \item \textbf{“NGO共和国”:} 由于政府失能,海地的许多公共服务功能实际上被国际NGO所取代。但这是一种不可持续且碎片化的模式,无法形成国家层面的统一规划和发展,甚至进一步削弱了本国政府的合法性和能力建设。
    \end{itemize}
\end{itemize}

\subsection{强制能力:国家的“佩剑”}

这是韦伯国家定义的核心,指国家通过其军队、警察和司法系统,有效维护社会治安、打击犯罪、保卫边疆、应对外部威胁的能力。它是国家所有其他能力得以施展的最终保障。一个连自身安全都无法保障的国家,其他一切都无从谈起。

\textbf{衡量指标:}
\begin{itemize}
    \item \textbf{领土控制程度:} 政府的法律和权力能够有效覆盖的国土比例。
    \item \textbf{犯罪率和治安状况:} 特别是凶杀率、抢劫率等恶性犯罪指标。
    \item \textbf{军队、警察的专业化和廉洁程度。}
    \item \textbf{司法系统的独立性和效率。}
    \item \textbf{暴力冲突的频率和规模。}
\end{itemize}

\textbf{案例对比:秩序的垄断与失控}

\begin{itemize}
    \item \textbf{强国家典半:日本的“安全神话”}
    日本拥有极强的强制能力,其社会秩序井然,犯罪率是全世界最低之一。
    \begin{itemize}
        \item \textbf{低犯罪率:} 日本的谋杀率仅为每10万人约0.3起(相比之下,美国约为5起,巴西高达27起)。公民普遍感到安全,夜间单独出行也无需担忧。
        \item \textbf{高效的警察系统:} 日本警察以其专业、高效和深入社区的“交番”(Koban,即警察岗亭)系统而闻名。警察与社区关系良好,被视为秩序的维护者和社区服务的提供者。
        \item \textbf{暴力的绝对垄断:} 在日本,私人拥有枪支受到极其严格的管制,任何非国家的暴力组织(如“暴力团”,即黑帮)都受到严密监控和法律打击。国家对暴力的垄断是毋庸置疑的。
    \end{itemize}
    \item \textbf{弱国家困境:墨西哥的“毒品战争”}
    墨西哥拥有正式的国家机构——军队、联邦警察、地方警察。然而,在国家的某些地区,其强制能力受到了强大贩毒集团的公然挑战,甚至被架空。
    \begin{itemize}
        \item \textbf{“双重权力”结构:} 在像锡那罗亚州或哈利斯科州这样的地方,大型贩毒卡特尔(如锡那罗亚卡特尔)拥有自己的重型武装、情报网络、社会基础甚至“税收”系统。它们与政府军警爆发的冲突,其激烈程度堪比内战。
        \item \textbf{国家机构的腐蚀:} 卡特尔通过贿赂和威胁,渗透和腐蚀了地方政府、警察和司法系统,使得国家的强制机器部分失灵甚至为其所用。
        \item \textbf{社会秩序的替代者:} 在一些政府缺位的社区,贩毒集团甚至扮演起秩序维护者和福利提供者的角色,修建教堂、分发食物,以换取当地民众的忠诚或默许。这极大地削弱了国家的合法性和强制能力。
    \end{itemize}
\end{itemize}

\hrulefill

\section{国家自主性:政府的“心脏”为谁而跳动?}

如果我们说国家能力是政府的“手臂”,那么国家自主性就是政府的“心脏”和“大脑”。它回答了一个更深层次的问题:\textbf{政府在做决策时,究竟是“为谁服务”?}

“国家自主性”指的是国家机构在多大程度上能够\textbf{独立于社会各利益集团(如财阀、地主、工会、地方势力、特定阶层或族群)的压力,制定和执行符合国家整体利益的政策}。它衡量的是政府在决策过程中,能否超越狭隘的特殊利益,而着眼于更广泛的公共利益。

\textbf{一个常见的误解:自主性 $\neq$ 独裁}
需要强调的是,国家自主性并非意味着政府脱离社会、闭门造车,更不等于独裁。一个健康的国家自主性,是在与社会保持有效互动、听取各方意见的基础上,仍能保持独立判断和行动的能力。它像一个经验丰富的船长,既要听取船员和乘客的意见,但最终必须依据海图和天气,做出最有利于整艘船航行的决策,而不是被船上嗓门最大或最有势力的乘客所绑架。

\subsection{高自主性 vs. 低自主性:两种不同的政治逻辑}

\begin{itemize}
    \item \textbf{高自主性国家:}
    \begin{itemize}
        \item \textbf{特征:} 政府(特别是其核心官僚机构)能够抵制来自强大游说团体、富裕精英或地方派系的压力,从而推行可能短期内不受欢迎但长期有利于国家发展的政策。
        \item \textbf{案例:} 战后日本的通商产业省(MITI)被认为是高自主性官僚机构的典范。MITI的精英官僚们独立于个别企业的短期利益,制定了日本长期的产业发展战略,成功引导日本从一个劳动密集型经济体,转型为技术和资本密集型的高科技强国。他们能够做到这一点,是因为MITI的官员拥有极高的声望、专业知识和职业保障,不受政治选举和企业游说的直接影响。
        \item \textbf{风险:} 过高的自主性也可能导致国家脱离社会,变得傲慢和僵化,无法回应民众的真实需求,甚至滑向威权主义。
    \end{itemize}
    \item \textbf{低自主性国家(或“被俘获的国家”):}
    \begin{itemize}
        \item \textbf{特征:} 政府的决策和行动被强大的社会利益集团所“俘获”(State Capture),国家政策不再服务于公共利益,而是服务于这些特殊集团的利益。
        \item \textbf{案例一:美国的“游说政治”}
        在美国,虽然国家能力很强,但其自主性在某些领域受到了强大游说集团的严重挑战。例如,美国的枪支暴力问题极其严重,但由于全国步枪协会(NRA)等枪支游说集团拥有巨大的政治影响力(通过政治献金、动员选民等方式),任何旨在加强枪支管制的法案都难以在国会通过。在这里,国家的决策被一个特定的利益集团所“俘获”,无法回应广大民众对公共安全的诉求。同样,制药公司强大的游说能力,也使得美国政府难以有效控制虚高的药品价格。
        \item \textbf{案例二:南非的“古普塔家族丑闻”}
        这是一个更极端的“国家俘获”案例。在南非前总统祖马执政期间,来自印度的古普塔家族通过与祖马家族的密切关系,系统性地影响了政府的人事任命、国有企业的合同审批和国家政策的制定,将国家机器变成了为自己家族牟利的工具。这导致了南非大规模的腐败、国有企业濒临破产和公共资源的巨大流失。
    \end{itemize}
\end{itemize}

\hrulefill

\subsection{“嵌入式自主性”:强国家的“秘密配方”}

通过上面的分析,我们似乎面临一个两难困境:国家自主性太高,可能脱离社会,变成压迫性的“利维坦”;自主性太低,又会被特殊利益集团俘获,无法服务于公共利益。那么,理想的“强国家”是如何平衡这对矛盾的呢?

政治学家彼得·埃文斯(Peter Evans)在其对发展型国家(如韩国、台湾)的研究中,提出了一个极具洞察力的概念——\textbf{“嵌入式自主性”(Embedded Autonomy)}。他认为,成功的“发展型国家”,其秘诀恰恰在于\textbf{将看似矛盾的“自主性”与“嵌入性”完美地结合起来}。

\begin{itemize}
    \item \textbf{自主性(Autonomy):} 指的是国家拥有一支高度专业化、有凝聚力、与社会其他利益集团相对绝缘的精英官僚队伍。这支队伍有能力独立地制定国家长远发展的战略目标,不受短期政治压力和寻租行为的干扰。
    \item \textbf{嵌入性(Embeddedness):} 指的是这支自主的官僚队伍,又通过各种正式和非正式的网络,深深地“嵌入”到社会之中,特别是与关键的产业界精英保持着密切、持续的沟通和协商。
\end{itemize}

\textbf{这种结合的魔力在于:}
\begin{itemize}
    \item \textbf{自主性}确保了国家不会被俘获,能够着眼于“大局”。
    \item \textbf{嵌入性}则确保了国家制定的政策是符合实际的、可执行的,并且能够获得关键社会力量的配合。国家不是在真空中发号施令,而是像一个“教练”一样,与“运动员”(企业界)紧密合作,共同赢得比赛。
\end{itemize}

\textbf{案例:韩国的经济企划院(EPB)与财阀(Chaebol)}
韩国在经济起飞时期,其经济企划院(EPB)就是“嵌入式自主性”的典范。EPB的官员是通过严格考试选拔出来的顶尖精英,他们享有极高的社会地位和职业保障,这确保了他们的\textbf{自主性}。同时,EPB与三星、现代等大型财阀之间,建立了一套紧密的、制度化的协商机制。政府通过出口目标、信贷优惠等方式引导财阀,财阀则向政府提供市场一线的真实信息和反馈。这种既独立又合作的“嵌入式”关系,使得韩国政府能够制定出既有雄心又切合实际的产业政策,并确保其得到有效执行,共同创造了“汉江奇迹”。

\textbf{反例:扎伊尔(今刚果金)的“掠夺型国家”}
与韩国形成鲜明对比的是蒙博托统治下的扎伊尔。蒙博托的政权是一个高度\textbf{自主}但完全\textbf{脱嵌(Disembedded)}的“掠夺型国家”。他的统治集团独立于社会任何群体,其唯一目标就是利用国家机器掠夺国家的钻石和矿产资源,中饱私囊。这个国家对社会没有任何责任感,也未与任何生产性社会力量建立联系,最终导致了国家经济的彻底崩溃和社会的长期动荡。

\hrulefill

\section{结论——强弱国家的历史根源与现实意义}

通过对国家能力和国家自主性的深入解剖,我们看到,一个“强”的国家,并不仅仅指其军事强大或经济体量巨大。一个真正强大的国家,是那个拥有\textbf{高能力}和\textbf{健康自主性}(特别是“嵌入式自主性”)的国家。它的“手臂”强壮有力,能够有效地汲取资源、规管社会、提供公共服务、维护最终秩序;同时,它的“心脏”和“大脑”能够独立思考,抵制特殊利益的诱惑,同时又深深植根于社会,与民众保持血肉联系,从而服务于最广泛的公共利益。

相反,“弱”国家则常常陷入“低能力、低自主性”的恶性循环:无力征税,导致无钱提供公共服务;无力规管,导致市场混乱、腐败横行;无力维持秩序,导致暴力频发、社会动荡。同时,其决策又极易被内部的军阀、部族、腐败官员或外部的强大势力所绑架,无法为国家和人民的未来做出正确的选择。

理解国家强弱的深层原因,不仅有助于我们分析不同国家的政治现实,也提醒我们,一个有效且负责任的政府,是国家发展和人民福祉的基石。而这些“强”与“弱”的国家,又如何在其内部构建起共同的认同,甚至引发“爱国”与“民族主义”的复杂情感呢?这正是我们第三章将要探讨的议题。

\textbf{教授小贴士:}
国家能力和自主性不是一成不变的。国家能力可以在外部压力、内部改革或危机中提升或削弱;国家自主性也可能因领导者更迭、社会力量结构变化或国际环境影响而波动。例如,一些国家在经历了战争或经济危机后,反而可能在某些方面增强了国家能力(如税收征收或应急管理),前提是它们能够进行有效的制度改革。反之,长期的腐败和治理不善则会侵蚀国家能力和自主性。因此,分析一个国家的强弱,需要考虑其历史进程和动态变化,而不仅仅是某个时点的静态快照。
