

\chapter{为什么“爱国”天经地义,但“民族主义”却是一把“双刃剑”?}

在上一章中,我们深入解剖了国家的“强”与“弱”,探讨了决定一个政府是服务于人民还是被利益集团俘获的“国家能力”与“国家自主性”。我们看到,一个强大的国家机器,能够有效汲取资源、规管社会、提供公共服务并维护秩序。这构成了国家治理的“硬件”基础。

然而,国家的强大不仅体现在其冰冷的制度和高效的能力上。一个国家若要真正凝聚人心,长治久安,还需要一种更深层次、更温暖、也更炽热的力量。它需要一个“灵魂”。

想象一下这样的场景:在奥运会的赛场上,当国歌奏响,国旗缓缓升起时,无论运动员还是观众,无数人热泪盈眶。在国家庆典的游行中,成千上万素不相识的人挥舞着同样的旗帜,分享着同一种自豪。在遥远的异国他乡,一句乡音、一道家乡菜,就能让两个陌生人瞬间产生亲切的联结。

这种强大的情感纽带,这种我们对一个我们称之为“祖国”的地方和一群我们称之为“同胞”的人所怀有的深厚情感,究竟从何而来?为什么全世界数以亿计的人,愿意为一个自己从未完全了解、其成员也从未全部谋面的抽象共同体,奉献自己的情感、财富,乃至生命?

这种情感,我们通常称之为“爱国主义”,并视其为天经地义的美德。然而,历史的另一面却向我们展示了这股力量的惊人破坏力。正是这种看似崇高的情感,在某些时候、某些地方,会演变为一种排外的、充满仇恨的、极具侵略性的意识形态,将世界拖入战争与种族灭绝的深渊。此时,它的名字,叫作“民族主义”。

本章的核心任务,就是要解开这个谜题。我们将首先厘清政治学中三个极易混淆的核心概念——国家(State)、政府(Government)和民族(Nation),它们是理解后续一切讨论的基础。然后,我们将深入探索“民族”这个“想象的共同体”是如何被建构起来的,并见证“民族主义”这股现代世界最强大的政治力量,如何既扮演了创造历史的“巨人”,又扮演了毁灭文明的“恶魔”。这把锋利无比的“双刃剑”,至今仍在深刻地塑造着我们的世界。

\hrulefill

\section{解构政治三位一体:国家、政府与民族}

在日常语境中,我们常常将“国家”、“政府”和“民族”混为一谈。我们会说“我们国家如何如何”,可能指的其实是“我们这届政府的政策”;我们会说“中华民族”,有时又用它来指代中华人民共和国这个“国家”。这种混用在日常交流中无伤大雅,但在政治学的分析中,精确地区分这三者,是避免思想混乱的第一步。我们可以用一个形象的比喻来理解它们的关系:

\begin{itemize}[noitemsep,topsep=0pt]
    \item \textbf{国家(State)是“房子”}:它是政治共同体的硬件设施和法律框架。
    \item \textbf{政府(Government)是“管家”}:它是当前负责管理这栋房子的团队。
    \item \textbf{民族(Nation)是“家人”}:它是居住在这栋房子里,并认为彼此有血缘或精神联结的共同体。
\end{itemize}

\subsection{国家:政治的“硬件”与“容器”}

正如我们在第一章和第二章中详细讨论的,\textbf{国家是一个拥有明确主权、固定领土,并对其领土内合法暴力拥有垄断权的政治实体}。

\begin{itemize}[noitemsep,topsep=0pt]
    \item \textbf{关键词是“制度”与“非人格化”}:国家是一套抽象的、相对持久的制度安排。它包括了宪法、法律体系、官僚机构、军队、警察等。它是一个非人格化的法律概念。
    \item \textbf{国家是“房子”}:它提供了政治活动发生的场所(领土),设定了游戏规则(法律),并拥有维护秩序的最终权力(暴力垄断)。这栋房子是相对稳定的,即使里面的“管家”换了一批又一批,房子本身依然存在。例如,法兰西第五共和国是一个国家,它的宪法和核心制度是稳定的,但其政府却在戴高乐、密特朗、希拉克、萨科齐、马克龙之间不断更迭。
\end{itemize}

\subsection{政府:权力的“操盘手”与“租客”}

\textbf{政府是具体负责管理国家、行使国家权力的机构和人员的总称}。它是国家的执行者,负责制定和实施法律与政策。

\begin{itemize}[noitemsep,topsep=0pt]
    \item \textbf{关键词是“动态”与“可变”}:政府是动态变化的,它可以通过选举、革命、政变或任期结束而更迭。
    \item \textbf{政府是“管家”或“租客”}:他们是当前拥有房子管理权的人。他们可以决定如何装修房子(制定政策)、如何使用房子的资源(财政预算),但他们并不拥有房子本身。当租期结束或被“房东”(在民主国家即人民)解雇时,他们就必须交出钥匙,换下一批管家。例如,我们说“拜登政府”或“特朗普政府”,指的都是在特定时期内管理美利坚合众国这个“国家”的具体团队。
\end{itemize}

\subsection{民族:共同体的“灵魂”与“想象”}

这是三者中最复杂、也最强大的概念。\textbf{民族是一个基于共同的文化、语言、历史、血缘、宗教或共同命运认同而形成的“想象的共同体”(Imagined Community)}。这个概念由著名学者本尼迪克特·安德森(Benedict Anderson)在其同名著作中提出,深刻地改变了我们对民族的理解。

\begin{itemize}[noitemsep,topsep=0pt]
    \item \textbf{关键词是“想象”与“情感”}:
    \begin{itemize}[noitemsep,topsep=0pt]
        \item \textbf{“想象”}并非指它是虚假的或捏造的,而是指一个民族的成员,即使是最小的民族,也永远不可能认识他们的大多数同胞,不可能与他们谋面,甚至不可能听到他们说话。然而,在他们每个人的心灵中,都存在着他们相互联结的意象。一个山东的农民和一个黑龙江的渔民,素未谋面,语言口音差异巨大,但他们都“想象”自己同属于“中华民族”这个大家庭。
        \item \textbf{“共同体”}则是因为,无论其内部存在着多大的不平等和剥削,民族总是被想象为一种深刻的、平等的同志情谊。正是这种兄弟情谊,在过去两个世纪中,驱使着数以百万计的人们,不是去剥削,而是心甘情愿地为他们素不相识的同胞,为这种有限的想象,献出自己的生命。
    \end{itemize}
    \item \textbf{民族是“家人”}:他们相信彼此共享着某种深刻的联结,这种联结超越了地域、阶级和职业。他们共享着一个共同的“家”(祖国),有着共同的“家史”(民族历史),讲着共同的“家常话”(民族语言)。
\end{itemize}

\textbf{民族是如何被“想象”出来的?——共同体的建构工具}

安德森指出,这种现代的民族“想象”并非凭空产生,而是通过一系列现代化的工具被大规模建构和传播的。

\begin{enumerate}[noitemsep,topsep=0pt]
    \item \textbf{印刷资本主义(Print-Capitalism)}:这是安德森理论的核心。在活字印刷术普及之前,人们生活在彼此隔绝的社区里,信息传播极其有限。而报纸和廉价小说的出现,创造了奇迹。当成千上万个互不相识的法国人,在同一天的早晨,用同一种标准化的法语,阅读着关于“法国”的同一则新闻时,一种“我们”正在共同经历着同一件事的感觉便油然而生。他们通过阅读,想象着无数个和自己一样的“同胞”也在做着同样的事。这创造了一个统一的话语场和时间感,将分散的个体联结成一个想象中的读者共同体。
    \item \textbf{语言的标准化}:民族主义的兴起,往往伴随着对方言的压制和对一种“国家标准语”的推广。通过建立国家级的教育系统,教授标准化的语言、语法和文学,国家成功地创造了一个能够无障碍交流的内部共同体,并同时在语言上将“我们”和那些讲不同语言的“他们”区分开来。法兰西学院(Académie française)自17世纪以来就致力于规范和“净化”法语,正是这种国家力量塑造民族语言的典型。
    \item \textbf{地图、博物馆与人口普查}:
    \begin{itemize}[noitemsep,topsep=0pt]
        \item \textbf{地图}:地图测绘技术的发展,首次将国家的轮廓清晰地、可视化地呈现在人们面前。地图上的那条边界线,将“我们的”领土从“他们的”领土中分割出来,赋予了“祖国”一个具体、神圣的形象。
        \item \textbf{博物馆}:国家博物馆通过对文物的精心挑选和陈列,建构了一部线性的、光荣的、从古至今一脉相承的“民族史诗”。它告诉人们,“我们”拥有共同的、辉煌的过去,因此也应拥有共同的未来。
        \item \textbf{人口普查}:普查将模糊的人口,转化为精确的、可分类的数据。它不仅让国家能够更有效地治理,也让每一个人在表格中被归类为“某民族”的一员,从而在心理上强化了这种身份认同。
    \end{itemize}
    \item \textbf{民族神话与“被发明的传统”}:
    每个民族都需要自己的英雄和神话。19世纪的民族主义浪潮中,欧洲各国纷纷“挖掘”或“创造”自己的民族英雄。法国人将抵抗凯撒的高卢首领维钦托利(Vercingétorix)塑造为“第一个法国人”;德国人则将曾在条顿堡森林击败罗马军团的日耳曼部落首领阿米尼乌斯(Arminius)尊为民族英雄。历史学家艾瑞克·霍布斯鲍姆(Eric Hobsbawm)指出,许多我们看似古老的“传统”,如苏格兰的格子裙和风笛,其实是在近代为了建构民族认同而被“发明”或“标准化”的。
\end{enumerate}

\subsection{当“三位一体”不再:现实世界的复杂图景}

理想状态下,一个国家(State)能够与一个民族(Nation)完美对应,形成“民族国家”(Nation-State),由一个政府(Government)来代表和管理。然而,现实世界远比这复杂,三者的错位和冲突,是理解当代政治纷争的一把钥匙。

\begin{itemize}[noitemsep,topsep=0pt]
    \item \textbf{有民族,无国家(Nation without a State)}:
    这是世界上许多冲突的根源。一个拥有强烈自我认同的民族,却没有属于自己的独立国家。
    \begin{itemize}[noitemsep,topsep=0pt]
        \item \textbf{库尔德人}:这是最典型的例子。大约有3000万库尔德人,拥有自己的语言、文化和历史,但他们却被分割在土耳其、伊拉克、叙利亚和伊朗四个国家的边境地区。在每个国家,他们都面临不同程度的压迫和同化政策,这反而激发了他们更强烈的民族主义。他们争取自治甚至独立的斗争,至今仍是中东地区不稳定的重要因素。
        \item \textbf{巴勒斯坦人}:在以色列建国后,大量巴勒斯坦人流离失所,散居在约旦、黎巴嫩、叙利亚以及被占领的领土上。对故土的共同记忆和对建国的共同渴望,将他们凝聚成一个坚韧的民族,他们的建国斗争深刻地影响着整个世界的政治格局。
    \end{itemize}
    \item \textbf{有国家,无(统一)民族(State without a Nation)}:
    这种情况在许多前殖民地国家,尤其是在非洲,非常普遍。
    \begin{itemize}[noitemsep,topsep=0pt]
        \item \textbf{尼日利亚的悲剧}:尼日利亚的边界是19世纪末英国殖民者在地图上随手划定的,它将北部以豪萨-富拉尼族为主的穆斯林地区、西南部以约鲁巴族为主的地区、以及东南部以伊博族为主的基督教地区,强行捏在了一个国家之内。这三个主要族群在语言、宗教、文化上差异巨大,历史上也缺乏共同的政治记忆。国家独立后,族群间的猜忌和对资源的争夺从未停止,最终在1967年酿成了惨烈的“比亚法拉战争”——伊博人宣布独立,成立比亚法拉共和国,引发了导致上百万人死亡的内战。今天的尼日利亚,虽然维持了形式上的统一,但人们的身份认同往往首先是“约鲁巴人”或“伊博人”,其次才是“尼日利亚人”。这种国家认同的薄弱,是其腐败、治理不力和社会冲突的深层原因。
    \end{itemize}
    \item \textbf{多民族国家(Multi-national State)}:
    一些国家承认其内部存在多个具有民族地位的群体,并试图通过制度安排来协调它们的关系。
    \begin{itemize}[noitemsep,topsep=0pt]
        \item \textbf{联合王国(United Kingdom)}:它并非一个单一的民族国家,而是由英格兰、苏格兰、威尔士和北爱尔兰四个“构成国”(Constituent countries)联合而成。苏格兰拥有自己的法律体系、教育系统和议会,其强烈的民族认同使得苏格兰独立运动长期存在,并在2014年举行了独立公投。
        \item \textbf{加拿大}:加拿大长期面临英语区和法语区(主要是魁北克省)的紧张关系。为了维系国家统一,加拿大实行官方双语政策和联邦制,并赋予了魁北克高度的文化和法律自主权。但魁北克主权运动(Quebec sovereignty movement)仍然是加拿大政治中一个挥之不去的主题。
    \end{itemize}
\end{itemize}

理解了国家、政府和民族这三者的区别与复杂关系,我们才能真正开始理解“民族主义”这股塑造了现代世界的力量,究竟从何而来,又将走向何方。

\hrulefill

\section{巨人的诞生:民族主义作为创造的力量}

“民族主义”(Nationalism)是一种意识形态和政治运动,它认为“民族”是人类社会最基本、最重要的归属单位,并主张每一个民族都应该拥有自决权(Self-determination),即建立属于自己的独立国家。简而言之,民族主义的政治原则就是:\textbf{民族的边界应该与国家的边界相重合}。

在18世纪之前,这种思想是不可想象的。但在那之后,民族主义如燎原之火,以前所未有的力量重塑了世界政治地图。从积极的方面看,它在人类历史上扮演了至关重要的创造性角色。

\subsection{锻造现代共同体:从臣民到公民}

民族主义最伟大的功绩,在于它将分散的、彼此隔绝的个体,凝聚成一个拥有共同目标、共同命运和强烈归属感的集体。

\begin{itemize}[noitemsep,topsep=0pt]
    \item \textbf{法国大革命的洗礼}:这是民族主义的“创世纪”。在革命之前,一个法国农民首先效忠的是他的领主和国王。但革命彻底颠覆了这一切。1789年的《人权宣言》宣称:“整个主权的本原,主要是寄托于国民(Nation)。” 这意味着,权力的合法性不再来源于“君权神授”,而是来源于全体“法兰西民族”。当欧洲的君主国组成联军企图扼杀革命时,革命政府颁布了“全民总动员令”(Levée en masse),号召所有法国人,无论男女老少,都为保卫“祖国”(La Patrie)而战。一支由“公民”组成的、为保卫“民族”而战的军队,其迸发出的战斗热情和凝聚力,是那些为薪水而战的雇佣兵军队无法比拟的。从此,“为国捐躯”成为一种崇高的美德。民族主义成功地将国王的“臣民”(Subjects)转化为了国家的“公民”(Citizens)。
\end{itemize}

\subsection{驱动国家统一与建设:聚合的力量}

在19世纪,民族主义是推动德意志和意大利这两个欧洲重要国家实现统一的核心动力。

\begin{itemize}[noitemsep,topsep=0pt]
    \item \textbf{德意志的“血与铁”}:在19世纪初,不存在一个叫“德国”的国家,只存在一个讲德语的、松散的“德意志邦联”,由普鲁士、奥地利、巴伐利亚等三十多个大小不一的邦国组成。拿破仑的征服极大地激发了德意志的民族情感。哲学家费希特(Johann Gottlieb Fichte)在柏林发表《告德意志国民书》,呼吁德意志人认识到自己独特的语言和文化,团结起来。格林兄弟搜集民间童话,不仅仅是出于文学兴趣,更是为了发掘德意志民族共同的文化基因。最终,普鲁士首相俾斯麦(Otto von Bismarck)巧妙地利用了这股民族主义浪潮,通过一系列精心策划的战争(普丹战争、普奥战争、普法战争),以“血与铁”的手段,排除了奥地利,击败了法国,于1871年在法国凡尔赛宫宣告了德意志帝国的成立。民族主义为国家的统一提供了强大的合法性。
    \item \textbf{意大利的“复兴运动”(Risorgimento)}:与德国类似,19世纪的意大利也是一个“地理概念”,被多个王国和外国势力(主要是奥地利)所分割。意大利的统一,是三位伟人合力的结果:马志尼(Giuseppe Mazzini)是统一运动的“灵魂”,他创立“青年意大利党”,用浪漫的民族主义理想唤醒了意大利人的民族意识;加里波第(Giuseppe Garibaldi)是“利剑”,他率领“红衫军”远征西西里,以惊人的军事冒险解放了意大利南部;而撒丁王国首相加富尔(Camillo di Cavour)则是“大脑”,他以高超的现实主义外交手腕,纵横捭阖,最终完成了意大利的统一大业。统一后,政治家马西莫·达泽利奥(Massimo d'Azeglio)说了一句名言:“\textbf{我们已经创造了意大利,现在我们必须创造意大利人。}” 这句话深刻地揭示了“国家建设”与“民族构建”是两个密不可分的过程。
\end{itemize}

\subsection{ 反抗殖民统治的旗帜:解放的力量}

在20世纪,民族主义的火种从欧洲传播到亚非拉大陆,成为被压迫民族反抗殖民统治、争取民族解放的最有力的思想武器。

\begin{itemize}[noitemsep,topsep=0pt]
    \item \textbf{印度的非暴力抗争}:在英国统治下的印度,是一个语言、宗教、种姓制度极其复杂的社会。将如此多元的群体凝聚成一个“印度民族”,是一项艰巨的工程。以甘地(Mahatma Gandhi)和尼赫鲁(Jawaharlal Nehru)为首的印度国民大会党,成功地将反抗英国殖民统治的斗争,塑造为一场全体印度人的民族解放运动。甘地本人就是一位卓越的民族构建大师。他身着土布,手摇纺车,这些行为本身就是强大的政治符号,号召印度人抵制英国的工业品,重拾民族的经济和文化自信。他的非暴力不合作运动(如“食盐进军”),成功地动员了千百万印度民众,最终迫使大英帝国在1947年承认印度独立。
    \item \textbf{越南的持久战}:越南的民族主义根植于其上千年反抗中国封建王朝统治的历史记忆中。在近代,这种传统又与反抗法国殖民主义和美国侵略的斗争相结合。胡志明(Hồ Chí Minh)领导的越南独立同盟会(越盟),巧妙地将共产主义的社会革命理论与强烈的民族主义情感融为一体。对于普通越南农民来说,他们参加战争,既是为了实现“耕者有其田”的社会理想,更是为了将外国侵略者赶出自己的家园,捍卫民族的独立与尊严。这种强大的精神力量,使得越南能够战胜比自己强大得多的法国和美国。
\end{itemize}

\subsection{ 提供合法性与社会凝聚力}

民族主义为国家和政府的统治提供了重要的合法性基础。当政府被视为民族利益的代表和捍卫者时,其统治更容易获得民众的认可和支持。在国家面临危机时(如战争、自然灾害、经济萧条),诉诸民族团结和爱国主义,是政府动员社会资源、共克时艰的最有效方式。它能够让人们暂时放下内部的阶级、地域、党派分歧,为了“国家”这个更大的共同体而努力。

\hrulefill

\section{潘多拉的魔盒:民族主义作为毁灭的力量}

然而,正如本章标题所揭示的,民族主义是一把锋利无比的“双刃剑”。它在创造了现代世界的同时,也开启了一个充满冲突、仇恨和暴力的“潘多拉魔盒”。当民族主义走向极端,它就会变成一种极具毁灭性的力量。

\subsection{“我们”与“他们”:排他性的致命逻辑}

民族主义的核心在于构建“我们”(in-group)的认同,但这几乎不可避免地要通过界定一个“他们”(out-group)来完成。当对“我们”的爱,转化为对“他们”的恨时,悲剧便拉开了序幕。

\begin{itemize}[noitemsep,topsep=0pt]
    \item \textbf{从独特性到优越性}:温和的民族主义强调本民族的独特性和价值。但极端的民族主义则会滑向“我族中心主义”(Ethnocentrism),宣称本民族在文化、血缘或道德上优于其他民族。
    \item \textbf{从差异到威胁}:极端的民族主义会将“他者”视为对本民族生存、纯洁性或利益的威胁。这种“威胁”可能是军事上的、经济上的,也可能是文化或人口上的。
\end{itemize}

\subsection{极端民族主义的黑暗篇章:战争与种族灭绝}

历史已经反复证明,极端的民族主义是国际冲突和战争最重要的根源之一。

\begin{itemize}[noitemsep,topsep=0pt]
    \item \textbf{案例一:纳粹德国与第二次世界大战}
    纳粹主义是民族主义最极端、最病态的表现形式。它将德意志民族定义为一个优越的“雅利安种族”,并认为德国的生存空间(Lebensraum)受到了国际犹太阴谋、共产主义和凡尔赛条约的不公正压迫。
    \begin{itemize}[noitemsep,topsep=0pt]
        \item \textbf{种族化的民族主义}:纳粹的民族主义不是基于文化或政治认同,而是基于虚假的“血缘科学”。在这种逻辑下,犹太人、罗姆人(吉普赛人)、斯拉夫人等被定义为“劣等种族”,是污染德意志民族纯洁性的“寄生虫”或“病毒”。
        \item \textbf{对外侵略}:为了夺取“生存空间”,纳粹德国悍然撕毁国际条约,吞并奥地利、捷克斯洛伐克,入侵波兰,最终点燃了第二次世界大战的战火。
        \item \textbf{种族灭绝(Holocaust)}:对内,纳粹政权通过了《纽伦堡法案》,剥夺犹太人的公民权,并最终在欧洲范围内对约600万犹太人进行了系统性的、工业化的大屠杀。这是民族主义“排他性”逻辑走到极致的、最骇人听闻的罪行。
    \end{itemize}
    \item \textbf{案例二:卢旺达大屠杀(1994)}
    这场在短短100天内导致约80万人死亡的悲剧,是族群民族主义(Ethnic nationalism)残酷性的集中体现。在卢旺达,胡图族(Hutu)和图西族(Tutsi)两个族群语言文化本无太大差异,其身份区隔是在德国和比利时殖民统治时期被人为强化和固化的。
    \begin{itemize}[noitemsep,topsep=0pt]
        \item \textbf{身份的政治化}:殖民者利用图西族作为统治的“代理人”,加剧了两族之间的不平等和怨恨。独立后,占人口多数的胡图族掌握了政权,并开始推行歧视图西族的政策。
        \item \textbf{仇恨的煽动}:1994年,在时任总统(胡图族)的座机被击落后,胡图极端分子掌控的政府和媒体(特别是“千丘自由广播电台”,RTLM)开始进行疯狂的仇恨宣传,将全体图西人描绘成国家的敌人、阴谋家,并用“蟑螂”(inyenzi)这种非人化的词语来称呼他们。
        \item \textbf{邻人间的屠杀}:在政府的组织和煽动下,大量普通的胡图族平民,拿起砍刀和棍棒,对自己曾经的邻居、同事、朋友——图西人,进行了惨无人道的屠杀。卢旺达的悲剧警示我们,在政治精英的操弄下,看似平静的社会可以在极短时间内被民族仇恨所吞噬。
    \end{itemize}
    \item \textbf{案例三:南斯拉夫的解体}
    在铁托(Josip Broz Tito)的强力统治下,南斯拉夫曾是一个奉行“兄弟情谊与团结”口号的多民族联邦国家。然而,铁托去世后,经济的衰退和政治的真空,为各路民族主义政客提供了舞台。
    \begin{itemize}[noitemsep,topsep=0pt]
        \item \textbf{民族主义的复活}:塞尔维亚领导人米洛舍维奇(Slobodan Milošević)率先打出“大塞尔维亚主义”的旗号,煽动塞尔维亚人的民族情绪以巩固个人权力。这引发了连锁反应,克罗地亚、波斯尼亚等地的民族主义情绪也被相继点燃。
        \item \textbf{暴力与“种族清洗”}:曾经的“南斯拉夫人”认同迅速瓦解,取而代之的是塞尔维亚人、克罗地亚人、波斯尼亚人(穆斯林)之间日益加深的对立。最终,南斯拉夫在1990年代初陷入了一系列血腥的内战。战争中,“种族清洗”(Ethnic cleansing)——即通过暴力、驱逐、屠杀等手段将特定族群从某一地区清除——成为交战各方蓄意采取的战争策略。萨拉热窝的围城战、斯雷布雷尼察的大屠杀,都是极端民族主义结出的恶果。
    \end{itemize}
\end{itemize}

\subsection{ 压制内部多元性}

即使不对外发动战争,民族主义也可能对国家内部造成伤害。为了维护所谓的“民族纯洁性”或“国家统一”,占主导地位的民族主义常常会压制国内少数族裔的语言、文化和宗教自由,强迫他们同化。西班牙的佛朗哥独裁时期,就曾严厉禁止在公共场合使用加泰罗尼亚语和巴斯克语。这种强制同化政策,往往会激起少数族裔更强烈的反抗,导致社会内部的紧张和分裂。

\hrulefill

\section{“爱国主义”还是“民族主义”?——一个重要但模糊的界限}

通过以上的分析,我们看到民族主义既有光明的一面,也有黑暗的一面。那么,我们日常生活中所说的、并普遍赞赏的“爱国主义”(Patriotism),与这个时而崇高、时而狰狞的“民族主义”,究竟是什么关系?

这是一个在政治学和哲学上充满争议的问题。一般而言,学者们试图从以下几个方面对它们进行区分:

\begin{itemize}[noitemsep,topsep=0pt]
    \item \textbf{对象不同}:
    \begin{itemize}[noitemsep,topsep=0pt]
        \item \textbf{爱国主义}:其爱的对象是“Patria”,即祖国、故土。它更侧重于对这个地方、其生活方式、其制度和其同胞的深厚情感和忠诚。它是一种“爱我所有”。
        \item \textbf{民族主义}:其核心是“Nation”,即民族共同体。它更强调民族的独特性、团结和政治自决权。
    \end{itemize}
    \item \textbf{态度不同}:
    \begin{itemize}[noitemsep,topsep=0pt]
        \item \textbf{爱国主义}:通常被认为是一种更\textbf{包容的、防御性的}情感。一个爱国者,爱自己的国家,但并不必然认为自己的国家比别国优越,也不必然仇视其他国家。正如乔治·奥威尔(George Orwell)所说:“爱国主义的本质,是献身于一个特定的地方和一种特定的生活方式,但并不想把它强加于人。”
        \item \textbf{民族主义}:则常常带有一种\textbf{排他性的、竞争性的,甚至是侵略性的}色彩。它强调“我们的民族”是独一无二的,甚至是最优秀的,其核心驱动力是“权力欲”。奥威尔认为,“民族主义者”的一个标志就是,他不仅为自己的民族服务,而且坚信自己的民族凌驾于一切,不受善恶的约束。
    \end{itemize}
    \item \textbf{认同基础不同(公民民族主义 vs. 族群民族主义)}:
    这是一个更深层次的区分,许多学者认为这才是关键所在。
    \begin{itemize}[noitemsep,topsep=0pt]
        \item \textbf{公民民族主义(Civic Nationalism)}:这种观点认为,民族是一个由共享相同政治信念、价值观和法律制度的“公民”所构成的政治共同体。\textbf{成员资格是自愿的、开放的},理论上任何种族、肤色、宗教的人,只要认同这个国家的政治理念(如自由、民主、法治),并遵守其法律,就可以成为这个民族的一员。这种民族主义通常被认为是更理性的、更包容的,与“爱国主义”的概念高度重合。法国和美国常被视为公民民族主义的典范。
        \item \textbf{族群民族主义(Ethnic Nationalism)}:这种观点认为,民族是一个基于共同血缘、祖先、语言和文化的“命运共同体”。\textbf{成员资格是先天的、封闭的},是由出身决定的,而非个人选择。这种民族主义强调的是“血浓于水”,具有强烈的排他性,更容易演变为极端的、不宽容的政治运动。德国、日本以及东欧许多国家的民族主义,传统上被认为更偏向族群民族主义。
    \end{itemize}
\end{itemize}

\textbf{界限的模糊性:一个警惕的视角}

然而,这种看似清晰的区分,在现实中却常常是模糊不清的。
\begin{itemize}[noitemsep,topsep=0pt]
    \item \textbf{“公民”外衣下的排他性}:即使是号称最“公民”的法国,其强调“法兰西价值观”和严格世俗主义(laïcité)的模式,在面对穆斯林移民时,也常常被批评为一种强制性的文化同化,带有排他色彩。
    \item \textbf{情感的转化}:健康的爱国主义情感,在特定的政治氛围和舆论操纵下,很容易就能被转化为具有攻击性的民族主义。当国家面临外部威胁或内部危机时,政治家们常常通过煽动“我们”对抗“他们”的叙事,将民众的爱国热情引向排外和仇恨。
    \item \textbf{“我的爱国主义,你的民族主义”}:在国际冲突中,人们往往倾向于将自己一方的行为描述为“爱国”,而将对方的行为贬斥为“狭隘、非理性的民族主义”。这使得这两个词本身也成为了政治话语斗争的工具。
\end{itemize}

因此,一个更审慎的看法是,\textbf{爱国主义和民族主义并非截然对立,而是一个连续体(continuum)的两端}。爱国主义是这个连续体上温和、内向、包容的一端;而民族主义则是其激进、外向、排他的一端。它们共享着对共同体的热爱这一情感内核,但其政治表达和后果却可以天差地别。警惕爱国主义向极端民族主义的滑落,是每一个现代公民的责任。

\hrulefill

\section{结论:驾驭现代世界的“双面神”}

民族主义,这个在法国大革命的熔炉中诞生的现代“双面神”,无疑是过去两个多世纪以来,塑造人类历史最强大的政治力量。

它是一股伟大的创造力。它将人们从对国王和领主的私人效忠中解放出来,塑造了现代公民的观念;它将四分五裂的土地凝聚成统一的国家,为现代经济和政治发展提供了框架;它也成为被压迫者反抗暴政和殖民统治的号角。没有民族主义,我们今天所知的世界政治地图将完全是另一番模样。

但它也是一股可怕的毁灭力。它以“我们”和“他们”的名义,制造了无数的隔阂、仇恨、战争和种族灭绝。它像一种现代的世俗宗教,一旦走向原教旨主义,就会变得极不宽容,要求绝对的忠诚,并以“民族”之名,将一切非人道的行为合理化。

理解民族主义的起源、发展及其两面性,对于我们分析当今世界的国际关系、国内政治冲突、移民问题、身份政治等核心议题,至关重要。它提醒我们,对共同体的爱是宝贵的,但这种爱需要理性的约束和对“他者”的尊重。

而这些被民族主义所凝聚起来的国家,又会选择怎样的政府形式来管理自身呢?是选择一个权力分立、相互制衡的总统,还是一个与议会共存亡的总理?总统和总理,这两个我们耳熟能详的称谓背后,又隐藏着怎样截然不同的权力逻辑和制度设计?这正是我们第四章将要揭示的奥秘。
