

\chapter{为什么大家都说“民主是个好东西”,但它的样子千差万别?}

在上一章中,我们深入解剖了总统制、议会制和半总统制这三种主要的政府体制。我们看到,不同的制度设计,如同不同的机械构造,决定了国家权力的核心引擎如何运转、制动和平衡。这些精巧的制度安排,最终都是为了承载一个更宏大、也更激动人心的理念——“\textbf{民主}”(Democracy)。

“民主”这个词,在今天的世界上,无疑是一个“好词”。几乎没有哪个国家的领导人会公开宣称自己反对民主。它与自由、平等、人权等概念一起,被普遍视为人类社会追求的崇高理想。从联合国的文件到各国政客的演讲,从学者的论著到街头抗议的标语,“民主”无处不在。

然而,当我们把目光从抽象的理念投向纷繁复杂的现实世界时,一幅令人困惑的图景便展现在眼前:

\begin{itemize}
    \item 在瑞士的小镇,公民们会定期聚集在一起,通过举手投票的方式,直接决定镇上是否要修建一条新的公路。这似乎最接近“人民的统治”的古老理想。
    \item 在美国,一场耗资数十亿美元、历时近两年的总统大选,最终可能由几个“摇摆州”的数万张选票决定胜负,而获胜的总统可能在全国普选票上少于对手。
    \item 在印度,这个号称“世界最大民主国家”的地方,超过九亿选民参与的选举,如同一场盛大而喧嚣的节日,但其政治运作又常常伴随着家族政治、种姓矛盾和金钱交易。
    \item 在匈牙利,一位通过合法选举上台的领导人,却利用其在议会的多数优势,系统性地修改宪法、控制媒体、削弱司法独立,一步步地侵蚀着民主的根基。
\end{itemize}

这些景象都发生在那些声称自己是“民主”的国家。这不禁让人思考:为什么同样被称为“民主”的制度,在不同国家会呈现出如此不同的面貌?我们该如何理解和衡量这些差异?民主仅仅意味着投票选举吗?还是有更深刻的内涵?

本章的目的,就是要带领大家深入民主的“万花筒”,探索这个既熟悉又陌生的概念。我们将首先回到源头,厘清民主的定义与核心要素;然后,我们将探讨如何像医生给病人做体检一样,从多个维度来衡量一个国家的民主“健康状况”;最后,也是最重要的,我们将分析民主在现实世界中呈现出的不同形态——从最理想的“\textbf{自由民主}”,到只有选举外壳的“\textbf{选举民主}”,再到民主的“冒牌货”——“\textbf{劣质民主}”和“\textbf{竞争性威权主义}”。通过这次旅程,我们希望能帮助大家建立一个更立体、更批判的民主认知框架,理解民主的多样性、复杂性,以及它在当代的脆弱性。

\hrulefill

\section{民主溯源:从“人民的统治”到现代定义}

要理解民主,我们必须先回到它的词源。\textbf{“民主”(Democracy)}一词源于古希腊语,由“\textbf{demos}”(人民)和“\textbf{kratos}”(统治、权力)两个词根组合而成,其最本初的含义就是“\textbf{人民的统治}”或“\textbf{由人民来统治}”。

\subsection{古典的理想:雅典的直接民主}

民主的最初实践,可以追溯到公元前5世纪的古希腊城邦——雅典。雅典的民主是一种\textbf{直接民主(Direct Democracy)}。
\begin{itemize}
    \item \textbf{公民大会(Ekklesia)}:城邦的最高权力机构是所有成年男性公民(不包括妇女、奴隶和外邦人)组成的公民大会。他们定期聚集在雅典卫城下的普尼克斯(Pnyx)山岗上,就战争、媾和、立法、人事任免等一切重大公共事务进行辩论和投票。
    \item \textbf{轮流执政}:为了防止权力集中,雅典的许多公职,如负责日常行政的“五百人会议”(Boule),其成员是通过抽签的方式从所有公民中选出的,任期很短。这体现了“轮流执政”(Ruling and being ruled in turn)的理想,即每个公民都有机会参与管理城邦。
\end{itemize}

雅典的直接民主,体现了“人民的统治”最纯粹的形态。然而,它也有着巨大的局限性:
\begin{enumerate}
    \item \textbf{规模限制}:它只适用于像城邦这样规模小、人口少的政治共同体。
    \item \textbf{排他性}:其“人民”(demos)的范围极其狭窄,将社会的大多数成员排除在外。
    \item \textbf{“多数人的暴政”风险}:公民大会容易被情绪和雄辩家所煽动,做出非理性的、压迫少数派的决定。苏格拉底之死,就是雅典民主投票处死一位哲学家的著名案例。
\end{enumerate}

\subsection{现代的演化:代议制民主的兴起}

随着现代国家的兴起,领土扩大、人口增多,直接民主变得不再可行。一种新的民主形式——\textbf{代议制民主(Representative Democracy)}——应运而生。在这种模式下,人民不再是自己直接统治,而是通过选举自己的\textbf{代表(Representatives)},组成议会或政府,来代为行使权力。

美国政治学家\textbf{罗伯特·达尔(Robert A. Dahl)},是当代民主理论的集大成者。他认为,现代大规模民主国家的理想形态,更准确地说,应该被称为“\textbf{多头政体}”(Polyarchy)。“多头政体”并非完美的民主,而是现实世界中能够最大限度接近民主理想的一种制度安排。它包含两个核心维度:

\begin{enumerate}
    \item \textbf{广泛的参与(Participation)}:绝大多数成年公民都拥有选举权和被选举权。
    \item \textbf{充分的竞争(Contestation)}:存在自由、公平、定期的选举,不同的政党和候选人可以公开竞争政治职位,公民可以在他们之间做出真实的选择。
\end{enumerate}

\subsection{现代民主的核心要素:不止于选举}

基于现代民主的实践,我们可以总结出其不可或缺的几个核心要素。一个健康的现代民主,就像一张由五根支柱支撑的桌子,缺一不可。

\begin{enumerate}
    \item \textbf{自由、公平和定期的选举(Free, Fair, and Regular Elections)}:
    这是民主最显著、最程序性的特征。
    \begin{itemize}
        \item \textbf{自由}:意味着所有成年公民都有平等的投票权(普选权),且在投票时不受恐吓和胁迫。同时,也意味着公民有权自由地组建政党或以独立候选人身份参选。
        \item \textbf{公平}:意味着选举的“游戏规则”对所有参与者都是公平的。反对党有公平的竞争机会,能够自由地宣传、集会,而不受执政党的打压。计票过程必须是透明和诚实的。
        \item \textbf{定期}:意味着选举必须按照宪法和法律规定的时间间隔定期举行,不能被当权者随意推迟或取消。
    \end{itemize}

    \item \textbf{公民自由与权利的保障(Protection of Civil Liberties and Rights)}:
    这是民主的实质性核心。民主绝不仅仅是选举日的投票行为。在一个真正的民主国家,公民在一周七天、一年365天里,都必须享有一系列基本的自由和权利。
    \begin{itemize}
        \item \textbf{言论自由}:公民有权自由地表达自己的政治观点,包括批评政府,而不用担心受到惩罚。
        \item \textbf{新闻与信息自由}:存在独立的、多元化的媒体(报纸、电视、网络),能够自由地报道新闻、监督政府,公民可以从多种渠道获取信息。
        \item \textbf{集会与结社自由}:公民有权和平地集会、游行、示威,来表达自己的诉求。他们也有权自由地组建和参加各种组织,如政党、工会、环保组织、社区协会等(即活跃的“公民社会”)。
    \end{itemize}

    \item \textbf{法治(Rule of Law)}:
    法治是民主的基石。它意味着\textbf{法律至上},而非人治。
    \begin{itemize}
        \item \textbf{法律面前人人平等}:无论是普通公民还是政府高官,都必须遵守同样的法律,受到法律同等的保护和约束。
        \item \textbf{独立的司法}:法院和法官必须独立于政府和议会的干预,能够公正地解释和适用法律,裁决纠纷,并保护公民的权利不受政府侵犯。
        \item \textbf{可预测性与非任意性}:政府的行为必须受到法律的约束,不能任意妄为。
    \end{itemize}

    \item \textbf{权力制衡(Checks and Balances)}:
    为了防止权力被滥用,民主制度必须将政治权力分散到不同的机构,并使其相互制约。正如我们在上一章看到的,总统制和议会制就是两种不同的权力制衡安排。其核心思想是,\textbf{以权力约束权力}。

    \item \textbf{问责制(Accountability)}:
    政府必须对其行为向公民负责。
    \begin{itemize}
        \item \textbf{纵向问责(Vertical Accountability)}:指公民通过选举,来追究政府的责任。如果选民对执政党的表现不满意,可以在下次选举中把它选下台。
        \item \textbf{横向问责(Horizontal Accountability)}:指国家内部的不同机构之间的相互监督和问责。例如,议会可以监督政府,法院可以审查法律是否违宪,独立的审计部门可以审查政府的财政支出。
    \end{itemize}
\end{enumerate}

这五大要素,共同构成了衡量一个国家民主成色的“黄金标准”。

\hrulefill

\section{民主的“体检”:如何衡量一个国家的民主健康状况?}

既然我们知道了民主的理想标准,那么在现实中,我们如何去衡量一个国家是否民主,以及其民主的质量有多高呢?政治学家和国际组织已经发展出了一系列相对成熟的“体检”指标和数据库,来对全球各国的民主状况进行打分和排名。

这些“民主指数”通常会综合考量以下几个维度:

\begin{enumerate}
    \item \textbf{选举过程与多元主义(Electoral Process and Pluralism)}:
    \begin{itemize}
        \item 选举是否自由、公平、透明?
        \item 是否存在真正的多党竞争?
        \item 反对党是否有现实的获胜机会?
        \item 选举管理机构是否独立?
    \end{itemize}

    \item \textbf{政府运作(Functioning of Government)}:
    \begin{itemize}
        \item 政府的权力是否受到议会和司法的有效制约?
        \item 政府是否清廉?腐败程度如何?
        \item 官僚体系是否专业,能否有效执行政策?
        \item 政府决策过程是否透明?
    \end{itemize}

    \item \textbf{政治参与(Political Participation)}:
    \begin{itemize}
        \item 投票率高低?
        \item 公民是否积极参与政治活动(如示威、请愿)?
        \item 公民社会(非政府组织)是否活跃?
        \item 妇女和少数族裔在政治中的代表性如何?
    \end{itemize}

    \item \textbf{政治文化(Political Culture)}:
    \begin{itemize}
        \item 民众对民主制度的支持度有多高?
        \item 社会中是否存在高度的信任和宽容?
        \item 民众是否倾向于通过和平、协商的方式解决分歧?
    \end{itemize}

    \item \textbf{公民自由(Civil Liberties)}:
    \begin{itemize}
        \item 言论和新闻自由是否得到保障?
        \item 是否存在独立的媒体?
        \item 集会和结社自由是否受限?
        \item 司法是否独立?法治状况如何?
        \item 个人是否免于恐惧和国家的任意监控?
    \end{itemize}
\end{enumerate}

\textbf{著名的民主测量项目:}

\begin{itemize}
    \item \textbf{“自由之家”(Freedom House)}:自1973年以来,每年发布《世界自由度报告》,从“政治权利”和“公民自由”两个维度对各国进行评分,并将其划分为“自由”、“部分自由”和“不自由”三类。
    \item \textbf{“经济学人智库”(EIU)}:每年发布“民主指数”(Democracy Index),从上述五个维度对167个国家和地区进行打分,并将其归类为“完全民主”、“有缺陷的民主”、“混合政体”和“威权政体”四种类型。
    \item \textbf{“V-Dem项目”(Varieties of Democracy)}:这是一个由全球数千名专家参与的、最全面、最细致的民主数据库。它提供了数百个关于民主不同方面的具体指标,允许研究者对民主进行多维度的深入分析。
\end{itemize}

这些指数虽然在具体方法和结果上略有差异,但它们共同为我们提供了一个观察全球民主趋势、比较不同国家民主表现的“仪表盘”。它们也一致地揭示了一个令人担忧的趋势:\textbf{自21世纪以来,全球范围内的民主,无论是在数量上还是在质量上,都呈现出停滞甚至衰退的迹象。}

\hrulefill

\section{民主的万花筒:从理想走向现实的不同形态}

在现实世界中,民主并非只有一种理想形态,而是呈现出一个从“完全民主”到“完全不民主”的连续光谱。在这个光谱上,我们可以识别出几种典型的政体形态。

\subsection{类型一:自由民主——民主的“优等生”}

这是最接近我们前述“黄金标准”的民主形态,也是许多西方发达国家所实践的模式。

\begin{itemize}
    \item \textbf{定义}:自由民主不仅拥有自由、公平、竞争性的选举,更重要的是,它对\textbf{公民的个人自由和权利提供了广泛而坚实的宪法保障},并且拥有\textbf{强大的法治和权力制衡机制}。它的核心在于“\textbf{自由主义}”(Liberalism)对“\textbf{民主}”(Democracy)的约束,即\textbf{多数人的统治(民主)必须受到个人权利和法治(自由主义)的限制}。
    \item \textbf{特征}:
    \begin{itemize}
        \item \textbf{宪政主义}:政府的权力受到一部成文或不成文宪法的严格限制。
        \item \textbf{权利保障}:言论、新闻、集会、结社、宗教等自由得到充分保障,并受到独立司法的保护。
        \item \textbf{法治至上}:法律面前人人平等,政府自身也必须守法。
        \item \textbf{权力制衡}:行政、立法、司法三权分立且相互制衡。
        \item \textbf{活跃的公民社会}:独立的非政府组织、工会、媒体等,对政府构成有效的社会监督。
        \item \textbf{对少数群体的保护}:多数人的意志不能侵犯少数群体的基本权利。
    \end{itemize}
    \item \textbf{案例}:北欧国家(如挪威、瑞典、丹麦)、加拿大、新西兰、德国、英国、法国、美国等,通常被认为是自由民主的典范。当然,即使在这些国家,其民主也并非完美无瑕,同样面临着民粹主义、政治极化、金钱政治等挑战。
\end{itemize}

\subsection{类型二:选举民主——“及格线”上的民主}

这是光谱上的下一个层级,也是世界上数量最多的一种“民主”形态。

\begin{itemize}
    \item \textbf{定义}:选举民主是指一个国家满足了民主的\textbf{最低程序性要求},即定期举行自由、公平、多党竞争的选举,能够实现权力的和平更迭。然而,在选举之外,其对公民自由的保障、法治的健全程度和权力制衡机制方面,却存在着明显的\textbf{缺陷}。
    \item \textbf{特征}:
    \begin{itemize}
        \item \textbf{选举是核心}:选举是真实和竞争性的,反对党有获胜的可能。这是它与威权政体的根本区别。
        \item \textbf{“硬件”齐全,“软件”不足}:它拥有民主的“硬件”(如选举、议会、多党制),但在保障民主有效运作的“软件”(如独立的司法、自由的媒体、强大的公民社会、健全的法治文化)方面却相对薄弱。
        \item \textbf{不稳定的权利}:公民的言论、集会等自由虽然在法律上存在,但在实践中可能受到政府的骚扰或限制。媒体可能受到政府的压力或所有权的集中控制。司法系统可能容易受到政治干预。
        \item \textbf{高水平的腐败和侍从主义(Clientelism)}:政治人物可能通过滥用国家资源、分配工作或福利,来换取选民的支持(即“庇护-侍从”关系),而非通过政策辩论。
    \end{itemize}
    \item \textbf{案例}:许多处于民主转型中的发展中国家,都属于选举民主的范畴。
    \begin{itemize}
        \item \textbf{菲律宾}:拥有竞争激烈的选举和活跃的媒体,但其政治长期被少数几个大家族所主导,腐败和暴力问题严重,法治相对薄弱。
        \item \textbf{乌克兰}:在2014年“广场革命”后,乌克兰的选举变得更具竞争性,媒体也更加自由。但其民主仍然受到寡头政治、司法腐败和俄罗斯干预的严重困扰。
        \item \textbf{印度尼西亚}:在1998年苏哈托倒台后,印尼成功转型为亚洲最大的民主国家之一,选举竞争激烈。但其民主质量仍受到腐败、宗教不宽容和军队潜在干政等问题的挑战。
    \end{itemize}
\end{itemize}

选举民主是民主的“及格线”,但它是一种脆弱、不稳定的状态。它可能在公民社会和民主力量的推动下,逐步巩固和深化,向自由民主迈进;也可能因为内部的矛盾和威权势力的反扑,而停滞不前,甚至倒退。

\subsection{类型三:劣质民主——民主的“变种病毒”}

这是近年来引起全球警惕的一种民主倒退现象。这个概念由美国学者\textbf{法里德·扎卡利亚(Fareed Zakaria)}在1997年首次提出,用来描述那些\textbf{虽然保留了选举的形式,但其自由主义的内核——即对公民自由、法治和权力制衡的尊重——却被系统性侵蚀}的政体。

\begin{itemize}
    \item \textbf{定义}:劣质民主是一种“有民主,无自由”的政体。它通过选举上台,却利用手中的权力来压制反对声音、削弱制衡机制、侵犯公民权利。在这种政体下,“\textbf{多数人的暴政}”(Tyranny of the Majority)可能取代对个人权利和少数群体的保护。
    \item \textbf{特征}:
    \begin{itemize}
        \item \textbf{选举依然存在,但日益不公}:选举可能仍然举行,但执政者会利用其掌握的国家资源,系统性地为自己创造不公平的竞争优势,如控制主流媒体、修改选举法、骚扰反对派候选人。
        \item \textbf{攻击“民主守门人”}:劣质民主的领导人,往往会集中攻击那些对权力构成制约的独立机构,如\textbf{独立的司法系统、自由的媒体和活跃的公民社会}。他们会将法官、记者、非政府组织描绘成“人民的敌人”或“外国代理人”。
        \item \textbf{以“人民”的名义集中权力}:这些领导人通常是民粹主义者,他们声称自己是“沉默的大多数”或“真正的人民”的唯一代表,并以此为借口,通过“合法”的手段(如修改宪法、颁布紧急状态法)来集中权力,打击政治对手。
        \item \textbf{民族主义与排外}:他们常常诉诸强烈的民族主义情绪,强调国家主权,排斥外部影响(如欧盟、国际人权组织)和内部的“他者”(如移民、少数族裔、性少数群体)。
    \end{itemize}
    \item \textbf{案例}:
    \begin{itemize}
        \item \textbf{匈牙利的“奥尔班模式”}:自2010年上台以来,匈牙利总理维克多·奥尔班(Viktor Orbán)领导的青民盟(Fidesz)政府,被广泛认为是劣质民主的典型。他利用在议会的绝对多数,通过了新宪法,削弱了宪法法院的权力;将大部分媒体置于政府或其盟友的控制之下;修改选举法,使其有利于本党;并严厉打压非政府组织和索罗斯资助的中欧大学。他将自己的模式称为“非自由主义民主”(Illiberal Democracy)。
        \item \textbf{土耳其的埃尔多安时代}:雷杰普·塔伊普·埃尔多安(Recep Tayyip Erdoğan)在执政初期曾被视为温和的民主改革者。但之后,他逐步加强了对媒体和司法系统的控制,并在2016年未遂政变后,进行了大规模的清洗,逮捕了数万名记者、学者、法官和公务员。2017年,他通过修宪公投,将土耳其从议会制转变为总统制,极大地集中了个人权力。
    \end{itemize}
\end{itemize}

劣质民主的危险之处在于,它的倒退过程是\textbf{渐进的、合法的},它不是通过一场突然的军事政变,而是通过一次次的选举、一次次的立法,像“温水煮青蛙”一样,逐步窒息民主的生命。

\hrulefill

\section{“竞争性威权主义”:披着民主外衣的“狼”}

在民主与威权的光谱上,还存在一个重要的灰色地带,它看起来很像劣质民主,但其本质更接近威权主义。这个概念由政治学家\textbf{史蒂文·莱维茨基(Steven Levitsky)}和\textbf{卢坎·韦(Lucan Way)}提出,即“\textbf{竞争性威权主义}”(Competitive Authoritarianism)。

\begin{itemize}
    \item \textbf{定义}:竞争性威权主义是一种\textbf{混合政体(Hybrid Regime)}。在这种政体中,形式上的民主制度(如选举、议会、多党制、宪法)广泛存在,并且被严肃对待。然而,执政者通过系统性地滥用国家权力,严重地、不公平地削弱反对派的竞争能力,使得\textbf{竞争虽然真实存在,但赛场却是极度倾斜的}。
    \item \textbf{与劣质民主的区别}:
    \begin{itemize}
        \item \textbf{出发点不同}:劣质民主通常被看作是\textbf{从一个相对健康的民主制度开始,逐步倒退和腐化}的过程。而竞争性威权主义,往往是\textbf{从一个威权体制转型而来,但转型并不彻底},威权主义的“基因”从一开始就深刻地嵌入在看似民主的制度框架之中。
        \item \textbf{竞争的公平性}:在劣质民主中,选举的竞争环境虽然恶化,但反对派在理论上仍有获胜的可能。而在竞争性威权主义中,执政者滥用权力的程度是如此系统和深入,以至于\textbf{反对派几乎没有现实的可能通过选举来赢得最高权力}。赛场是如此不平,以至于比赛结果在很大程度上是预先注定的。
    \end{itemize}

    \item \textbf{特征:一个倾斜的赛场}
    竞争性威权主义的执政者,不会像传统独裁者那样完全取消选举或禁止反对党,而是通过四种主要方式来“玩弄”民主游戏:
    \begin{enumerate}
        \item \textbf{选举领域}:选举定期举行,但充满猫腻。执政党利用国家资源进行竞选宣传;控制选举委员会;在投票和计票环节进行舞弊;随意取消反对派候选人的资格。
        \item \textbf{立法领域}:议会存在且可以辩论,但通常是“橡皮图章”。执政党利用其控制的多数,随意修改法律,或通过贿赂、威胁等方式压制议会中的反对声音。
        \item \textbf{司法领域}:法院在形式上独立,但执政者通过控制法官的任命、晋升和预算,或直接对法官施压,来确保司法判决有利于自己。法院常常被用来对付政治对手和商业敌人。
        \item \textbf{媒体领域}:存在独立的媒体,但它们面临着来自政府的持续骚扰,如吊销执照、税务稽查、诽谤诉讼、甚至对记者的暴力威胁。同时,国家控制着主要的电视台和报纸,进行铺天盖地的政治宣传。
    \end{enumerate}

    \item \textbf{案例}:
    \begin{itemize}
        \item \textbf{普京领导下的俄罗斯}:被广泛视为竞争性威权主义的典型。俄罗斯定期举行总统和议会选举,也存在多个反对党。然而,克里姆林宫牢牢控制着国家电视台,这是大多数俄罗斯人获取信息的主要来源。主要的反对派领袖(如纳瓦利内)被监禁或被禁止参选。选举过程被指控存在大量舞弊。独立的媒体和非政府组织受到《外国代理人法》等法律的严厉打压。
        \item \textbf{查韦斯/马杜罗时期的委内瑞拉}:查韦斯通过选举上台后,利用其个人魅力和石油财富,系统性地重塑了委内瑞拉的政治体制。他修改宪法,控制了最高法院,将国家石油公司的收入用于政治项目,并建立了一个庞大的、忠于自己的民兵组织。选举依然举行,但反对派面临着一个由国家机器全面支持的、极其强大的对手。
    \end{itemize}
\end{itemize}

竞争性威权主义是冷战后最常见的一种非民主政体。它比传统的独裁统治更具迷惑性,也更稳定,因为它为社会不满提供了一个(有限的)发泄渠道,并在国际上维持了一个“民主”的假象。

\hrulefill

\section*{结论:理解民主的多样性、复杂性与脆弱性}

通过本章对民主“万花筒”的探索,我们可以得出几个重要的结论:

\begin{enumerate}
    \item \textbf{民主是一个光谱,而非开关}:民主并非一个简单的“是”或“否”的问题。一个国家很少是100\%的民主或0\%的民主,而是处于这个光谱上的某个位置。比较政治学的任务,就是精确地定位一个国家的位置,并分析其移动的方向。
    \item \textbf{选举只是必要条件,而非充分条件}:拥有选举,是成为民主国家的入场券,但这远远不够。一个高质量的、可持续的民主,更依赖于对公民自由的保障、对权力的制衡以及健全的法治。只重选举、不重自由的“选举民主”或“劣质民主”,是脆弱和危险的。
    \item \textbf{警惕“合法”的民主侵蚀}:当代对民主最大的威胁,可能不再是坦克和政变,而是那些以“人民”或“法律”的名义,渐进地、一步步地掏空民主实质的政治力量。劣质民主和竞争性威权主义的兴起,正是这一趋势的体现。
    \item \textbf{民主没有“终点”}:即使是那些被视为“完全民主”的国家,其民主制度也并非一劳永逸。它们同样需要其公民保持警惕,积极参与,不断地与各种反民主的力量进行斗争,来维护和更新自己的民主。民主是一种持续的、永不完结的实践过程。
\end{enumerate}

理解民主的这些不同形态,有助于我们避免简单地将所有举行选举的国家都贴上“民主”的标签,从而能更准确、更批判地评估其政治现实。

然而,除了我们本章讨论的制度设计,还有哪些“软实力”因素,在深刻地影响着一个国家的政治生态和公民的政治参与意愿呢?为什么有些地方的人们特别“爱管闲事”(参与政治),而另一些地方的人们却对政治冷漠?这正是我们下一章将要揭示的,隐藏在制度背后的文化与社会秘密。

