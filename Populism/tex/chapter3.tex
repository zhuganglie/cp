\chapter{空悬的王座:主流政治的失灵}

\textbf{本章论点:} 民粹主义者并非凭空崛起,他们填补的是因信誉扫地和反应迟钝的主流政党所留下的政治真空。

想象一个古老的王国。宫殿依然富丽堂皇,国王头戴王冠,身披紫袍,但他的臣民在私下里却用嘲讽的语调谈论他。他们知道,国王的决策不再是为了王国的福祉,而是为了维系他与一小撮贵族的奢华生活;他们看到,法官的判决可以被金钱左右,将军的忠诚可以被权力收买。国王的命令仍在颁布,但已无人真心信服;王国的法典依然存在,但已沦为一纸空文。这顶王冠虽然还在国王头上,但它所象征的权威与信任早已荡然无存。王座,实际上已经空了。它只是在等待一个足够大胆的人,一个宣称能代表人民真正意愿的人,走上前去,将它据为己有。

这便是21世纪初许多国家政治图景的真实写照。在我们剖析了全球化带来的经济怨愤和身份认同引发的文化焦虑之后,我们必须将目光投向故事的第三个,也是或许最关键的一个主角:那个本应扮演社会矛盾“减压阀”和民意“整合器”角色的主流政治体制。民粹主义的崛起,不仅是因为选民变得愤怒了,更是因为那些传统的“政治代理人”——主流政党和政治家们——失灵了。他们非但没能有效回应民众的诉求,反而自身成为了民众不信任和鄙夷的对象。

民粹主义者们并非是攻破了一座固若金汤的城堡,他们更像是走进了一座无人看守、大门洞开的宫殿。这个巨大的政治真空,是由主流政治的信誉破产、政策趋同和反应迟钝共同造成的。在民粹主义的幽灵登堂入室之前,王座早已空悬。

\section{大趋同:当左右不再分明}

冷战的结束曾被誉为“历史的终结”,它也深刻地重塑了西方国家的政治光谱。传统的左翼与右翼之争——围绕着国家干预与自由市场、劳工与资本的根本对立——似乎在一夜之间变得过时了。在弗朗西斯·福山所描绘的那个由自由民主和市场经济大获全胜的新时代里,一种新的政治共识悄然形成,并迅速主导了几乎所有主流政党的议程。这便是所谓的“第三条道路”。

英国的托尼·布莱尔、美国的比尔·克林顿、德国的格哈德·施罗德是这条道路最著名的倡导者。他们领导下的中左翼政党(工党、民主党、社民党)进行了一次历史性的“转向”。他们不再高喊国有化和阶级斗争的口号,转而拥抱全球化、放松金融管制、强调财政纪律——这些在过去都是中右翼政党的招牌菜。他们的逻辑是:我们接受市场经济的活力和效率,但要用其创造的财富来投资于教育、医疗等社会项目,实现“社会公平”。这是一种试图超越传统左右之争的实用主义,旨在将市场效率与社会关怀相结合。

在当时,这看起来是一个聪明的、与时俱进的选择。它帮助中左翼政党赢得了选举,也确实在一段时间内维持了经济的增长。然而,这种“大趋同”的深远后果,却在当时被严重低估了。当中左翼开始采纳右翼的经济政策时,中右翼政党为了保持区别,有时不得不在文化和社会议题上变得更保守,但在核心的经济议程上,双方的差距变得前所未有地微小。

对于选民而言,这意味着什么?这意味着,无论你投票给工党还是保守党,你得到的都将是一个信奉全球化和自由市场的政府;无论你选择民主党还是共和党(在小布什之前),你都将面对一个对华尔街和跨国公司友好的执政团队。政治选择,从根本路线的抉择,退化成了管理风格和政策微调的差异。政治辩论的核心,不再是“我们应该走向何方?”这个关乎国家命运的宏大问题,而是“我们如何能更好地管理现状?”这个技术官僚式的问题。

这种政策上的“面目模糊”侵蚀了政党的根基。政党本应是代表不同社会群体利益、提供不同社会愿景的组织。但当它们看起来都差不多时,选民的忠诚度便开始瓦解。一位来自“铁锈地带”的失业工人可能会问:那个曾经代表工人阶级的民主党,现在和支持华尔街的共和党有什么本质区别?一位担心本国文化被侵蚀的保守派选民可能会想:那个拥抱多元文化和欧盟一体化的保守党,和工党的主张又有多大不同?

当主流选项无法提供真正的替代方案时,一种深刻的无力感和政治犬儒主义便开始蔓延。人们觉得,政治不过是一场由一小撮精英阶层上演的“假面舞会”,无论谁上台,他们的生活都不会有实质性的改变。这种“TINA”(There Is No Alternative,别无选择)的氛围,为那些敢于打破常规、宣称要提供一个“真正选择”的局外人,创造了完美的登场机会。民粹主义者恰恰就利用了这一点,他们对着感到厌倦和失望的选民大声疾呼:“他们都一样!只有我,才能代表你们,带来真正的改变!”

\section{“政治阶层”:一个自我封闭的特权世界}

与政策趋同相伴而生的,是“政治阶层”(the political class)的固化。随着政党日益专业化和官僚化,政治不再是各行各业的公民代表参与公共事务的领域,而逐渐变成了一项“职业”。今天的政治家,很多都是从大学一毕业就开始其政治生涯:担任议员助理、进入党部工作、在智库任职,然后一步步攀上权力的阶梯。他们的人生轨迹高度相似,社交圈子高度重合,他们与普通民众的日常生活渐行渐远。

这个群体,无论其党派归属是左是右,都共享着一套相似的语言体系、行为准则和世界观。他们频繁出入于权力中心(华盛顿、布鲁塞尔、伦敦),与记者、说客、企业高管和高级官僚们构成了一个紧密的共生网络。这种脱离民众的“精英闭环”,让普通人感觉政治是一个遥远的、与己无关的“他们”的世界。

而层出不穷的政治丑闻,则成为了压垮骆驼的最后一根稻草,彻底摧毁了民众对这个阶层的信任。这些丑闻的形式五花八门:从直接的贪污受贿,到利用公款报销私人奢靡开销(如英国的“议会开支丑闻”),再到利用职权为亲信或金主提供便利。每一个被曝光的丑闻,都在反复印证着民众心中那个最糟糕的猜想:这群人不是在为我们服务,而是在为他们自己服务。他们满口“公共利益”,心里想的却是个人私利。

更具腐蚀性的是所谓的“旋转门”现象:政府高官在卸任后,迅速进入他们曾经监管的大公司或金融机构担任高薪职位。这不仅引发了严重的利益冲突质疑,更让民众确信,政府和资本已经结成了牢不可破的利益同盟。当一个负责制定金融政策的财政部长,卸任后立刻被华尔街的投行以千万年薪聘用时,人们有理由怀疑,他当初制定的政策,究竟是为了国家,还是为了给自己未来的雇主铺路?

全天候的媒体周期和社交媒体的兴起,则将这一切无限放大。政治人物的每一次失言、每一次虚伪的表演、每一个丑闻的细节,都会被24小时不间断地报道、分析和嘲讽。这使得政治从一场严肃的公共辩论,变成了一场永不落幕的娱乐真人秀。公众对政治的观感,不再是尊敬,而是厌倦、鄙夷和不信任交织的复杂情绪。

当“政治家”这个词几乎等同于“骗子”时,任何将自己塑造为“反政治家”的人,都会自动获得巨大的道德优势。特朗普之所以能吸引众多支持者,一个重要原因就是他刻意表现出的“非政客”形象。他说话不打草稿,用词粗俗直白,打破各种政治正确。在许多选民看来,这种“粗鲁”恰恰是“真实”的体现,与那些言辞优雅却谎话连篇的建制派政客形成了鲜明对比。

\section{案例聚焦(一):意大利——旧体系的崩塌与“五星”的升起}

如果说有一个国家完美地展示了主流政治失灵如何为民粹主义铺平道路,那一定是意大利。意大利的经历,如同一场漫长而痛苦的政治实验,预演了许多其他国家后来发生的故事。

故事的序幕在20世纪90年代初拉开。一场名为“净手运动”(Mani pulite)的大规模反腐调查,如同一场政治地震,彻底摧毁了意大利战后建立的整个政党体系。执政数十年的天主教民主党和长期作为主要反对党的社会党等传统大党,几乎在一夜之间土崩瓦解,无数政客锒铛入狱。这是一个国家政治体系的“熔断”。

在这片巨大的政治废墟上,崛起了第一代民粹主义者——媒体大亨西尔维奥·贝卢斯科尼。他创建了“意大利力量党”,将自己包装成一个成功的商人、一个政治局外人,承诺要用经营企业的方式来“拯救”意大利。他利用自己掌控的媒体帝国,成功地绕过了传统的政治渠道,直接与选民对话。他的崛起,本身就是对旧有政治精英彻底不信任的产物。

然而,贝卢斯科尼的时代同样充满了丑闻和未兑现的承诺。在他与中左翼政党轮流执政的近二十年里,意大利的经济停滞不前,公共债务飙升。民众发现,换了一批人上台,但政治腐败和效率低下的问题依然如故。对建制派的失望,演变成了对整个政治体系的绝望。

正是在这种背景下,一个更彻底、更激进的民粹主义力量——“五星运动”(MoVimento 5 Stelle)应运而生。它的创始人,是一位家喻户晓的喜剧演员贝佩·格里洛。他通过博客和全国巡回的“吐槽”式演讲,用最尖酸刻薄的语言,将所有传统政客斥为“僵尸”、“寄生虫”,将议会称为“一个装满骗子的盒子”。

“五星运动”的核心理念是“反政治”。它拒绝被定义为左翼或右翼,宣称自己代表的是“公民”,对抗的是整个“特权阶层”(la casta)。它承诺要用网络直选来决定政策,实现“直接民主”,让每一个普通公民都能参与决策。它的口号极具煽动性:“把他们都送回家!”(Vaffanculo-Day,意为“滚蛋日”,是其早期著名的政治活动)。

在2013年的大选中,“五星运动”一跃成为国会第一大党,震惊了整个欧洲。在2018年,它更是成功上台执政。它的支持者来自社会各个阶层:有对经济不满的年轻人,有对政治腐败感到恶心的中产阶级,也有感觉被传统左翼抛弃的蓝领工人。他们唯一的共同点,就是对现有政治体系的彻底唾弃。

意大利的故事是一个警示:当主流政党因为腐败、僵化和脱离民众而失去信誉时,它们不仅仅是输掉一场选举,而是在摧毁整个政治生态的土壤。这片被污染的土壤,将不再能长出温和理性的政治果实,而只会为那些最极端、最煽动、最“反体制”的力量提供养分。

\section{案例聚焦(二):巴西——腐败丑闻铺就的威权之路}

在地球的另一端,巴西的案例则以一种更为惨烈的方式,揭示了系统性腐败如何成为民粹主义强人上台的催化剂。

从21世纪初开始的十余年里,巴西由卢拉·达席尔瓦和其继任者迪尔玛·罗塞夫领导的左翼劳工党(PT)执政。这是一个鼓舞人心的故事:劳工党通过大规模的社会福利计划,成功地让数千万人摆脱了贫困,巴西也一度作为“金砖国家”的代表,在国际舞台上崭露头角。劳工党,成为了巴西新的“建制派”。

然而,光环之下,腐败的毒瘤正在疯狂滋长。2014年,一场名为“洗车行动”(Operação Lava Jato)的司法调查,揭开了一个堪称史上最大规模的腐败网络。调查发现,巴西国家石油公司(Petrobras)的高管与多家顶级建筑公司相勾结,通过虚报工程合同价格,将巨额资金作为回扣和贿赂,系统性地输送给包括劳工党在内的几乎所有主流政党的政治家。

这场丑闻的规模之大、牵涉之广,令整个国家为之震动。它不仅涉及金钱,更涉及对民众信任的彻底背叛。那个曾经承诺要为穷人代言、涤荡腐败的劳工党,自己却成为了腐败体系的核心。与此同时,巴西经济陷入严重衰退,失业率飙升,暴力犯罪猖獗。经济危机、安全危机和政治信任危机三者叠加,点燃了民众滔天的怒火。数百万巴西人走上街头,要求总统下台,要求惩治所有腐败分子。

在这样一个充满愤怒、恐惧和背叛感的社会氛围中,一个长期处于政治边缘的人物——雅伊尔·博索纳罗——看到了机会。博索纳罗当了近三十年的国会议员,一直是个无足轻重、以极端言论博眼球的“怪人”。他公开怀念巴西历史上的军事独裁时期,发表歧视女性、同性恋和少数族裔的言论,鼓吹用暴力来解决犯罪问题。在正常的政治环境下,他这样的人绝无可能染指总统宝座。

但2018年的巴西,已经不再“正常”。当所有主流政客都被腐败丑闻所玷污时,博索纳罗这个从未身居高位、从未掌握过实权的“局外人”,反而显得“干净”了。他的粗鲁和极端,被许多人解读为一种不与腐败同流合污的“真诚”和“强硬”。他精准地抓住了民众最关切的两个痛点:腐败和犯罪。他承诺要用“铁腕”来扫除腐败、镇压罪犯,要“把巴西从社会主义手中拯救回来”。

他将自己塑造为“上帝、祖国和家庭”的捍卫者,对抗腐败的左翼精英、堕落的自由派媒体和制造混乱的犯罪分子。对于那些对现状感到极度失望和恐惧的选民来说,博索纳罗的出现,仿佛是乱世中的唯一希望。他们投票给他,不是因为他们完全认同他的所有极端观点,而是因为他们相信,只有这样一个“强人”,才能砸烂那个腐朽不堪的旧体系,带来秩序和改变。

最终,博索纳罗以压倒性优势赢得了2018年的总统大选。他的胜利,是巴西主流政治信誉彻底破产的直接后果。它雄辩地证明,当民众对民主制度内的纠错能力感到绝望时,他们会愿意将权力交给一个承诺要“打破一切”的威权人物,哪怕这本身就对民主构成了巨大的威胁。

\section{结语:被遗弃的权力真空}

民粹主义的浪潮之所以能在21世纪席卷全球,经济的困顿和文化的焦虑固然是催生不满的“沃土”,但主流政治的失灵,才是那个打开了泄洪闸的“扳机”。它为汹涌的民意洪水,清空了流向权力的河道。

在过去的几十年里,从欧洲到美洲,主流的中左翼和中右翼政党陷入了一场“大趋同”,它们提供的政策选项越来越相似,让选民感到别无选择。与此同时,一个脱离民众的“政治阶层”悄然形成,他们被一连串的腐败丑闻和利益输送所玷污,彻底失去了民众的信任。政治,在许多人眼中,不再是实现公共福祉的神圣事业,而沦为了一场自私自利的肮脏游戏。

这个空悬的王座,这个被主流精英们因傲慢、腐败和迟钝而亲手制造出的权力真空,是民粹主义者们收到的最丰厚的礼物。他们不需要进行艰苦的攻坚战,只需要走到台前,对着台下早已怒火中烧的民众说出他们最想听的话:“你们的痛苦,都是这群腐败精英造成的。他们背叛了你们。现在,把权力交给我,我将为你们夺回属于你们的一切。”

理解了这一点,我们才能明白,应对民粹主义的挑战,仅仅在经济上进行一些修补,或是在文化上呼吁一些包容,是远远不够的。它要求一场深刻的政治变革。它要求主流政党必须重新找回自己清晰的身份和愿景,提供真正有区别的政策选项;它要求政治精英必须重建与民众的信任链接,用行动证明他们的服务对象是公众,而非自身的利益。

然而,在许多国家,当主流政治力量意识到问题的严重性时,往往为时已晚。民粹主义者们已经坐上了王座。接下来,他们将如何运用手中的权力?他们会兑现承诺,还是会带来更大的灾难?在本书的下一部分,我们将开启一场全球巡礼,近距离观察这些“权力操盘手”们迥然不同却又暗合某种规律的“剧本”。