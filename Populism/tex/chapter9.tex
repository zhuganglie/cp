\chapter{分裂的社会:“我们”的崩解}

\textbf{本章论点:} “我们vs他们”的二元对立逻辑不只是口头言辞,它在现实中毒化了社会凝聚力,加剧了政治极化。

民粹主义的核心逻辑,可以用一句简单的话来概括:世界被划分为泾渭分明的两个阵营——“纯洁的人民”和“腐败的精英”。民粹主义领袖们,无论其意识形态底色是左是右,都熟练地操弄着这把“二元对立”的利剑。他们将社会中的复杂矛盾和多元观点,粗暴地简化为一场“正义与邪恶”的终极对决。在这场对决中,他们及其追随者,当然代表着“人民”的正义一方,而所有对他们构成挑战或批评的群体,则被毫不犹豫地打入“精英”的邪恶阵营。

这种“我们vs他们”的叙事,在竞选集会和社交媒体上,往往以口号和标签的形式呈现,比如特朗普的“让美国再次伟大”和“假新闻”,或是英国脱欧派的“夺回控制权”和“人民的公敌”。这些简洁有力的词句,如同淬了毒的箭矢,深深地刺入了社会肌理。然而,这仅仅是故事的开始。当民粹主义者们成功地赢得选举、掌握权力之后,这种“敌我二元”的逻辑,便不再仅仅是一种修辞策略,而成为了一种塑造社会现实的强大力量。它像一种蔓延开来的病毒,侵蚀着社会信任,撕裂着共同体纽带,最终导致一个曾经多元、开放、充满活力的社会,一步步走向分裂和对立。

本章,我们将深入剖析民粹主义的“我们vs他们”逻辑,是如何从一种政治动员手段,演变为一种深刻的社会病症的。我们将看到,这种逻辑不仅仅是言语上的攻击,更是一种系统性的社会工程,它从细微之处影响着人与人之间的互动,从根本上改变着一个社会的文化氛围和政治生态。我们将通过数据、案例和个人故事,来揭示民粹主义如何破坏社会信任,加剧政治极化,并最终使得一个国家的凝聚力走向崩解。我们还将探讨,生活在一个同胞被描绘成敌人的社会里,普通人所承受的情感与心理代价。

\section{信任的消解:当同胞变成“敌人”}

在任何一个健康的社会里,信任都是至关重要的“社会资本”。它是一种弥漫在人与人之间、公民与机构之间的无形纽带,是社会合作、公共参与和政治稳定的基石。然而,民粹主义的“我们vs他们”逻辑,从根本上瓦解了这种信任。它通过持续不断地制造和强化对立,使得社会成员之间不再以平等的公民相待,而是以敌对的阵营成员相处。

\subsection{从政治对手到“人民公敌”}

民粹主义的第一步,往往是将政治对手“非人化”。在正常的民主竞争中,不同政党或政治人物之间的分歧,被视为在政策、理念或发展路径上的差异。人们可以尊重对方的观点,即使并不认同。然而,在民粹主义的语境下,政治对手不再是拥有不同政治观点的同胞,而是“人民公敌”,是“叛徒”,是“邪恶势力”的同盟。

例如,特朗普在竞选期间,给所有批评他的民主党人,都贴上了“社会主义者”甚至“共产主义者”的标签。他将希拉里·克林顿描绘成一个“腐败”、“不诚实”的“骗子”,暗示她对美国构成了生存威胁。对于支持希拉里的选民,他则用一句轻蔑的“可悲的人”(deplorables)来概括,将他们贬低为与自己所代表的“真正美国人”截然不同的异类。同样,在英国脱欧公投期间,留欧派的政治家被脱欧派媒体和活动人士攻击为“叛徒”,甚至被指责为收受欧盟资金、出卖国家主权。

这种言辞上的暴力,绝不仅仅是竞选中的“过激言论”。当它成为一种常态,当它被一个国家的最高领导人反复使用时,它就会逐渐渗透到社会意识的深层,改变人们看待政治分歧的方式。在民粹主义的社会里,不同政治立场的公民,不再被视为可以对话、可以妥协的同胞,而被视为需要警惕、需要斗争、甚至需要消灭的敌人。政治辩论不再是观点的交锋,而变成了一场你死我活的零和博弈。

\subsection{从媒体批评到“真相之战”}

在民粹主义的社会里,独立媒体也无法幸免于“敌我二元”的逻辑。媒体本应扮演监督权力、传递信息、促进公共讨论的角色。然而,当民粹主义者将自己定义为“人民”的唯一代表时,任何对他们的批评,都会被解读为对“人民”的攻击。

特朗普将所有批评他的主流媒体,都斥之为“假新闻媒体”(fake news media),是“人民的敌人”(enemies of the people)。他鼓励他的支持者不信任这些媒体的报道,甚至攻击记者是“骗子”和“坏人”。同样,在匈牙利,欧尔班政府系统性地打压独立媒体,通过税收、监管和所有权变更等手段,将越来越多的媒体置于政府控制之下。在波兰,法律与公正党政府也采取了类似的做法,试图将公共电视台变成执政党的宣传工具。

这种对媒体的攻击,其目标并不仅仅是压制批评声音,更是要从根本上摧毁人们对客观事实的共同认知。当人们不再相信权威媒体,不再尊重专业记者,而是只从自己信任的渠道(往往是社交媒体或亲政府媒体)获取信息时,一个社会就失去了进行理性讨论的基础。每个人都只相信自己愿意相信的东西,事实不再是客观的存在,而变成了可以被随意塑造和利用的工具。这就是民粹主义者所追求的“真相之战”:通过摧毁共同的“信息地基”,为自己创造一个可以不受约束地定义现实的世界。

\subsection{从社会多元到身份对立}

民粹主义的“我们vs他们”逻辑,往往与某种特定的身份政治(identity politics)相结合,从而在社会内部制造更深层次的分裂。这种身份政治,可以是基于民族、宗教、种族、文化或意识形态的。民粹主义者常常将“人民”定义为一个具有共同身份的同质化群体,而将所有不属于这个群体的人,视为对“人民”的威胁。

例如,莫迪领导下的印度,印度教民族主义日益抬头。穆斯林等少数族裔,被视为对印度教文化和印度国家认同的潜在威胁,他们的权利和安全受到了侵蚀。同样,在杜特尔特统治下的菲律宾,贩毒嫌疑人被“非人化”,他们的生命权被公然剥夺,而那些为他们辩护的人权活动家,也被贴上了“毒品同情者”的标签,成为了社会中的“异类”。

即使在没有出现大规模暴力冲突的西方国家,民粹主义的兴起也加剧了社会群体的对立。例如,在特朗普执政的美国,种族关系明显恶化,白人至上主义势力抬头,针对少数族裔的仇恨犯罪增多。在英国脱欧后,针对移民的攻击事件也有所上升。

民粹主义的身份政治,不仅仅是制造了新的社会分歧,更重要的是,它使得原有的社会多元性变得难以被容忍。在一个健康的社会里,不同族群、宗教、文化和政治立场的个体,可以和平共处,彼此尊重,在法律框架下公平竞争。然而,当一种特定的身份被定义为“国家认同”的核心,当其他身份被视为“异己”甚至“威胁”时,社会就失去了包容性和开放性。多元的声音不再被视为丰富社会文化、促进社会进步的动力,而被视为破坏社会团结、威胁社会稳定的因素。

\section{政治极化:失去共同的“对话空间”}

信任的消解,往往伴随着政治极化(political polarization)的加剧。政治极化指的是,社会成员在政治观点上的分歧日益扩大,中间地带逐渐萎缩,不同政治阵营之间的敌对情绪日益加剧。民粹主义的“我们vs他们”逻辑,是政治极化的重要推手。

\subsection{观点两极分化,中间派消失}

在民粹主义的社会里,人们的政治观点往往呈现出两极分化的趋势。温和、中立或持有混合观点的人越来越少,大多数人都被迫站队,选择一个明确的政治阵营。政治光谱不再是一个连续的平面,而变成了两个互不相容的孤岛。

这种现象,在美国的政治版图中表现得尤为明显。皮尤研究中心(Pew Research Center)的一项长期调查显示,美国民主党和共和党之间的意识形态差距,在过去几十年里持续扩大,达到了历史最高水平。两党在几乎所有重要议题上的立场都截然相反,从税收、医保、移民到枪支管制、气候变化,几乎没有共同立场。更令人担忧的是,两党选民对对方的负面评价越来越高,认为对方不仅在政策上错误,而且在道德上可疑,甚至对国家构成了威胁。

在民粹主义语境下,即使是那些原本可以达成共识的议题,也往往会因为政治立场的不同而被“劫持”。例如,在新冠疫情期间,戴口罩和接种疫苗等公共卫生措施,在美国却被高度政治化。许多共和党人,尤其是特朗普的支持者,将这些措施视为侵犯个人自由,甚至将其与某种政治阴谋论联系起来。科学共识不再重要,重要的是,这些措施是否与自己所属的政治阵营的立场相符。

\subsection{“回音室效应”与“信息茧房”}

政治极化的加剧,也与所谓的“回音室效应”(echo chamber effect)和“信息茧房”(information cocoon)密切相关。在民粹主义的社会里,人们越来越倾向于只接触那些与自己观点一致的信息,而主动屏蔽或回避那些与自己观点相左的信息。

社交媒体在这一过程中扮演了重要的角色。社交媒体算法会根据用户的偏好,推送其感兴趣的内容,这使得人们更容易陷入一个只听到自己声音的“回音室”里。在这样的“回音室”里,人们会觉得自己所相信的就是真理,而其他观点都是错误的、可笑的,甚至是危险的。

这种“信息茧房”的形成,使得不同政治阵营之间的相互理解和沟通变得更加困难。人们不再愿意倾听对方的观点,不再尊重不同的声音,而是将所有与自己不同的人,都视为“愚蠢”或“邪恶”的。曾经作为社会成员之间桥梁的公共空间逐渐消失,取而代之的是一个个相互隔绝的“阵营”和“部落”。

\subsection{污名化与人身攻击}

在政治极化的社会里,不同阵营之间的交流往往充满了敌意和攻击性。人们不再就事论事,而是热衷于给对方贴标签、扣帽子,进行人身攻击和道德审判。

例如,在美国的政治辩论中,民主党人常常被共和党人贴上“自由派”、“社会主义者”、“白左”等标签,暗示他们脱离现实、过于理想主义,甚至对国家认同不忠诚。而共和党人则常常被民主党人贴上“保守派”、“种族主义者”、“法西斯”等标签,暗示他们思想落后、歧视弱势群体,甚至对民主制度构成威胁。

这种标签化和污名化,使得理性的讨论变得不可能。人们不再关注对方所说的内容,而是关注对方的身份和立场。一旦一个人被贴上了某个负面标签,那么他所说的任何话,都会被自动地视为不可信、不值得尊重的。

更令人担忧的是,这种政治极化往往会从线上蔓延到线下,导致社会冲突和暴力事件的增多。当人们将政治对手视为不共戴天的仇敌时,就很容易采取过激的行为,甚至诉诸暴力。

\section{社区崩解:日常生活中的“敌我”逻辑}

民粹主义的“我们vs他们”逻辑,并不仅仅停留在宏观的政治层面,它还会深刻地影响到人们的日常生活,侵蚀着社区的凝聚力。

\subsection{邻里关系中的政治隔阂}

在政治极化的社会里,即使是在日常生活中,人们也越来越难以避免政治立场的冲突。邻里之间、同事之间、朋友之间,往往会因为政治观点不同而产生隔阂,甚至反目成仇。

原本,社区是人们共享生活、互助合作的场所。然而,当政治分歧渗透到社区的每一个角落,当人们不再以普通居民的身份相处,而是以“特朗普支持者”或“拜登支持者”、“脱欧派”或“留欧派”的身份相处时,社区的信任和凝聚力就会被逐渐瓦解。

原本可以友好相处的邻居,可能会因为在社交媒体上的一次争论而不再往来;原本可以合作共事的同事,可能会因为政治立场不同而在工作中相互拆台;原本可以无话不谈的朋友,可能会因为一次激烈的辩论而形同陌路。政治不再仅仅是一种个人观点,而变成了一种划分敌我的界线,它侵蚀着社会关系,使得人们越来越难以跨越政治立场的差异,建立真诚的联系。

\subsection{家庭内部的政治战争}

更令人痛心的是,民粹主义的“我们vs他们”逻辑,甚至会撕裂家庭。在政治极化的社会里,家庭成员之间也可能因为政治观点不同而发生激烈的冲突,甚至彻底决裂。

原本,家庭是最重要的情感纽带,是人们可以无条件信任和依靠的港湾。然而,当政治分歧变得不可调和,当家庭成员之间不再尊重彼此的观点,而是将对方视为“愚蠢”、“顽固”甚至“邪恶”时,家庭的温暖和安全感就会被破坏。

父母可能会因为子女支持不同的政治人物而感到失望和愤怒;夫妻之间可能会因为在重大政治议题上意见相左而争吵不休;兄弟姐妹之间可能会因为政治立场的不同而相互指责和攻击。原本应该相互理解、相互支持的家庭成员,却因为政治而变成了“敌人”,家庭的和谐和幸福被政治纷争所吞噬。

\subsection{公民参与的“阵营化”}

民粹主义的“我们vs他们”逻辑,也影响着人们参与公共生活的方式。在政治极化的社会里,人们参与公共事务,往往不再是为了促进公共利益,而是为了维护自己所属阵营的利益,攻击和压制其他阵营。

例如,在社会运动中,参与者不再是为了争取更广泛的社会支持,而是为了展示自己阵营的力量,甚至与对方阵营发生冲突。在选举中,选民不再是理性地比较不同候选人的政策,而是盲目地支持自己所属阵营的候选人,对其他候选人进行妖魔化和攻击。

公民参与不再是促进社会整合和凝聚力的手段,而变成了加剧社会分裂和对立的工具。公共生活不再是一个不同观点可以平等交流、相互尊重的平台,而变成了一个不同阵营相互争夺、相互排斥的战场。

\section{情感与心理代价:生活在“分裂”的阴影下}

生活在一个被民粹主义“我们vs他们”逻辑所撕裂的社会里,人们不仅在社会关系和公共参与方面面临挑战,在情感和心理层面也会承受巨大的代价。

\subsection{焦虑与恐惧}

民粹主义的社会往往充满了不确定性和潜在的冲突。当社会被划分为敌对的阵营,当不同群体的利益和价值观难以调和,人们很容易对未来感到焦虑和恐惧。

人们可能会担心,自己所属的群体是否会受到攻击和歧视;可能会担心,社会的撕裂是否会导致更大的动荡和冲突;可能会担心,自己所珍视的价值观和生活方式是否会被颠覆。这种焦虑和恐惧,会影响到人们的日常生活,降低幸福感和安全感。

\subsection{愤怒与仇恨}

民粹主义的“我们vs他们”逻辑,会不断地激发人们对“对方”阵营的愤怒和仇恨。当人们被灌输一种“对方”是自己苦难根源、是社会进步阻碍的观念时,很容易产生强烈的负面情绪。

这种愤怒和仇恨,可能会驱使人们采取过激的行为,例如在网络上发表攻击性言论、参与暴力冲突、歧视和排斥“对方”阵营的成员。仇恨的蔓延,会进一步加剧社会的撕裂,形成恶性循环。

\subsection{无助与绝望}

在民粹主义的社会里,即使是那些不认同“我们vs他们”逻辑的人,也可能感到无助和绝望。当社会被极化,当理性讨论变得不可能,当不同阵营之间的鸿沟越来越深时,人们很容易对改变现状失去信心。

他们可能会感到,自己的声音被淹没,自己的努力毫无意义,整个社会正在走向崩溃。这种无助和绝望,可能会导致人们对政治和社会失去兴趣,选择逃避和沉默。

\subsection{身份认同的困惑}

对于那些不愿站队、不认同任何一方极端立场的个体,民粹主义的社会可能会带来身份认同的困惑。他们可能既不认同“我们”阵营的排他性,也不认同“他们”阵营的某些观点,却又难以在两极对立的社会中找到自己的位置。

他们可能会感到孤独和隔离,可能会质疑自己的价值观和信仰,可能会在寻求归属感的过程中迷失方向。

\section{结语:守护共同的“我们”}

民粹主义的“我们vs他们”逻辑,并不仅仅是一种政治策略,它是一种深刻的社会病症。它侵蚀着社会信任,加剧着政治极化,瓦解着社区凝聚力,也给人们带来了巨大的情感和心理负担。

在一个被民粹主义阴影笼罩的社会里,人与人之间的连接被割断,公共生活失去了理性对话的基础,社会成员之间的