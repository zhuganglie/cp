\chapter{失序的世界:从联盟到“本国优先”}

\textbf{本章论点:}民粹主义内向的本质,与二战后建立在联盟与合作基础上的国际秩序存在根本性冲突。

想象一下,你生活在一个精心搭建的社区里。邻里之间有共同的规章制度,有共享的公共设施,有遇到困难时互相扶持的默契。大家虽然各有各的家庭,但都明白,只有共同维护这个社区的和谐与秩序,每个家庭才能真正安居乐业。然而,有一天,社区里的一些家庭开始高喊“我家优先!”的口号。他们质疑公共规章的合理性,拒绝分摊公共设施的维护费用,甚至开始拆除自家与邻居之间的围墙,声称要“夺回”属于自己的领地。

这并非危言耸听的寓言,而是21世纪以来,全球政治舞台上正在真实上演的剧目。在本书的前几章中,我们深入剖析了民粹主义在各国崛起的深层土壤——从经济的怨愤、文化的焦虑,到主流政治的失灵。我们也巡礼了形形色色的民粹主义“操盘手”,他们或喧嚣冲动,或精明隐忍,或铁腕强硬,或悲情救赎,但他们都共享着一个核心逻辑:将社会简化为“纯洁的人民”与“腐败的精英”之间的道德斗争。当这股力量从国内政治的舞台,蔓延到国际关系的场域时,它所带来的冲击,远比我们想象的更为深远和危险。

本章将把我们的目光投向全球,探讨民粹主义内向的、排他性的本质,是如何与二战后建立起来的、以多边主义和国际合作为基石的全球秩序发生根本性冲突的。我们将看到,“本国优先”的口号,不仅仅是国内政策的调整,更是一把锋利的解构之刃,它正在切割着曾经维系世界和平与繁荣的国际联盟、贸易协定和共同应对全球挑战的努力,将我们推向一个更加碎片化、更加不可预测的“失序世界”。

\section{旧秩序的基石:从废墟中崛起的全球共识}

要理解民粹主义对国际秩序的冲击,我们首先需要回顾一下这个“旧秩序”是如何建立起来的。第二次世界大战的硝烟散尽,人类付出了数亿生命的代价,才终于从极端的民族主义和零和博弈的深渊中爬出来。痛定思痛,国际社会的主流精英们达成了一个深刻的共识:要避免历史悲剧重演,就必须建立一个超越狭隘国家利益、以合作和共同安全为目标的国际体系。

于是,一系列旨在促进和平、稳定和繁荣的国际机构和规则应运而生:

\begin{itemize}
    \item \textbf{联合国(UN):} 作为全球最大的政府间组织,旨在维护国际和平与安全,促进国际合作。
    \item \textbf{布雷顿森林体系(Bretton Woods System):} 建立了国际货币基金组织(IMF)和世界银行(World Bank),旨在稳定全球金融体系,促进经济发展。
    \item \textbf{关税与贸易总协定(GATT)及其继承者世界贸易组织(WTO):} 致力于降低贸易壁垒,促进自由贸易,相信经济上的相互依赖能减少冲突。
    \item \textbf{北大西洋公约组织(NATO):} 在冷战背景下,建立了以美国为核心的集体防御联盟,强调“一个都不能少”的共同安全承诺。
    \item \textbf{欧洲联盟(EU):} 从煤钢共同体起步,逐步发展成为一个超国家实体,旨在通过经济和政治一体化,彻底消除欧洲大陆的战争根源。
\end{itemize}

这个体系的核心理念是多边主义:即通过多方协商、共同制定规则、集体行动来解决全球性问题。它强调共享主权(在某些领域,国家自愿让渡部分主权以换取更大的集体利益)、规则导向(而非强权政治)和共同责任。在长达半个多世纪的时间里,这个体系虽然不完美,也面临诸多挑战,但它确实在很大程度上避免了大规模战争,促进了全球经济的空前增长,并在应对贫困、疾病和环境问题上发挥了重要作用。它代表了一种对人类理性、合作和进步的信念。

然而,正如我们在前几章所见,全球化的果实分配不均,以及传统政治精英的失职,让一部分人感到被抛弃、被背叛。民粹主义者正是抓住了这种不满,将矛头指向了这些国际机构和全球化本身。在他们看来,这些机构不是合作的象征,而是“全球主义精英”用来剥夺国家主权、出卖国家利益的工具。

\section{“本国优先”的利刃:对国际秩序的切割}

民粹主义的国际观,是其国内“人民vs精英”叙事的自然延伸。在国际舞台上,“人民”变成了“本国人民”,而“精英”则变成了“全球主义精英”、“国际官僚”以及那些与本国利益相悖的“外国势力”。因此,民粹主义的国际政策,核心就是“本国优先”(Our Country First)——将狭隘的国家利益置于一切之上,对国际合作持怀疑甚至敌视态度。

\subsection{案例聚焦(一):特朗普的“美国优先”——从盟友到交易对象}

唐纳德·特朗普的“美国优先”(America First)政策,是民粹主义国际观最直接、最激进的体现。他将国际关系视为一场零和博弈,认为美国在过去几十年里,被盟友“占了便宜”,被贸易伙伴“欺骗”,被国际组织“束缚”。他的目标是“夺回”美国的经济主权和战略自主权。

\begin{itemize}
    \item \textbf{对北约的冲击:} 北约是二战后西方世界最重要的军事同盟。然而,特朗普多次公开质疑北约的价值,称其“过时”,并指责欧洲盟友没有承担足够的防务开支。他甚至威胁要退出北约。这种言论极大地动摇了盟友对美国安全承诺的信心,削弱了北约的凝聚力,也给俄罗斯等潜在对手发出了危险的信号。集体防御的基石——“一个都不能少”的承诺,在“美国优先”的冲击下变得摇摇欲坠。
    \item \textbf{贸易协定的解构:} 特朗普上任后,立即退出了《跨太平洋伙伴关系协定》(TPP),并威胁退出《北美自由贸易协定》(NAFTA),最终以《美墨加协定》(USMCA)取而代之。他认为这些多边贸易协定损害了美国的就业和产业。他还对中国、欧盟等主要贸易伙伴征收高额关税,发动贸易战。这种从自由贸易向保护主义的急剧转向,以及从规则导向的多边贸易体系向强权主导的双边交易的转变,严重扰乱了全球供应链,增加了国际贸易的不确定性。
    \item \textbf{退出国际协议:} 特朗普政府退出了《巴黎气候协定》和《伊朗核协议》,并一度威胁退出世界卫生组织(WHO)。这些举动表明,他将国际协议视为对美国主权的束缚,而非解决全球性问题的共同框架。这不仅损害了美国在国际社会中的领导力,也使得应对气候变化、核扩散和全球疫情等人类共同挑战的努力变得更加艰难。
    \item \textbf{对联合国和多边机构的蔑视:} 特朗普对联合国等国际机构持明显的蔑视态度,认为它们效率低下、官僚主义,且经常对美国指手画脚。他削减了美国对这些机构的资金支持,并经常无视其决议和呼吁。这种对多边主义的系统性攻击,削弱了全球治理的权威性和有效性。
\end{itemize}

在特朗普的眼中,盟友不再是共享价值观的伙伴,而是需要重新谈判的“交易对象”;国际规则不再是共同遵守的契约,而是可以随意撕毁的废纸。这种“交易式外交”和“实力至上”的逻辑,将国际关系从一个复杂的合作网络,简化为一场赤裸裸的利益博弈,使得全球秩序的稳定性和可预测性大大降低。

\subsection{案例聚焦(二):英国脱欧——一场自我放逐的“主权至上”实验}

如果说“美国优先”是全球霸主对既有秩序的“掀桌子”,那么英国脱欧(Brexit)则是一场中等强国为了“夺回控制权”而进行的自我放逐实验。它将民粹主义对主权的极端强调,推向了极致。

\begin{itemize}
    \item \textbf{与欧盟的“离婚”:} 英国曾是欧盟的重要成员,深度融入欧洲一体化进程。然而,脱欧派成功地将欧盟描绘成一个官僚、低效、侵蚀英国主权的“怪物”。“夺回控制权”的口号,精准地击中了民众对主权流失的焦虑。脱欧公投的结果,是英国与欧盟长达数年、充满争议和痛苦的“离婚”过程。
    \item \textbf{贸易与边境的困境:} 脱欧后,英国与欧盟的贸易关系变得复杂,新的海关检查和贸易壁垒给英国企业带来了巨大负担。北爱尔兰问题更是成为了一个无解的死结,因为它涉及到英国主权、欧盟单一市场完整性和北爱和平协议之间的复杂平衡。曾经无缝连接的边境,现在成为了政治和经济的痛点。
    \item \textbf{“全球英国”的愿景与现实:} 脱欧派承诺,脱离欧盟后,英国将能够自由地与世界各国签订贸易协定,成为一个“全球英国”。然而,现实远比想象中复杂。与欧盟这个巨大单一市场的脱钩,使得英国在国际贸易谈判中失去了重要的筹码。在国际舞台上,英国的影响力也因脱欧而受到削弱,其在欧洲的传统盟友关系也变得紧张。
    \item \textbf{对国际合作的负面示范:} 英国脱欧向世界传递了一个信号:即使是深度一体化的区域合作,也可能因为民粹主义的冲击而分崩离析。它鼓励了其他国家内部的疑欧派和民族主义势力,使得区域一体化和多边合作面临更大的挑战。
\end{itemize}

英国脱欧的案例表明,当“主权至上”被无限放大,并凌驾于经济利益和国际合作之上时,其代价可能是巨大的。它不仅损害了自身的经济利益,也削弱了国际合作的基石,使得曾经紧密的联盟关系变得疏远。

\section{失序的涟漪:全球性挑战下的无力}

“本国优先”的民粹主义浪潮,不仅仅是改变了国家间的双边关系,它更像一块块巨石,投入了全球性挑战的汪洋大海,激起了层层失序的涟漪。当各国都只顾自家门前雪时,那些需要全球协同才能解决的问题,便无人问津,甚至被利用来加剧国际紧张。

\subsubsection*{新冠疫情的“各自为战”:}

2020年初爆发的新冠疫情,是21世纪以来人类面临的最大公共卫生危机。它本应成为全球合作的典范,但我们看到的却是截然相反的景象。

\begin{itemize}
    \item \textbf{边境关闭与旅行限制:} 各国纷纷关闭边境,限制人员流动,这在短期内或许有助控制疫情,但也切断了全球供应链,阻碍了医疗物资和专业人员的流通。
    \item \textbf{“疫苗民族主义”:} 当疫苗研发成功后,富裕国家纷纷囤积疫苗,而贫穷国家却迟迟无法获得足够的供应。这种“疫苗民族主义”不仅加剧了全球不平等,也延长了疫情的持续时间,因为病毒在全球任何一个角落的传播,都可能产生新的变种,威胁到所有人。
    \item \textbf{甩锅与指责:} 一些民粹主义领导人将疫情的责任推卸给外国,煽动排外情绪,而非专注于科学应对和国际合作。这种政治化疫情的做法,使得全球协同抗疫的努力举步维艰。
\end{itemize}

如果说在没有民粹主义冲击的时代,全球在应对埃博拉、SARS等疫情时尚能形成一定程度的合作,那么在“本国优先”的当下,面对新冠这样规模的危机,国际社会却显得如此无力,令人扼腕。

\subsection{气候变化的“拖延症”:}

气候变化是人类面临的另一项生存威胁,它需要全球范围内的减排和能源转型。然而,民粹主义的兴起,使得应对气候变化的国际努力遭遇了严重挫折。

\begin{itemize}
    \item \textbf{退出国际承诺:} 特朗普政府退出《巴黎气候协定》,以及一些民粹主义政党对气候科学的否认,都削弱了全球减排的共识和动力。
    \item \textbf{优先短期经济利益:} 民粹主义者往往将环保政策描绘成“损害就业”、“阻碍发展”的负担,优先考虑短期内的化石燃料产业和经济增长,而非长期的环境可持续性。
    \item \textbf{“气候民族主义”:} 一些国家指责其他国家没有承担足够的责任,或者将气候行动视为对本国经济的限制,从而拒绝采取行动。这种“气候民族主义”使得全球共同应对这一挑战的窗口期正在迅速关闭。
\end{itemize}

\subsection{从多边主义到交易主义:}

民粹主义的崛起,使得国际关系从一个以规则为基础、以多边机构为平台的体系,转向了一个更加以权力为导向、以双边交易为主导的模式。

\begin{itemize}
    \item \textbf{国际规范的弱化:} 当大国随意退出国际协议,无视国际法庭的裁决时,国际规范和国际法的权威性就会受到侵蚀。这使得国际关系变得更加丛林化,强权政治的色彩日益浓厚。
    \item \textbf{联盟体系的松动:} 曾经坚不可摧的联盟,如北约,也面临着内部的裂痕和外部的压力。盟友之间信任的流失,使得集体行动的意愿和能力都大打折扣。
    \item \textbf{地缘政治风险的增加:} 当合作的意愿降低,竞争和对抗的意愿上升时,地缘政治的紧张局势就会加剧。地区冲突、军备竞赛和贸易摩擦的风险都会随之升高。世界变得更加不稳定,更加不可预测。
\end{itemize}

\section{旧秩序的裂缝与民粹主义的趁虚而入}

当然,我们必须承认,二战后建立的国际秩序并非完美无缺。它在设计之初就带有大国政治的烙印,其治理结构也存在民主赤字,未能充分反映发展中国家的声音。全球化带来的财富分配不均,也确实让一部分人感到被边缘化。这些都是旧秩序的“裂缝”,而民粹主义者正是趁虚而入,利用这些不满来攻击整个体系。

他们将国际合作描绘成“精英的阴谋”,将全球化带来的负面效应归咎于“外国势力”和“国际官僚”,从而为“本国优先”的政策主张披上“为民请命”的外衣。他们承诺,只要打破旧的国际束缚,国家就能重新获得自由,人民就能重新获得尊严。这种简单而有力的叙事,对于那些在复杂世界中感到迷茫和无助的民众来说,具有强大的吸引力。

然而,民粹主义所提供的“解决方案”,往往是拆解而非修复。它没有试图弥补旧秩序的缺陷,而是选择将其彻底摧毁。其结果,往往是“拆了东墙补西墙”,甚至“拆了东墙,西墙也塌了”。当国际合作的平台被削弱,当全球治理的机制被破坏,当各国都只顾自身利益而忽视共同挑战时,最终受损的将是所有国家,包括那些高喊“本国优先”的国家。

\section{结语:碎片化的未来与共同的命运}

民粹主义的浪潮,正在将我们带入一个“失序的世界”。这个世界不再有清晰的规则,不再有可靠的盟友,不再有共同应对危机的意愿。曾经被视为理所当然的国际合作,现在变得奢侈而脆弱。

从“美国优先”到英国脱欧,从对国际机构的蔑视到对全球挑战的“各自为战”,民粹主义的内向本质,正在系统性地解构二战后建立的国际秩序。它将世界从一个相互连接、相互依存的“地球村”,拉回到一个充满竞争、充满不信任的“部落世界”。

然而,全球性挑战,如气候变化、大流行病、核扩散和恐怖主义,并不会因为国家关闭边境、退出协议而消失。它们是无国界的,需要全球协同才能有效应对。当国际合作的平台被拆除,当各国都只顾眼前利益时,这些挑战将以更猛烈、更不可预测的方式反噬我们所有人。

这个“失序的世界”并非命中注定。它是一个选择的结果。民粹主义的剧本,为我们描绘了一个充满诱惑的“本国优先”的幻象,但其代价,是全球共同的和平与繁荣。理解民粹主义对国际秩序的深远影响,是我们认识当前世界动荡根源的关键一步。它迫使我们思考:在一个日益互联互通的时代,我们究竟是选择走向碎片化的未来,还是重新找回合作的智慧与勇气,共同塑造一个更加稳定、更加包容的世界?这,将是摆在我们面前最严峻的时代命题。

