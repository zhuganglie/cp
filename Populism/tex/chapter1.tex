\part{沃土——旧秩序为何崩裂}

\chapter{破碎的承诺:全球化的幽灵}

\textbf{本章论点:} 20世纪末的全球经济共识创造了巨大财富,却也遗留下了一个数量庞大、心怀怨恨的群体。

当历史的钟摆迈向20世纪末,世界似乎正站在一个崭新纪元的门槛。柏林墙的轰然倒塌,不仅象征着冷战的终结,更仿佛为一种全新的全球秩序——一个由资本、技术和信息自由流动所主导的“地球村”——铺平了道路。这便是全球化的黄金时代,一个充满玫瑰色承诺的时代。人们被告知,一个更加繁荣、更加开放、更加和平的世界指日可待。然而,二十余年倏忽而过,当我们回望这段镀金的岁月,却发现在炫目的光环之下,阴影早已悄然滋长。那曾经被许诺给每一个人的繁荣,似乎并没有均匀地洒向世界的每一个角落,反而在一部分人心中积聚起深深的怨恨。这怨恨,如同一个无形的幽灵,在全球化的盛宴散场后,开始在曾经的沃土上徘徊,并最终为民粹主义的崛起提供了最直接的养料。

\section{“历史终结论”下的玫瑰色图景:全球化的黄金承诺}

让我们将时钟拨回到上世纪90年代初。弗朗西斯·福山那本著名的《历史的终结与最后的人》似乎为整个时代定下了基调。在他看来,人类历史的意识形态之争已经结束,自由民主制与市场经济取得了最终的胜利。这种乐观情绪弥漫在西方世界的每一个角落,并迅速向全球扩散。

“全球化”成为了这个时代的关键词。其核心理念并不复杂:拆除国家间的贸易壁垒,让商品、服务、资本和信息在全球范围内自由流动;鼓励跨国公司将生产链条延伸至成本更低的地区;相信市场这只“看不见的手”能够最有效地配置资源,从而创造前所未有的财富。世界贸易组织的成立(1995年),北美自由贸易协定(NAFTA)的签署(1994年),以及中国在2001年加入世贸组织,都被视为这一历史进程中的里程碑事件。

当时的政治精英们,无论是美国的克林顿、英国的布莱尔,还是德国的施罗德,都热情地拥抱了这一趋势。他们所推行的“第三条道路”或类似的政策,试图在传统的左翼与右翼之间找到平衡,其经济政策的核心,便是对全球化市场力量的信任。跨国公司高歌猛进,以前所未有的规模在全球布局;新兴经济体,特别是亚洲国家,凭借低廉的劳动力成本和宽松的监管环境,迅速融入全球产业链,实现了经济的腾飞;发达国家的消费者则享受到了来自世界各地的廉价商品,从电子产品到服装鞋帽,选择空前丰富。

科技的进步,尤其是互联网的普及,更是为全球化插上了翅膀。信息以前所未有的速度跨越国界,世界似乎真的变小了。人们相信,这种互联互通不仅能带来经济效益,还能促进不同文化间的理解与融合,最终导向一个更加和平与理性的世界。这是一个充满希望的叙事:蛋糕会越做越大,每个人都能从中分得一杯羹,即使分配不均,增长的浪潮最终也会惠及所有人。这个承诺,如此诱人,如此不容置疑。

\section{看不见的代价:被遗忘的角落与无声的怨愤}

然而,在这幅宏大而光鲜的全球化图景之下,一些不和谐的音符早已开始浮现,只是在当时震耳欲聋的赞歌声中,它们显得微弱而被忽视。那些被全球化浪潮席卷、甚至吞噬的群体,他们的失落与怨愤,正在悄然积聚。

\paragraph*{锈迹斑斑的工业带:去工业化的浪潮与失落的社群}

全球化的一个直接后果,便是发达国家传统制造业的加速衰退,即所谓的“去工业化”。当资本可以自由地流向劳动力成本更低的地区,那些曾经以重工业为傲的城市和地区,便首当其冲地成为了牺牲品。美国的“铁锈地带”(Rust Belt),从宾夕法尼亚州的匹兹堡到俄亥俄州的克利夫兰,再到底特律和密尔沃基,曾经是美国工业的心脏,机器轰鸣,烟囱林立,为一代又一代蓝领工人提供了稳定的工作和体面的生活。然而,从上世纪七八十年代开始,随着工厂的接连倒闭和外迁,这里逐渐变得锈迹斑斑。高耸的厂房被废弃,机器蒙尘,成千上万的工人失去了赖以生存的工作。

这不仅仅是经济上的打击,更是一种身份的剥夺和社区的瓦解。对于那些世代在工厂工作的家庭而言,工厂不仅是谋生之所,更是他们身份认同的核心。工作的失去,意味着尊严的丧失,意味着对未来的迷茫。曾经充满活力的社区,随着年轻人的出走和商业的凋敝,逐渐失去了生机。毒品、犯罪、家庭破裂等社会问题随之而来。类似的景象,也出现在英格兰北部、法国洛林地区以及德国鲁尔区等老牌工业区。那里的人们感觉自己被时代抛弃了,被那些在光鲜亮丽的金融中心和高科技园区里指点江山的精英们遗忘了。他们的父辈曾为国家建设流血流汗,而他们却成为了全球化“成本效益”计算中的那个可以被轻易抹去的“成本”。

\paragraph*{机器的进击:自动化焦虑与技能的贬值}

与产业转移相伴而行的,是技术进步带来的自动化浪潮。机器人、人工智能、先进的生产管理系统,在提高生产效率的同时,也无情地取代了大量重复性的劳动岗位。这不仅仅局限于制造业,银行柜员、行政助理、甚至一些初级分析师的工作,都面临着被机器取代的风险。

对于拥有高技能、能够适应新技术需求的劳动者而言,这或许意味着新的机遇。但对于那些技能相对单一、教育程度不高的普通工人来说,自动化带来的更多是焦虑和恐慌。他们曾经熟练掌握的技艺,在一夜之间可能变得一文不值。学习新技能需要时间、金钱和机会,而这些对于一个中年失业的工人来说,往往是奢侈品。这种对“技术性失业”的恐惧,以及感觉自身价值不断贬低的无力感,进一步加剧了社会底层的焦虑情绪。他们看到的是一个未来,在那里,他们可能不再被需要。

\paragraph*{停滞的阶梯:中产的困境与日益扩大的鸿沟}

全球化承诺的“共同繁荣”并没有完全兑现。虽然全球范围内的绝对贫困有所减少,但在许多发达国家内部,贫富差距却在持续扩大。数据显示,自上世纪80年代以来,大部分发达国家最顶层1\%人群的收入和财富占比急剧上升,而中产阶级和底层民众的实际收入却长期停滞不前,甚至有所下降。

中产阶级,这个曾经被视为社会稳定基石的群体,感受到了前所未有的压力。他们发现,维持父辈那样的生活水平变得越来越困难。房价高企,教育和医疗费用不断上涨,而工资却原地踏步。曾经清晰可见的向上流动的阶梯,似乎变得模糊不清,甚至断裂。他们努力工作,遵守规则,却发现自己离“美国梦”或类似的国家梦想越来越远。

与此同时,那些在全球化浪潮中如鱼得水的金融精英、跨国公司高管和科技新贵们,却积累了惊人的财富。他们的生活方式与普通民众渐行渐远,仿佛生活在两个平行的世界。这种日益悬殊的贫富差距,以及随之而来的机会不均等,侵蚀着社会的公平感和凝聚力。“我们”与“他们”的鸿沟,不仅体现在财富上,更体现在认知和情感上。

\section{致命一击:2008,信任堤坝的崩塌}

如果说去工业化、自动化焦虑和中产阶级困境是全球化背景下缓慢积累的结构性矛盾,那么2008年的全球金融危机,则像一颗引爆了所有潜藏不满的炸弹,彻底击垮了普通民众对现有经济和政治体制的信任。

这场源自美国次贷危机的金融海啸,迅速席卷全球,将世界经济拖入自“大萧条”以来最严重的衰退。银行倒闭,股市暴跌,企业大规模裁员,无数普通家庭失去了毕生积蓄、住房和工作。然而,令人错愕和愤怒的是,那些制造了这场危机的金融机构高管们,大多没有受到应有的惩罚,反而是一些“大到不能倒”的银行,动用了纳税人的巨额资金进行纾困。

“华尔街的贪婪,纳税人买单”,这句口号精准地捕捉了当时弥漫在社会中的普遍情绪。人们看到,在危机面前,精英阶层似乎总有办法保护自己的利益,而普通民众则只能默默承受代价。政府的救援行动,在许多人看来,不是为了保护普通储户和投资者,而是为了维系那个制造了问题的腐败系统。这种强烈的被背叛感和不公正感,是前所未有的。

2008年的金融危机,成为了一个关键的转折点。它不仅暴露了全球金融体系的脆弱性和监管的缺失,更重要的是,它彻底动摇了自冷战结束以来建立的所谓“精英共识”的合法性。人们开始质疑:那些信誓旦旦承诺会带来普遍繁荣的经济学家、政治家和金融家们,究竟是无能,还是根本就不在乎普通人的死活?信任的堤坝一旦崩塌,便很难修复。怀疑和愤怒的种子,开始在更广泛的人群中生根发芽。

\section{“我们”与“他们”:精英的盛宴与民众的账单}

金融危机的冲击,叠加此前数十年积累的经济结构矛盾,共同催生了一种强烈的社会情绪:这个世界被一小撮“精英”所操纵,他们制定规则,攫取利益,而把代价和风险转嫁给“人民”。

这种“我们”(the people)与“他们”(the elite)的二元对立叙事,在全球化的“输家”和感觉被剥夺的群体中找到了广泛的共鸣。这里的“精英”,指向非常广泛:不仅包括华尔街的金融大鳄、跨国公司的CEO,也包括身居高位的政客、主流媒体的记者编辑、甚至大学里的专家学者。在许多普通人看来,这些人形成了一个封闭的、自利的“建制派”,他们说着普通人听不懂的行话,享受着普通人无法企及的特权,却对普通人的疾苦漠不关心。

全球化所创造的巨大财富,大部分流向了这些人。当工厂倒闭、工人失业时,他们却在讨论自由贸易的宏观效益;当自动化让一些人失去工作时,他们却在赞美技术进步的伟大;当中产阶级为生活苦苦挣扎时,他们却在达沃斯论坛上举杯庆祝全球经济的融合。这种认知上的巨大鸿沟,使得“精英”与“民众”之间的隔阂越来越深。

普通民众感觉自己的声音被淹没了,自己的诉求被忽视了。他们厌倦了主流政党那些陈词滥调式的承诺,厌倦了专家们那些脱离现实的分析。他们渴望有人能站出来,替他们说出心中的不满和愤怒,挑战那个看似坚不可摧的“体制”。这种普遍的观感——精英阶层攫取了大部分利益,而普通民众却在承担代价——为民粹主义领袖的登场铺平了道路。他们将以“人民的代言人”的姿态出现,承诺要将权力从“腐败的精英”手中夺回来,还给“真正的人民”。

\section{案例聚焦:大西洋两岸的经济怒火——特朗普与英国脱欧}

2016年,大西洋两岸接连发生的两件政治“黑天鹅”事件——英国公投决定脱离欧盟(Brexit)和唐纳德·特朗普当选美国总统——震惊了世界,也集中展现了积压已久的经济怨愤所能爆发出的巨大能量。尽管两者的具体背景和表现形式有所不同,但其背后都涌动着相似的经济叙事逻辑。

\paragraph*{“让美国再次伟大”:铁锈地带的呼喊与特朗普的崛起}

唐纳德·特朗普,一个毫无从政经验的房地产大亨和电视真人秀明星,却奇迹般地赢得了2016年美国总统大选。他的核心支持者,很大一部分便来自那些在全球化进程中备受打击的“铁锈地带”和传统制造业地区的白人蓝领阶层。这些人,曾经是美国中产阶级的中坚力量,拥有稳定的工作、体面的收入和强烈的社群认同感。然而,数十年来的工厂外迁和自动化浪潮,让他们失去了工作,也失去了往日的荣光。

特朗普敏锐地捕捉到了这个群体的失落与愤怒。他用最直白、甚至粗俗的语言,将矛头直指他所谓的“糟糕的贸易协定”(如北美自由贸易协定NAFTA)和“不公平的国际竞争”(特别是来自中国和墨西哥的竞争),声称这些因素“偷走”了美国人的工作岗位。他承诺要“让美国再次伟大”(Make America Great Again),要将制造业带回美国,要为“被遗忘的男人和女人们”夺回属于他们的未来。

他的竞选集会,往往更像是一场宣泄不满的狂欢。当他痛斥“华盛顿的沼泽”(corrupt Washington establishment)、“撒谎的媒体”(fake news media)和“全球主义精英”(globalist elites)时,台下爆发出阵阵欢呼。对于那些感觉自己被主流政治彻底抛弃的人来说,特朗普的出现,仿佛一道刺破黑暗的光。他不是一个传统的政客,他口无遮拦,他挑战政治正确,他看起来就像一个局外人,一个能够真正理解他们痛苦并为他们而战的“强人”。经济上的怨恨,成为了特朗普构建其民粹主义叙事的核心基石。他成功地将一群在经济上感到绝望和被背叛的选民动员起来,颠覆了传统的政治格局。

\paragraph*{“夺回控制权”:英国脱欧背后的经济账本与民怨}

几乎在特朗普崛起的同一时期,大洋彼岸的英国也经历了一场政治地震。在2016年6月的公投中,51.9\%的英国选民投票支持脱离欧盟。这场公投的结果,同样与深层次的经济不满密切相关。

“脱欧派”的核心论点之一,便是英国每年向欧盟缴纳巨额的成员国费用,却没有得到相应的回报。那个印在红色巴士上极具煽动性的口号——“我们每周给欧盟3.5亿英镑,让我们把钱花在国民医疗服务体系(NHS)上吧”——虽然其准确性备受争议,却成功地触动了许多英国民众的神经。他们认为,这些本应用于改善本国公共服务和基础设施的资金,却被浪费在了布鲁塞尔那些遥远而官僚的欧盟机构身上。

此外,欧盟的人员自由流动原则,也成为经济焦虑的焦点。一些人认为,来自东欧等欧盟成员国的移民,拉低了本国低技能岗位的工资水平,并给住房、医疗、教育等公共服务带来了巨大压力。尽管经济学家们对移民的整体经济影响看法不一,但在那些感受到竞争压力或公共服务紧张的地区和人群中,这种担忧是真实存在的。

“夺回控制权”(Take Back Control)是脱欧派最具号召力的口号。这不仅仅是指夺回政治主权,也包含了夺回对本国经济政策、贸易规则和边界的控制权。许多支持脱欧的选民,特别是来自英格兰北部和中部那些经历过去工业化阵痛的地区的人们,感觉欧盟的规则束缚了英国的发展,而脱离欧盟则可以让他们摆脱这些束缚,重振本国经济,优先照顾本国公民的利益。与特朗普的支持者类似,他们也对伦敦的金融精英和亲欧盟的建制派充满了不信任,认为这些人从欧盟成员国身份中获益,却忽视了普通民众的困境。

\paragraph*{共同的旋律:对失落的经济主权和尊严的追寻}

比较特朗普的胜选和英国脱欧,我们可以清晰地看到一条共同的主线:两者都成功地利用了民众在全球化背景下积累的经济焦虑和对精英阶层的不满。无论是“让美国再次伟大”还是“夺回控制权”,其核心都是对一种失落的经济主权和国民尊严的追寻。

在这两种叙事中,都存在着一个“受害者”群体——那些在全球化浪潮中被边缘化、被剥夺、被遗忘的普通民众;也都存在着一个“加害者”——腐败自私的国内精英与不公平的外部力量(如其他国家、国际组织)。民粹主义领袖或运动,则将自己定位为这些“受害者”的拯救者,承诺要打破旧秩序,重建一个更公平、更符合“人民”利益的新秩序。这种基于经济怨愤的动员,具有强大的情感穿透力,因为它触及了人们对美好生活的向往以及对不公正待遇的本能反抗。

\section{结语:怨恨的沃土与民粹的幽灵}

20世纪末开启的全球化浪潮,无疑创造了巨大的物质财富,也深刻地改变了世界的面貌。然而,正如本章所揭示的,这场盛宴并非人人均沾。在繁荣的表象之下,去工业化的阵痛、自动化带来的焦虑、中产阶级的困境以及日益扩大的贫富鸿沟,共同织就了一张巨大的怨恨之网。2008年的金融危机,则如同最后一根稻草,彻底压垮了许多人对现有体制的信任。

这片由“破碎的承诺”和普遍的经济不安全感所浇灌的土地,变得异常“肥沃”。它不再相信传统精英的陈词滥调,不再对主流媒体的报道照单全收,不再对既有的政治经济秩序抱有幻想。它渴望改变,渴望发出自己的声音,渴望寻找到能够代表自身利益的强大力量。

正是这片怨恨的沃土,为民粹主义的幽灵提供了绝佳的栖身之所。那些承诺要倾听“被遗忘的人民”的声音、挑战“腐败的精英”、重建“伟大国家”的民粹主义领袖们,如同嗅到血腥味的鲨鱼,纷纷登场。他们或许来自左翼,或许来自右翼,其具体的政策主张可能千差万别,但他们都精准地把握住了弥漫在社会底层的这种经济怨愤,并将其转化为强大的政治动能。

然而,经济因素只是故事的一部分。正如我们将在下一章中看到的,民粹主义的崛起,同样深深植根于一种更为复杂和微妙的文化焦虑——对身份失落和生活方式受到威胁的恐惧。只有将经济的怨恨与文化的焦虑结合起来审视,我们才能更全面地理解,为何在21世纪的今天,民粹主义的浪潮会如此汹涌澎湃。全球化的幽灵,不仅在经济层面投下阴影,更在深层搅动着人们的身份认同与文化归属。