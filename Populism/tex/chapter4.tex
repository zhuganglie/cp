\part{剧本——全球民粹主义操盘手}
\chapter{右翼民族主义者:特朗普与英国脱欧派}

如果说本书的第一部分描绘了一片因经济不平、文化焦虑和政治失信而变得异常“肥沃”的土地,那么从本章开始,我们将聚焦于那些在这片土地上精准播种、并收获了惊人权力的“操盘手”。他们风格各异,背景不同,但他们手中似乎都握着一本相似的“剧本”。\textbf{这本剧本的核心,便是将弥漫在社会中的不满情绪,转化为一股排他性的、以民族为名的政治力量。}

2016年,是这本剧本被演绎得淋漓尽致的一年。大西洋两岸,两场被主流精英视为“不可能发生”的政治地震,相隔数月,接连引爆。6月23日,古老的英伦三岛用一场全民公投,选择挣脱它已融入四十余年的欧洲一体化进程;11月8日,美利坚合众国的选民将一位毫无从政经验、言辞出格的房地产大亨兼电视真人秀明星——唐纳德·特朗普——送入了白宫。

“英国脱欧”(Brexit)与特朗普的胜选,如同两面棱镜,折射出右翼民族主义民粹主义在21世纪的强大威力。它们并非孤立的意外,而是同一股时代暗流在不同政治土壤上的两次标志性喷发。前者是一场由理念驱动、多头并进的“人民起义”;后者则是一场由个人魅力主导、颠覆政党的“个人秀”。尽管表现形式迥异,但剖开其喧嚣的表象,我们会发现,它们遵循着一套惊人相似的运作逻辑,一套足以颠覆传统政治游戏规则的“剧本”。本章,我们将深入解剖这套剧本,看看这两场运动的操盘手们,是如何绕过旧的权力守门人,如何用全新的语言重塑政治辩论,并最终将“我们vs他们”的叙事,烙印在数千万选民的心中。

\section{剧本第一幕:绕过守门人——媒体的武器化}

传统政治中,信息流动的路径相对清晰:政治人物通过主流媒体(报纸、电视台、广播)这一“守门人”,将信息传递给公众。媒体通过采访、编辑和评论,对信息进行过滤和阐释。然而,特朗普和英国脱欧派深谙,要挑战建制派,首先必须打破其对信息渠道的垄断。他们不只是利用媒体,更是将媒体本身“武器化”,开辟了直达民众的全新战场。

\subsection{特朗普的“组合拳”:真人秀的狂欢与推特的战争}

唐纳德·特朗普本身就是媒体时代的产物。他在成为总统候选人之前,最广为人知的身份是真人秀《学徒》中那位说出“你被解雇了!”(You're fired!)的霸道总裁。他深知,在注意力稀缺的时代,政治早已不仅仅是政策辩论,更是一场争夺眼球的娱乐秀。他将这一认知发挥到了极致。

首先,他将竞选集会(rally)变成了一场场精心编排的政治奇观。传统的政治集会往往是程序化的、略显沉闷的。而特朗普的集会,则更像是一场摇滚音乐会与摔角比赛的结合体。震耳欲聋的音乐、巨大的屏幕、挥舞的旗帜,以及最重要的——特朗普本人长达一两个小时、几乎没有讲稿的即兴“独白”。他在这里不是在宣讲政策,而是在进行一场集体的情感动员。他用最简单、最口语化的词汇,讲述着一个“我们被背叛了”的故事。他与台下的支持者频繁互动,形成一种强大的“呼喊-回应”模式。当他痛斥“不诚实的媒体”时,人群会齐声高喊“CNN烂透了!”;当他提到希拉里·克林顿时,人群则爆发出“把她关起来!”(Lock her up!)的怒吼。

这些集会不仅是为现场观众准备的,更是为电视直播量身定做的。电视台,尤其是24小时新闻频道,无法抗拒这种充满戏剧性和冲突性的“免费内容”。特朗普深知,争议就是流量。他越是口无遮拦,越是发表惊人言论,媒体就越是无法将镜头从他身上移开。据统计,在2016年大选期间,特朗普获得了价值数十亿美元的免费媒体曝光,远远超过任何一位竞争对手。他成功地让媒体成为了他宣传的扩音器,即使那些报道大多是批判性的。对于他的支持者而言,媒体的批判恰恰证明了他是那个敢于挑战“腐败体制”的局外人,反而增强了他的可信度。

如果说大型集会是特朗普的“常规武器”,那么推特(Twitter)就是他的“精确制导导弹”。这个社交媒体平台,成为了他绕过所有传统媒体“守门人”的私人频道。每天清晨、深夜,他随时随地发布简短、有力、充满个人情绪的推文。他用它来攻击对手(“爱撒谎的泰德”、“不诚实的希拉里”),用它来反击负面新闻(“假新闻!”),用它来设定当天的政治议程,更用它来与他的支持者进行直接的、看似亲密的交流。

特朗普的推文风格是颠覆性的。它充满了拼写错误、语法问题和全大写字母的咆哮,这与传统政治家字斟句酌的公关文稿形成了鲜明对比。然而,这种“不完美”恰恰塑造了一种“真实感”。他的支持者觉得,这是特朗普本人在说话,未经任何包装,发自肺腑。当他发出那条著名的、不知所云的“covfefe”推文时,主流社会一片嘲讽,但他的粉丝却乐在其中,认为这正是他不受条条框框束缚的魅力所在。通过推特,特朗普建立了一个庞大的、忠诚的数字部落。他不再需要《纽约时报》或CNN的认可,他拥有自己的报纸和电视台,24小时不间断地向他的“人民”广播。

\subsection{英国脱欧派的“立体战”:红色巴士的视觉轰炸与小报的舆论合奏}

与特朗普个人英雄主义式的媒体战不同,英国脱欧运动的媒体策略更像一场多兵种协同的“立体战”。他们没有一个像特朗普那样独一无二的媒体焦点,但他们通过多种渠道,成功地将一个核心信息植入了公众的脑海。

这场战役中最具标志性的武器,无疑是那辆红色的双层巴士。车身上用巨大的白色字体印着一句极具煽动性的话:“我们每周给欧盟3.5亿英镑,让我们把钱花在我们的国民医疗服务体系(NHS)上吧。”(We send the EU £350 million a week, let's fund our NHS instead.)这辆巴士巡游在英国的大街小巷,成为了脱欧派最强大的视觉符号。

这个口号是宣传史上的一个杰作。首先,它将一个复杂的问题(英国的欧盟预算贡献)简化为一个具体、惊人且易于记忆的数字。其次,它将“付出”(给欧盟的钱)与一个深受英国民众珍视的“回报”(国民医疗服务体系NHS)直接对立起来,制造了一种“我们的钱被外国人拿去,却没有用在刀刃上”的强烈不公感。尽管这个数字的准确性很快就遭到了经济学家和事实核查机构的广泛质疑和驳斥,但这已经无关紧要。在情感面前,事实显得苍白无力。当留欧派试图用复杂的经济数据来辩解时,他们已经输掉了这场宣传战。红色巴士的口号,已经像病毒一样传播开来。

如果说红色巴士是地面部队,那么英国强大的右翼小报(tabloids)就是他们的“重炮集群”。像《太阳报》、《每日邮报》和《每日快报》这样的报纸,几十年来一直在向它们的数百万读者灌输一种疑欧、反移民的情绪。它们用耸人听闻的标题,将欧盟描绘成一个由面目模糊的官僚统治、不断侵蚀英国主权、并用各种荒唐规定(比如禁止卖弯的香蕉)来束缚英国人民的“怪物”。

在公投期间,这些小报火力全开。它们连篇累牍地报道关于移民犯罪、福利被滥用以及欧盟官僚主义的故事,将留欧派描绘成不爱国、脱离群众的“精英”。在公投日当天,《太阳报》的头版标题是“独立日”(Independence Day),敦促读者“挣脱布鲁塞尔的枷锁”。这种长年累月的舆论铺垫,为脱欧运动创造了极其有利的民意基础。他们不需要从零开始说服民众,只需要点燃早已埋下的干柴。

此外,脱欧派也悄悄地运用了更先进的数字武器。像“投票脱欧”(Vote Leave)这样的官方竞选团体,与数据分析公司(如后来声名狼藉的剑桥分析公司)合作,通过社交媒体(主要是Facebook)向特定的摇摆选民群体,推送量身定制的广告信息。这些信息往往聚焦于移民问题或主权问题,以激发目标受众的恐惧和焦虑。这是一场看不见的、在数字空间里进行的精准动员。

无论是特朗普的“推特治国”,还是脱欧派的“巴士+小报”组合,其核心都是一致的:绕过、甚至摧毁传统媒体的“守门人”角色,建立一条直达民众的情感和信息通道。他们明白,在新的媒体环境下,谁能定义议题、谁能激发情感,谁就能赢得战争。

\section{剧本第二幕:语言的魔力——“我们”与“他们”的创世神话}

如果说媒体策略是“术”,那么叙事构建就是“道”。右翼民族主义民粹主义最核心的技巧,在于将复杂的社会现实,简化为一个清晰、有力、充满道德色彩的二元对立故事:一边是“纯洁的、善良的、真正的我们”,另一边是“腐败的、邪恶的、虚伪的他们”。

\subsection{塑造“我们”:为“被遗忘的人民”加冕}

在这套剧本里,“人民”不是一个包含了所有公民的宽泛概念,而是一个经过精心筛选的、具有特定身份的群体。

特朗普口中的“人民”,是“被遗忘的男人和女人们”(the forgotten men and women)。他们是谁?他们是生活在“铁锈地带”的失业工人,是阿巴拉契亚山区的煤矿工人,是中西部的农民,是那些感觉被全球化浪潮抛弃、被沿海地区的文化精英鄙视的白人蓝领阶层。他们曾是“美国梦”的基石,是国家的脊梁,但现在,他们感觉自己成了自己国家里的陌生人。

特朗普的语言,赋予了这个群体前所未有的尊严和力量。他告诉他们:“你们的痛苦是真实的,你们的愤怒是正当的。你们不是失败者,而是受害者。那些在华盛顿、在华尔街、在好莱坞的精英们背叛了你们。而我,是你们的声音。”他的竞选口号“让美国再次伟大”(Make America Great Again),精准地击中了这个群体对一个想象中的、更美好、更强大的过去的怀旧之情。戴上那顶红色的MAGA帽子,就意味着加入了这个“真正美国人”的部落,意味着从一个“被遗忘者”变成了一名“爱国者”。

同样,英国脱欧运动所诉诸的“人民”,也是一个特定的群体。他们是“沉默的大多数”(the silent majority),是那些相信“英国性”(Britishness)、珍视传统、遵守规则的普通人。他们感觉自己的生活方式正受到外来文化和布鲁塞尔法令的侵蚀。脱欧派领袖奈杰尔·法拉奇(Nigel Farage)最喜欢将自己描绘成一个在酒吧里喝着啤酒、为小人物(the little guy)说话的普通人,以此来对抗那些在伦敦金融城里喝着香槟、大谈全球化的“精英”。

“夺回控制权”(Take Back Control)这个口号,与“让美国再次伟大”异曲同工。它同样诉诸一种失落感和对主权的渴望。它向“沉默的大多数”承诺,脱离欧盟将使他们能够重新掌控自己的边界、法律和命运。投票“脱欧”,因此被描绘成一场普通人反抗遥远而傲慢的精英统治的“解放运动”。

\subsection{定义“他们”:一个不断扩大的敌人名单}

一个强大的“我们”,需要一个或多个清晰的“他们”作为对立面。右翼民粹主义的剧本,提供了一份详尽的、可以随时增补的“敌人名单”。

\begin{enumerate}
    \item \textbf{腐败的国内精英(The Corrupt Elite):} 这是所有民粹主义故事的头号反派。在特朗普的叙事里,他们是“华盛顿的沼泽”(the Washington swamp),一个由政客、说客、官僚组成的自利集团。在脱欧派的口中,他们是“威斯敏斯特的泡沫”(the Westminster bubble)和“布鲁塞尔的官僚”,一群脱离群众、出卖国家利益的“全球主义者”。将国内的政治对手描绘成与人民为敌的腐败分子,可以将政策分歧道德化,将其变成一场正邪之战。
    \item \textbf{撒谎的主流媒体(The Lying Media):} 这是精英的喉舌和帮凶。特朗普将任何对他不利的报道都贴上“假新闻”(fake news)的标签,系统性地摧毁《纽约时报》、CNN等主流媒体的公信力。这是一种高明的“政治免疫”策略:通过预先将信源污名化,他可以确保即使有确凿的负面证据出现,他的支持者也会因为不信任信息来源而选择无视。同样,脱欧派也将所有警告脱欧经济风险的预测,斥之为“恐惧计划”(Project Fear),是建制派用来吓唬人民的阴谋。
    \item \textbf{危险的外国他者(The Dangerous Other):} 这是右翼民族主义剧本中最具煽动性和危险性的部分。它将经济和文化上的焦虑,具体化为对特定外来群体的恐惧和敌意。
    \begin{itemize}
        \item 对特朗普而言,这个“他者”主要是来自墨西哥的非法移民和穆斯林。他竞选时那段臭名昭著的言论——“他们(墨西哥)带来的不是最好的人……他们带来了毒品,他们带来了犯罪,他们是强奸犯”——以及他要求“全面禁止穆斯林进入美国”的主张,都赤裸裸地诉诸于种族和宗教偏见。那句响彻云霄的“建墙!建墙!”(Build the wall!)的口号,不仅是一个政策提议,更是一个强大的象征,象征着将“我们”纯洁的内部与“他们”危险的外部彻底隔绝。
        \item 对于英国脱欧运动,这个“他者”主要是来自欧盟东欧成员国的移民。尽管官方的“投票脱欧”运动在言辞上相对谨慎,但其盟友,特别是法拉奇领导的英国独立党(UKIP),则毫不避讳地将移民问题作为核心议题。公投前夕,法拉奇公布了一张极具争议的海报,画面上是一长队看似来自中东的难民,配文是“引爆点”(Breaking Point)。这张海报被广泛批评为煽动仇恨,但它精准地利用了部分民众对移民失控和社会不堪重负的恐惧。
    \end{itemize}
\end{enumerate}

通过这套“我们vs他们”的叙事,特朗普和脱欧派成功地将复杂的社会问题(如经济衰退、身份认同危机)简化为一个个易于理解、能够激发强烈情感的“抓手”。你的工作丢了?不是因为自动化和产业结构调整,而是因为墨西哥人或中国人“偷走”了它。你的社区变了?不是因为社会自然变迁,而是因为不受控制的移民和“多元文化主义”的阴谋。你的声音没人听?不是因为代议制民主的固有缺陷,而是因为“沼泽”里的精英们在密谋反对你。这个剧本,为那些感到失落和愤怒的人们,提供了一个简单明了的解释,一个宣泄情绪的出口,以及最重要的——一个需要为之战斗的敌人。

\section{剧本第三幕:比较视角——独角戏与大合唱}

尽管共享着相似的剧本核心,但特朗普的崛起和英国脱欧在“表演”风格和组织形态上,却呈现出鲜明的对比。理解这些差异,有助于我们把握右翼民粹主义的多样面貌。

\subsection{个人崇拜的“独角戏” vs. 理念驱动的“大合唱”}

特朗普的运动,从始至终都是一场围绕他个人的“独角戏”。核心吸引力在于特朗普本人——他的名人光环、他的“交易大师”形象、他敢于挑战一切的“斗士”姿态。支持者们追随的不是一套严谨的意识形态,而是特朗普这个人。他们相信他,是因为他“有钱,不需要听任何人的”,是因为他“说出了我们不敢说的话”。对他的忠诚,是一种高度情绪化、个人化的联结。MAGA运动如果失去了特朗普,就如同失去了灵魂。

相比之下,英国脱欧更像一场由共同理念驱动的“大合唱”。虽然它也有像鲍里斯·约翰逊(Boris Johnson)、迈克尔·戈夫(Michael Gove)和奈杰尔·法拉奇这样的领军人物,但没有哪一个人能够像特朗普那样,成为运动的唯一化身。人们投票脱欧,可能因为认同约翰逊的乐观主义,也可能因为被法拉奇的激进所吸引,但归根结底,他们投票支持的是“脱欧”这个理念本身——为了主权,为了控制移民,为了摆脱布鲁塞尔。这是一场围绕“事业”(the cause)而非“领袖”(the leader)的运动。即使没有这些领军人物,脱欧的理念在英国社会中也早已根深蒂固。

\subsection{体制外的“颠覆者” vs. 体制内的“叛乱者”}

特朗普是一个彻头彻尾的体制外“颠覆者”。他从未担任过任何公职,他以一个商人的身份,向整个华盛顿政治建制(包括他所竞选的共和党建制派)宣战。他像一头闯入瓷器店的公牛,打碎了共和党内原有的派系平衡和游戏规则,最终通过敌意收购(hostile takeover)的方式,劫持了整个政党。他的胜利,是美国政治中一次罕见的、由外向内的颠覆。

英国脱欧的故事则更为复杂,它更像一场体制内的“叛乱”。脱欧运动的核心领导层——约翰逊和戈夫——本身就是执政的保守党内部的高级成员,是体制的一部分。他们的行动,是一场针对本党领袖(时任首相戴维·卡梅伦)和主流派系的“政变”。他们巧妙地利用了党内长期存在的疑欧派势力,并与体制外的民粹力量(法拉奇的独立党)结成了一个“权宜同盟”。因此,脱欧的胜利,可以看作是执政党内部的一股强大势力,借助民粹的浪潮,成功地推翻了原有的领导核心和国家大政方针。

\subsection{殊途同归:共享的民族主义-民粹主义内核}

然而,无论是一人主导的独角戏,还是多声部合唱的交响;无论是体制外的颠覆,还是体制内的叛乱,这两场运动的终点是相同的。它们都成功地将一种右翼的、民族主义的民粹主义逻辑,注入了国家政治的核心。

“让美国再次伟大”与“夺回控制权”,本质上是同一个故事的不同版本。它们都承诺要逆转全球化的潮流,让国家回归到一个想象中的、拥有完全主权和文化同质性的“黄金时代”。
“建墙”与“控制边境”,都是将国家安全与排斥外来者直接挂钩的民族主义解决方案。
攻击“华盛顿沼泽”与攻击“布鲁塞尔官僚”,都是将民众的不满从具体的政策问题,转移到对整个精英治理体系的道德审判上。
最终,特朗普和英国脱欧派都成功地将2016年的选举或公投,定义为一场“人民”对抗“精英”的终极对决。在这场对决中,他们是“人民”唯一的代言人,而投票给他们,则是选民能够“夺回”自己国家的唯一方式。

\section{结语:剧本的胜利与民主的警钟}

解剖特朗普与英国脱欧派的“剧本”,我们看到了一套精心设计、威力巨大的政治运作体系。它通过武器化的媒体策略,打破了旧的信息垄断;通过创造“我们vs他们”的强大叙事,将复杂的社会矛盾转化为简单的道德斗争;并通过灵活多样的表演形式,适应了不同的政治环境。

2016年的结果证明,这套剧本是成功的。它成功地动员了那些在全球化中失落、在文化变迁中焦虑、在政治中感到被漠视的庞大群体,将他们的怨愤、怀旧与希望,锻造成了一股足以颠覆现有政治秩序的强大力量。他们不是历史的偶然,而是我们这个时代结构性矛盾的必然产物。他们为无数心怀不满的民众,提供了一个看似能够解决所有问题的“快捷方式”:将权力交给一个强有力的领袖,或者通过一次公投“一劳永逸”地解决问题,然后将国家的大门向世界关上一部分。

然而,当大幕落下,当竞选的狂热逐渐褪去,这套剧本的深远影响才刚刚开始显现。它所依赖的社会分裂、对事实的攻击以及对外部世界的敌意,并不会随着选举的结束而消失。恰恰相反,它们已经深深地嵌入了政治肌体之中。

特朗普的四年任期和英国脱欧后的漫长挣扎,都将是这套剧本的延续。但故事并未就此结束。在世界的其他角落,一些更为老练的“操盘手”正在观察和学习。他们看到,赢得选举只是第一步。如何将民粹主义的胜利,转化为对国家机器的长期掌控,如何将一时的群众狂热,固化为一种不可逆转的权力结构,将是他们要演绎的下一幕。匈牙利的欧尔班·维克托,便是这方面的一位“大师”。在下一章,我们将前往多瑙河畔,看看这位“有耐心的民粹主义者”,是如何上演一出更为系统、更为深刻的“夺权”大戏。