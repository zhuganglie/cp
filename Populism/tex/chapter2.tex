\chapter{失落的部落:身份与怀旧的政治}

\textbf{本章论点:} 除了经济因素,民粹主义更深层地根植于一种文化焦虑——对丧失民族身份和生活方式的恐惧。

如果说上一章我们探讨的经济怨愤是点燃民粹主义烈火的干柴,那么本章将要深入剖析的文化焦虑,则是那片浸透了燃油、一触即燃的土地。当一个人的钱包变瘪时,他会感到愤怒;但当他感觉自己脚下的土地正在变得陌生,自己所珍视的传统正在消逝,甚至自己在家园中都仿佛成了一个“外人”时,他所感受到的,将是一种更为深刻、更具存在性的恐惧。这种恐惧,关乎“我们是谁”,关乎我们归属何方,关乎我们代代相传的生活方式能否延续。民粹主义的操盘手们精准地捕捉到了这种深层的文化焦虑,并将其锻造成了比经济诉求更具凝聚力和动员力的强大武器。

在许多西方社会,尤其是在那些曾经拥有稳固主体民族和文化认同的国家里,一种“失落的部落”情结正在蔓延。这个“部落”的成员,往往是那些在人口结构、社会规范和文化景观的剧变中,感到被边缘化、被冒犯、被剥夺了文化主导权的群体。他们未必是经济上最贫困的人,但他们一定是文化上最失落的人。他们怀念一个想象中的“黄金时代”,并恐惧一个正在到来的、他们无法认同的未来。这种对身份的坚守和对过去的怀旧,共同谱写了一曲哀婉而愤怒的政治悲歌,而民粹主义领袖,则自告奋勇地成为了这曲悲歌的指挥。

\section{脚下的大地在移动:人口变迁与文化恐慌}

想象一下,你生活在一个你祖辈世代居住的小镇。这里的教堂钟声、街角面包店的香气、邻里间的方言问候,构成了你对“家”的全部记忆。然而,在短短一二十年间,你发现周围的一切都在悄然改变。新的面孔越来越多,他们说着你听不懂的语言,信奉着不同的神,遵循着不同的生活习俗。老街区的店铺换上了陌生的招牌,空气中飘散着异域的香料味,学校里孩子的构成也发生了根本性的变化。

这并非危言耸听,而是过去半个世纪以来许多欧美国家社区的真实写照。战后持续的移民潮,尤其是世纪之交后来自中东、北非和南亚等文化差异更显著地区的移民,深刻地改变了西方社会的人口版图。以法国为例,据估计其人口中拥有穆斯林背景的比例已接近10\%;在德国,2015年难民危机一年之内就涌入了超过一百万寻求庇护者;在美国,拉美裔人口的快速增长和白人人口比例的相对下降,已是不可逆转的趋势。

对于信奉多元文化主义的城市精英和知识分子而言,这或许是社会充满活力和包容性的体现。然而,对于那些身处变化第一线的本地居民,尤其是教育程度不高、流动性不强的中下层民众来说,这种变化带来的感受远比“多元”一词复杂得多。它首先带来的是一种物理空间上的“被侵占感”。他们感觉自己的社区、自己的城市,正在变得不再“属于自己”。这种感受与经济上的竞争(如担心移民抢走工作、拉低工资)交织在一起,但其核心是一种更深层次的文化不安。

这种不安,在法国极右翼作家雷诺·加缪(Renaud Camus)提出的“大置换”(Le Grand Remplacement)阴谋论中得到了最极端的表达。该理论声称,欧洲的白人基督徒人口,正在被一个由“全球主义精英”精心策划的计划所系统性地“置换”,取而代之的是非欧洲的、主要是穆斯林的移民。尽管这一理论被主流社会斥为荒谬和种族主义,但它之所以能在特定人群中产生市场,正是因为它用一种夸张、偏执的方式,说出了许多人内心深处那种模糊的恐惧:我们正在被取代,我们的文明正在消亡。

当民粹主义者谈论“失控的边境”时,他们所触动的,并不仅仅是民众对国家安全的担忧,更是这种对人口结构变化所引发的文化恐慌。他们将移民描绘成一种对本土文化纯洁性和生活方式的威胁。在他们的叙事中,移民不仅是经济上的负担,更是文化的“他者”,是潜在的社会秩序破坏者。这种论调在那些公共服务(如学校、医院)因人口快速增长而备受压力的地区,尤其具有说服力。人们将生活质量的下降,直观地归咎于“新来的人”,而忽略了背后更复杂的政府投资不足、政策失当等结构性问题。

与人口变迁相伴的,是社会规范的剧烈摇摆。在过去几十年里,西方社会经历了一场深刻的价值观革命:女权运动的兴起、LGBTQ+权益的合法化、种族平权运动的深入,以及所谓“政治正确”话语的普及。这些进步无疑推动了社会的整体文明进程,但也让一部分固守传统价值观的人感到无所适从,甚至感觉自己遭到了冒犯和围攻。

一个信奉传统家庭观念的父亲,可能会对同性婚姻的合法化感到困惑和不安;一个习惯了在男性主导环境中工作的工人,可能会对工作场所日益强调的性别平等和反骚扰规范感到束手束脚;一个在成长过程中从未被要求审视自身“白人特权”的普通人,可能会对批判性种族理论感到愤怒,认为这是在否定他个人的努力和国家的历史。

在他们看来,这个世界变得越来越“陌生”,充满了各种需要小心翼翼遵守的规则和不容置疑的新信条。他们所熟悉的那个“正常”世界——男人养家糊口,女人相夫教子,国家以基督教文化为荣——正在被一个由“自由派精英”和“社会正义战士”所主导的新秩序所取代。他们感觉自己的价值观、自己的信仰,甚至自己的语言,都被贴上了“落后”、“偏执”甚至“歧视”的标签。他们成了自己社会中的“道德少数派”,被要求为自己未经选择的身份(白人、男性、异性恋)而感到羞愧。这种被剥夺了话语权和道德制高点的感觉,是一种巨大的羞辱,也催生了强烈的逆反心理。民粹主义者正是利用了这种逆反心理,他们公开嘲笑“政治正确”,为那些感觉被压抑的传统价值观“正名”,让那些“失落的部落”成员们感到,终于有人敢于说出他们的“心里话”了。

\section{昨日重现:怀旧的魔力与想象的共同体}

面对一个令人困惑和不安的现在,人们很自然地会向过去寻求慰藉。怀旧,这种对逝去时光的甜美伤感,便成为了一种强大的社会情绪和政治动员工具。民粹主义者深谙此道,他们是兜售怀旧情绪的大师,承诺带领人民“回到过去”——一个被他们精心美化和浪漫化的“黄金时代”。

唐纳德·特朗普的竞选口号“让美国再次伟大”(Make America Great Again),堪称怀旧政治的典范。这个口号从未明确定义美国“何时”伟大以及“如何”伟大,但这正是其高明之处。它为每个心怀不满的选民都提供了一个可以自由填充的想象空间。对于“铁锈地带”的失业工人来说,“伟大”可能意味着1950年代工厂轰鸣、工会强大、一人工作足以养活全家的时代;对于福音派基督徒来说,“伟大”可能意味着基督教价值观在社会生活中占据核心地位、学校里可以公开祈祷的时代;对于许多白人来说,“伟大”则可能意味着一个种族等级分明、白人文化占据绝对主导地位、社会秩序井然的时代。

这个被想象出来的“伟大美国”,是一个高度滤镜化的版本。它刻意忽略了那个时代的种族隔离、性别歧视、冷战恐惧以及社会内部的种种不公。但对于那些在当今世界中感到失落的人来说,这些历史的阴暗面并不重要。重要的是,那个“过去”代表了一种确定性、一种凝聚力和一种国家自豪感,而这些,正是他们感觉当下所缺失的。

同样,英国脱欧派的口号“夺回控制权”(Take Back Control),也充满了浓厚的怀旧色彩。它唤起了一种对“日不落帝国”荣光的遥远记忆,以及对一个能够独立自主、不受布鲁塞尔“官僚”掣肘的强大英国的向往。脱欧宣传中反复出现的二战意象——敦刻尔克精神、不列颠之战——都在暗示,英国正在面临一场新的生存之战,需要再次展现其坚韧不拔的民族性格,从“欧洲大陆的束缚”中解放出来。

这种怀旧叙事之所以有效,是因为它为复杂的现实问题提供了一个简单的情感解决方案。你不必去理解全球供应链的复杂性,不必去分析自动化对劳动力的影响,你只需要相信,我们所有的问题,都源于我们偏离了那个“伟大的过去”。只要我们能“回去”,一切都会好起来。民粹主义领袖将自己塑造成那个能够带领人民穿越时空、重返伊甸园的向导。

这种对“想象的共同体”的怀旧,本质上是一种防御性的文化姿态。它试图在一个日益多元、流动和碎片化的世界里,重新划定“我们是谁”的边界。这个“我们”,通常被定义为国家的“真正”人民,是那些继承了“纯正”文化血脉的“土著”。而所有不符合这个标准的人——移民、少数族裔、信奉“全球主义”的精英——都被视为对这个共同体的威胁。因此,怀旧政治不可避免地与排外主义联系在一起。要“让美国再次伟大”,就必须建起高墙,阻止“他者”的涌入;要“夺回控制权”,就必须脱离那个被视为异质文化集合体的欧盟。

\section{故土的陌生人:当“我们”不再是中心}

当人口结构的变化、社会规范的迁移和对过去的怀旧这三股力量汇合在一起时,便催生出一种极具政治爆发力的情感体验:感觉在自己的国家里,变成了一个“陌生人”。

这种“故土陌生人”的感觉,在社会学家阿莉·霍赫希尔德(Arlie Russell Hochschild)对美国路易斯安那州茶党支持者的研究中得到了深刻的描绘。她发现,这些白人保守派选民感觉自己像是在排一个长队,等待实现“美国梦”。他们遵守规则,辛勤工作,耐心等待。但他们看到,队伍移动得越来越慢,而与此同时,他们感觉有很多人——少数族裔、移民、女性,甚至濒危的棕色鹈鹕——在“插队”,在政府的帮助下跑到了他们前面。而那些制定规则的“精英”(奥巴马政府),似乎还在指责他们这些排队的人是“种族主义者”、“落后者”。

这种“被插队”的感觉,是一种深刻的地位焦虑。它不仅仅是经济地位的焦虑,更是文化和社会地位的焦虑。在过去,作为国家的主体民族和文化核心,他们是当然的“主角”,他们的价值观就是社会的主流价值观,他们的生活方式就是“正常”的生活方式。而现在,他们感觉自己正在被推向舞台的边缘,甚至变成了故事里的“反派”。他们所珍视的一切——他们的宗教信仰、他们的爱国情怀、他们的传统家庭观——都在被解构、被批判。

这种从中心到边缘的坠落感,是难以忍受的。它带来的是一种深刻的怨恨:对那些“插队者”的怨恨,对那些帮助“插队者”的精英的怨恨,以及对整个“不公平”体系的怨恨。他们感觉自己被背叛了,被那些本应代表他们的政治家和制度所抛弃。他们成了“失落的部落”,在自己的土地上流浪,寻找着身份的庇护所。

民粹主义者正是这个“失落部落”的酋长。他们告诉这个部落的成员:你们的感觉是对的。你们不是偏执狂,你们是这个国家真正的主人。你们的困境,不是你们的错,而是那些腐败的精英和忘恩负义的“他者”造成的。他们用“我们vs他们”的简单叙事,为这种复杂的文化焦虑提供了一个清晰的敌人和宣泄的出口。他们承诺要恢复这个部落昔日的荣耀,让他们重新成为国家的主人。这种承诺,对于那些感觉自己一无所有、只剩下身份认同的人来说,具有无法抗拒的诱惑力。

\section{案例聚焦:欧洲文化保卫战的前线——勒庞与德国选择党}

在欧洲这片古老的大陆上,关于身份与文化的焦虑尤为突出。法国的玛丽娜·勒庞(Marine Le Pen)和德国的选择党(Alternative für Deutschland, AfD)是两个最典型的案例,他们都将自己定位为民族文化的“守护者”,以对抗其所感知的来自内部和外部的威胁。

\subsection{法国的“去妖魔化”与文化壁垒:玛丽娜·勒庞的国民联盟}

玛丽娜·勒庞从其父亲让-马里·勒庞手中接过“国民阵线”(后更名为“国民联盟”)的领导权后,推行了一项精明的“去妖魔化”(dédiabolisation)策略。她努力与父亲那一代人赤裸裸的种族主义和反犹言论划清界限,试图将该党包装成一个更温和、更“正常”的爱国主义政党。然而,其排外主义和文化保守主义的内核从未改变,只是换上了一套更精致、更符合当代法国政治语境的外衣。

勒庞的核心武器,是法国独特的“世俗主义”(laïcité)原则。在传统上,世俗主义旨在确保国家的中立,将宗教排除在公共领域之外。但在勒庞及其追随者的手中,它被巧妙地改造成为一把针对伊斯兰教的利剑。他们声称,穆斯林移民的某些文化习俗——例如女性在公共场合佩戴头巾(hijab)或罩袍(burqa)——与法国的世俗传统和性别平等价值观格格不入。因此,以捍卫世俗主义为名禁止这些服饰,便不再是宗教歧视,而是保卫法兰西共和国的核心价值。

在勒庞的叙事中,法国正面临双重威胁:自下而上,是“失控的”伊斯兰化,它正在侵蚀法国的文化认同;自上而下,是布鲁塞尔的“全球主义”精英,他们通过欧盟的超国家权力,试图消解法国的国家主权和独特性。因此,“国民联盟”的解决方案也是双重的:对内,要严格限制移民,强化对“法国价值观”的认同,捍卫一种以美食、美酒、特定节假日和生活方式为代表的“法兰西艺术生活”(art de vivre à la française);对外,则要从欧盟手中“夺回控制权”,恢复法国的边境和法律主权。

勒庞的言论精准地击中了法国社会中那部分对国家身份感到焦虑的群体。他们或许并不认同老勒庞的粗暴,但他们确实对社区的变化感到不安,对伊斯兰教在法国社会中日益增长的可见度感到警惕,对法国在全球化浪潮中似乎正在失去其独特魅力而感到失落。勒庞为他们提供了一套看似“理性”和“爱国”的话语,来包装和正当化他们的文化恐惧。她告诉他们,热爱法国、希望法国“保持原样”,并不是什么可耻的事情,而是一种值得骄傲的权利。

\subsection{德国的“主导文化”与历史修正:德国选择党(AfD)}

德国选择党的崛起,则与2015年的欧洲难民危机密切相关。这个最初由一群反对欧元的经济学教授创建的政党,在默克尔政府决定向中东难民开放边境后,迅速转型为一个以反移民、反伊斯兰为核心诉求的右翼民粹主义政党。

与勒庞相似,AfD也试图为自己的排外立场寻找文化上的合法性。他们提出了捍卫德国“主导文化”(Leitkultur)的口号。这个“主导文化”具体指什么,往往是模糊的,但其核心指向一个以基督教传统、德语、德国历史和特定社会规范为基础的文化共同体。AfD的领导人反复强调,“伊斯兰教不属于德国”,认为其教义和实践与德国的启蒙传统、法治精神和男女平等的价值观根本上是冲突的。

AfD的策略比勒庞更为激进,尤其是在对待历史的态度上。德国由于其纳粹历史,战后形成了一种深刻的“政治正确”,即对任何形式的极端民族主义和种族主义都保持高度警惕。AfD则公然挑战这种禁忌。其领导人亚历山大·高兰(Alexander Gauland)曾声称,希特勒和纳粹只是德国千年辉煌历史中的“一小撮鸟屎”,意在淡化纳粹罪行,呼吁德国人摆脱历史负罪感,重新建立一种“健康的”民族自豪感。

这种历史修正主义的言论,在德国社会引起了巨大争议,但也吸引了那些厌倦了国家“赎罪文化”、渴望德国成为一个“正常国家”的选民。尤其是在前东德地区,AfD的支持率异常之高。那里的民众在两德统一后经历了剧烈的社会转型,许多人感觉自己被视为“二等公民”,其在东德时期的生活经历和身份认同未得到充分尊重。AfD成功地将这种被剥夺感和对现状的不满,引导向了对移民和“柏林精英”的愤怒。

通过将文化焦虑、历史怨愤和对建制派的不信任巧妙地结合在一起,AfD成功地打破了德国战后的政治共识,成为了联邦议院中的一股重要力量。他们向那些感觉被时代抛弃、被主流话语压制的“失落的部落”成员承诺:我们将恢复你们的文化尊严,捍卫你们的德国身份,让你们的声音重新被听到。

\section{结语:当经济与文化怨愤共振}

本章的核心在于揭示,民粹主义的崛起,远非单一的经济现象。它深深植根于同样强大,甚至更为根本的文化土壤之中。对身份失落的恐惧,对生活方式受到威胁的焦虑,以及对一个想象中“黄金时代”的怀旧,共同构成了一股强大的社会暗流。当这股文化上的怨愤,与上一章我们所讨论的经济上的不安全感发生共振时,其所产生的政治能量是惊人的。

一个在经济上感到被剥夺的人,同时在文化上也感觉自己成了“故土的陌生人”,他所感受到的无力感和愤怒将是双倍的。他会觉得,不仅自己的饭碗被抢走了,连自己的家园、自己的灵魂归属地也正在被侵蚀。这时,任何承诺能同时解决这两个问题的政治力量,都将对他产生致命的吸引力。

民粹主义领袖正是看准了这一点。他们将经济问题“文化化”(例如,将失业归咎于移民,而非自动化),也将文化问题“政治化”(例如,将社会规范的变迁归咎于“精英”的阴谋)。他们编织了一个宏大的叙事,将“纯洁的人民”所遭受的所有苦难——无论是经济上的还是文化上的——都归结为同一个敌人:那个与“外人”勾结、出卖国家利益的“腐败精英”。

理解了这层文化和身份的维度,我们才能更全面地把握民粹主义为何在21世纪的今天如此势不可挡。它不仅仅是对新自由主义经济秩序的反叛,更是一场针对后现代文化多元主义的“部落战争”。然而,这些“失落的部落”为何会感到如此孤独无助?为何在他们看来,传统的主流政党无法再为他们提供庇护?这便引出了我们下一章将要探讨的问题:主流政治的失灵,是如何为民粹主义者留下了这个巨大的权力真空,让他们得以“空悬的王座”上加冕。