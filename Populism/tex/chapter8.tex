\part{后果——民粹主义时代下的生活}

\chapter{对真相的战争:从“假新闻”到“我的事实”}
想象一下,你和一群人正计划穿越一片广袤而危险的森林。你们手中有一张经过几代人勘探、修正和验证的地图。这张地图或许不完美,但它标明了河流、山脉、沼泽和安全的路径。现在,一位魅力非凡的领袖站了出来,他指着地图大声宣称:“这张地图是骗人的!它是森林里的权贵们为了把我们困在这里而编造的谎言!” 接着,他掏出一张自己手绘的、简单得多的地图,上面只有一条笔直的大道,直通他所承诺的“应许之地”。他告诉你们,只有懦夫和叛徒才会相信那张旧的“精英地图”,真正的爱国者应该追随他这张“人民的地图”。

于是,人群分裂了。一部分人继续相信经过验证的旧地图,而另一部分人则狂热地追随新领袖,将他的手绘草图奉为唯一的真理。很快,他们不再只是在争论哪条路更好,而是在争论河流是否存在,山脉是否只是幻象。他们生活在两个完全不同的现实里,共享的森林在他们眼中变成了两个无法兼容的世界。最终,他们不再是探险的同伴,而成了不共戴天的敌人。

这并非一则寓言,而是我们这个时代正在上演的真实剧目。民粹主义的崛起,不仅仅是政策偏好或政治风格的转变,它更是一场深刻的认识论危机。它的运作,需要且必然导致一场针对“真相”本身的战争。这场战争的目标,不是为了赢得辩论,而是为了摧毁辩论赖以存在的基础。民粹主义领袖深知,要让民众接受他们那套“我们vs他们”的简单叙事,就必须首先摧毁所有能够提供独立、复杂、有时甚至是令人不悦的真相的机构。

这场战争并非民粹主义的副作用,而是其核心战略。它侵蚀着民主社会最根本的基石——一个共享的现实。当一个社会失去了共同的事实基础,它就失去了一起思考、一起解决问题的能力。剩下的,便只有赤裸裸的权力斗争和部落间的相互敌视。本章,我们将深入这场战争的各个前线,揭示民粹主义者是如何系统性地攻击真相的裁决者,如何为他们的追随者构建一个另类的现实,以及这种“后真相”的诱惑,为何对现代人的心灵具有如此致命的吸引力。

\section{第一前线:处决信使——对“主流媒体”的系统性攻击}
在这场战争中,第一个被送上断头台的,永远是新闻媒体。这并非偶然。在一个健康的民主社会中,自由、独立的媒体扮演着至关重要的“守门人”和“监督者”角色。它们是公民获取信息、了解世界的主要渠道,也是监督政府权力、揭露谎言和腐败的关键力量。因此,任何想要垄断叙事权的民粹主义领袖,都必须将媒体——尤其是那些享有公信力的主流媒体——视为头号敌人。

唐纳德·特朗普是这套战术的集大成者,他将对媒体的攻击变成了一种政治行为艺术。他最伟大的“发明”,无疑是“\textbf{假新闻}”(Fake News)这个标签。这个词本身并不新鲜,但特朗普赋予了它全新的、武器化的含义。在他手中,“假新闻”不再是一个用于描述虚假信息的客观词汇,而是一个可以随时发射、用于摧毁任何他不喜欢的报道的“认知导弹”。一篇揭露其商业往来疑点的深度调查?假新闻!一项显示其支持率下降的民意调查?假新闻!一段记录其争议性言论的视频?还是假新闻!

这个标签的威力在于其简单粗暴的有效性。它不需要复杂的反驳或证据,只需要重复。通过日复一日、年复一年地将《纽约时报》、CNN等主流媒体斥为“假新闻”的制造者,特朗普成功地在他的支持者心中植入了一种“先发制人的怀疑”。这是一种高明的“政治免疫”策略:只要预先将信源污名化,那么无论这个信源未来发布多么确凿的证据,他的追随者都会因为不信任信源本身而选择无视。

更进一步,特朗普将对媒体的攻击仪式化、部落化。在他的大型竞选集会上,痛斥媒体是必不可少的保留节目。当他用手指着记者席,怒吼他们是“\textbf{人民的敌人}”(enemy of the people)时,台下数万名支持者会报以雷鸣般的欢呼和嘘声。这不仅仅是在表达不满,更是在进行一场集体身份的确认。通过共同憎恨一个“敌人”,支持者们之间的情感纽带被极大地强化了。对媒体的每一次攻击,都在巩固“我们”(诚实的爱国者)与“他们”(撒谎的精英媒体)之间的对立。

当然,这套剧本并非特朗普独有。在巴西,雅伊尔·博索纳罗反复攻击该国最大的报纸《圣保罗页报》,称其为“厕所报”;在匈牙利,我们在第五章已经看到,欧尔班·维克托没有选择仅仅攻击媒体,而是通过各种经济和政治手段,系统性地将大部分媒体收购、改造,使其成为政府的宣传机器;在印度,纳伦德拉·莫迪的政府和支持者们,则将任何敢于提出批评的记者和媒体,都打上“反国家”的标签。

所有这些做法,其最终目标都是一致的:摧毁独立媒体作为社会“真相公证人”的地位。当所有信使都被处决或被宣布为不可信时,信息的高地便出现了真空。此时,民粹主义领袖便可以通过他的“直通车”——无论是特朗普的推特、莫迪的广播节目“心灵对话”,还是杜特尔特的深夜电视讲话——成为其追随者唯一的、最终的、也是最可信赖的信息来源。他不再需要与事实辩论,因为他自己,已经成为了事实的定义者。

\section{第二前线:罢黜专家——对知识权威的全面否定}
战火很快从媒体蔓延到了更广阔的知识领域。民粹主义的核心逻辑——“拥有常识的人民”对抗“脱离群众的精英”——天然地与任何形式的专业知识权威相冲突。在民粹主义者看来,专家、学者、科学家,不过是精英阶层的一部分,他们用普通人听不懂的术语和复杂的模型,来服务于自身的利益,并试图剥夺“人民”对自己生活的决策权。因此,对专家的攻击,是民粹主义“赋权于民”叙事的必然延伸。

气候变化议题是这场战争的典型战场。几十年来,全球绝大多数科学家通过海量的研究,达成了“全球正在变暖,且人类活动是主要原因”的压倒性共识。然而,在民粹主义者的叙事中,这一科学共识被描绘成一个巨大的“骗局”。特朗普宣称气候变化是“中国人为了损害美国制造业而捏造的概念”;博索纳罗则为亚马逊雨林的加速砍伐辩护,将环保组织视为阻碍巴西发展的“外国代理人”。

他们巧妙地将一个科学问题,转化为了一个政治和经济问题。他们告诉选民:这些所谓的“气候专家”和“全球主义精英”,正试图用环保的借口,来摧毁我们的工作(煤炭、石油产业),提高我们的生活成本,并向超国家机构让渡我们的主权。在这种叙事下,相信科学,就等于背叛国家和人民的利益。于是,一个基于证据和理性的议题,变成了一场基于身份和忠诚的站队。

2020年爆发的新冠肺炎大流行,则将这场对专家的战争推向了高潮,并以一种前所未有的规模在全球上演。当病毒肆虐时,公共卫生专家和科学家的建议(如佩戴口罩、保持社交距离、实施封锁)本应成为各国政府制定政策的基石。然而,在许多民粹主义领袖治下的国家,这些专家反而成了被攻击的目标。

在美国,安东尼·福奇博士这位备受尊敬的传染病专家,被特朗普及其支持者描绘成一个试图通过制造恐慌来攫取权力、破坏经济的“\textbf{深层政府}”成员。戴口罩这一简单的公共卫生措施,被政治化为一场关于个人自由的文化战争。与此同时,特朗普、博索纳罗等人则大力推广各种未经科学验证的“神奇疗法”,如羟氯喹。这背后是一种典型的民粹主义逻辑:领袖凭借其超凡的直觉和与“人民”的神秘链接,拥有超越“所谓专家”的特殊智慧。追随领袖的直觉,比相信科学家的证据,更能体现对部落的忠诚。

这场战争还延伸到了司法和情报领域。当法院的判决阻碍了民粹主义领袖的议程时,法官们便不再是独立的法律守护者,而是“阻碍人民意愿”的“激进派法官”。当情报机构的报告与领袖的政治宣传相矛盾时(例如,关于外国干预选举的报告),这些机构便会被斥为阴谋颠覆政府的“\textbf{深层政府}”(Deep State)。

无论是科学家、医生、法官还是情报官员,他们工作的共同点是:尊重证据、遵循程序、并对超越党派政治的专业准则负责。而这,恰恰是民粹主义的强人意志所不能容忍的。通过系统性地摧毁对所有知识权威的信任,民粹主义领袖清空了所有能够制约其权力的独立判断标准。在一个没有专家、只有“敌人”和“朋友”的世界里,领袖的意志便成为了唯一的法律和唯一的真理。

\section{构建另类现实:阴谋论的生态系统}
摧毁旧的现实地图只是第一步,更重要的是,必须为追随者们提供一张全新的、能够自圆其说的另类现实地图。这张新地图,就是由无数阴谋论编织而成的。

阴谋论并非民粹主义时代的新发明,但它在今天扮演的角色却至关重要。它不再是边缘群体的窃窃私语,而成为了民粹主义政治的“操作系统”。阴谋论之所以如此有效,是因为它为复杂、混乱、常常令人感到无力的世界,提供了一个极其简单、清晰且在情感上极具吸引力的解释框架。

你的生活不如意?不是因为全球经济结构调整、技术变革和教育不平等等复杂因素,而是因为一个由“全球主义精英”组成的秘密集团在背后操纵一切。国家的传统正在消亡?不是因为社会自然演进和文化多元发展,而是因为金融大亨乔治·索罗斯正在资助一场旨在用移民“大置换”本土人口的邪恶计划。

在所有这些阴谋论中,美国的“匿名者Q”(QAnon)堪称一个登峰造极的范例。它已经不是一个单一的阴谋论,而是一个包罗万象的另类宇宙。在这个宇宙里,世界被一个由信奉撒旦的恋童癖组成的“深层政府”阴谋集团所统治,这个集团的成员包括民主党政客、好莱坞明星和金融大亨。而唐纳德·特朗普,则是被军方情报部门秘密选中、前来拯救世界、发动一场名为“风暴”的最终审判的救世主。

QAnon的信徒们拥有自己的一套“事实”、术语和“解码”世界的方式。他们从特朗普的每一次拼写错误、每一个手势中解读出“秘密信息”。这个另类现实是完全封闭且能够自我强化的:任何与QAnon叙事相悖的证据,都被解释为阴谋集团用来掩盖真相的“假新闻”;任何对QAnon的打击,都反而证明了“风暴”即将来临。

这个另类现实的构建,离不开一个庞大而高效的另类媒体生态系统。这个生态系统由右翼新闻网站、谈话电台、播客、YouTube频道、Facebook群组和无数社交媒体“网红”组成。它们像一个巨大的共鸣腔,不断地放大和重复着民粹主义领袖的叙事和各种阴谋论。

在美国,像福克斯新闻(Fox News)这样的主流有线电视频道,与布莱巴特新闻网(Breitbart)、The Gateway Pundit等网站,以及无数保守派意见领袖的社交媒体账户,共同构成了一个强大的信息闭环。在这个闭环里,关于“被窃取的选举”、亨特·拜登的笔记本电脑、新冠病毒的“真相”等叙事被反复传播,直到它们在受众心中变得比任何来自“主流媒体”的事实都更加“真实”。

社交媒体平台的算法,则在这个过程中扮演了“超级催化剂”的角色。为了最大限度地延长用户停留时间,算法倾向于向用户推荐更具煽动性、更能激发强烈情绪的内容。一个对主流媒体略感怀疑的用户,可能会被算法一步步地引导至越来越极端的阴谋论内容,最终陷入一个难以自拔的“信息茧房”(filter bubble)。在这个茧房里,他所看到的一切,都在验证他最初的偏见,而所有相反的观点,都已被算法过滤掉了。

最终,一个与现实世界平行的另类宇宙被成功地构建起来。生活在这个宇宙里的人们,与生活在现实世界中的人们,使用着不同的语言,相信着不同的事实,怀抱着不同的恐惧,也期盼着不同的未来。

\section{心理的诱惑:我们为何渴望“我的事实”}
面对这一切,一个最令人困惑的问题是:为什么?为什么有如此多的人,愿意放弃可验证的事实,转而拥抱那些漏洞百出、有时甚至荒诞不经的阴谋论和另类叙事?将他们简单地归结为“愚蠢”或“无知”,是一种傲慢的、无益的简化。要理解这一现象,我们必须深入其背后强大的心理诱惑。

首先,是认知失调(Cognitive Dissonance)和动机性推理(Motivated Reasoning)的力量。我们的政治观点,往往不仅仅是观点,更是我们身份认同的核心部分。如果你将自己定义为一位特定领袖的坚定支持者,那么当出现与这位领袖的言行或能力相悖的负面信息时,你的内心就会产生巨大的冲突。承认这些信息,就等于承认自己的判断失误,甚至否定自己的身份认同,这是非常痛苦的。因此,一个更轻松的心理选择是:拒绝这些信息,并攻击信息的来源。相信“假新闻”的叙事,可以完美地解决这种认知失调,保护我们免受心理上的不适。

其次,是对确定性的渴望。我们生活在一个充满不确定性的时代。全球化、技术革命、社会变迁,让许多人感到脚下的大地在移动,未来变得模糊不清。而民粹主义的简单叙事和阴谋论,恰恰提供了一种虚幻但诱人的确定感。它们将一个复杂的世界,简化为一场正邪分明的善恶之战,并为你指明了清晰的敌人。相信阴谋论,会让你感觉自己掌握了某种秘密知识,成为了少数“看清真相”的智者,从而在混乱的世界中重新获得一种掌控感。

最后,也是最重要的一点,是对部落归属感的需求。人类是社会性动物,归属于一个群体是我们最基本的需求之一。在政治极化的时代,相信和传播同样一套“另类事实”,成为了一种表达对部落忠诚的“投名状”。在社交媒体上分享一篇痛斥“假新闻”的文章,或是一个关于“深层政府”的阴谋论视频,其主要功能已经不是传递信息,而是在向你的同伴们发出信号:“我是你们中的一员。”这是一种强大的社会黏合剂。反之,质疑部落的“真理”,则可能面临被排斥和孤立的风险。因此,对许多人来说,坚持“我的事实”,是为了保住自己在“我们”这个集体中的位置。

\section{结语:破碎的镜子}
一个健康的社会,就像一群人共同注视着一面巨大的镜子。通过这面镜子,他们看到自己的全貌——有光荣,也有瑕疵;有共识,也有分歧。他们或许会为镜子中的影像该如何解读而争论,但他们至少承认,他们看到的是同一面镜子。

民粹主义的战争,其最终目的,就是彻底砸碎这面共享的镜子。当镜子碎裂成成千上万的碎片后,每一个部落都捡起其中一块。他们端详着自己手中的碎片,看到的只是自己被扭曲、放大的影像。他们坚信,自己手中的这块碎片,才是唯一真实的世界。他们对着其他部落高喊:“你们的镜子是假的!”却不知道,他们所有人都已经失去了看见全局的能力。

这场对真相的战争,是民粹主义时代最深远的后果,也是其最具腐蚀性的遗产。它并非暂时的政治分歧,而是一种长期的、对社会认知能力的结构性破坏。一个无法就基本事实达成共识的社会,不可能进行有意义的民主辩论,不可能形成有效的公共政策,更不可能在面对危机时团结一致。当“你的事实”和“我的事实”之间只剩下不可逾越的鸿沟时,对话便宣告死亡,妥协变得毫无可能。

这种共享现实的崩解,不可避免地会从认知领域蔓延到社会领域。当我们将一部分同胞视为生活在谎言中的“敌人”时,我们与他们之间的社会信任便荡然无存。这种信任的瓦解,正是我们将在下一章深入探讨的主题——一个在“我们vs他们”的逻辑下,日益走向分裂与崩解的社会。