\chapter{强人救世主:莫迪与杜特尔特}


如果说唐纳德·特朗普和英国脱欧派是右翼民族主义的表演者,欧尔班·维克托是蚕食民主制度的工程师,那么我们现在要遇到的,是民粹主义光谱中一个更古老、也可能更具原始力量的原型---强人救世主。

他们不只是承诺要``夺回控制权''或``让国家再次伟大'',他们的承诺更为根本:从混乱中恢复秩序,用铁腕重塑国家之魂。他们不仅仅是政治领袖,他们将自己塑造成国家的化身、人民的守护神、历史使命的执行者。他们的崛起,往往伴随着对果断行动甚至铁血暴力的渴求。在那些国家制度脆弱、腐败横行、民众普遍缺乏安全感的社会里,这种``\textbf{救世主}''的叙事具有无与伦比的吸引力。

要理解这一模式,我们必须将目光投向两个截然不同但又惊人相似的亚洲民主大国:印度的\textbf{纳伦德拉·莫迪(Narendra Modi)}和菲律宾的\textbf{罗德里戈·杜特尔特(Rodrigo Duterte)}。他们一位是禁欲苦行的印度教民族主义者,另一位是满口脏话的``惩罚者'',但他们共享着同一套剧本:将民粹主义与强烈的个人崇拜、民族或宗教身份认同相结合,承诺以雷霆手段扫除国家的一切``污秽'',并在此过程中,将自己塑造成国家意志与力量的唯一象征。

本章将剖析这套剧本,探讨为何在21世纪的今天,仍有数以亿计的选民心甘情愿地将希望寄托于一个``强人'',以及这种选择将为他们的国家带来怎样的未来。

\section{剧本第一幕:混乱的舞台与秩序的承诺}
强人并非凭空出现,他们是在民众对混乱的极度厌倦和对现有体制的彻底失望中应运而生的。在他们登场前,舞台早已搭好,剧本只缺一位主角。

\subsection{菲律宾的``罪恶之城''}

在2016年杜特尔特当选总统之前,菲律宾正深陷于一种弥漫全社会的无力感之中。猖獗的毒品交易、高发的犯罪率、警察系统的腐败无能,以及一个似乎永远无法触及普通人生活的政治精英阶层,共同构成了一幅令人窒息的图景。对许多菲律宾人来说,法律和秩序不是现实,而是一种奢望。毒品不仅是社会问题,更是一种侵蚀家庭、吞噬未来的``癌症''。

正是在这样的背景下,时任达沃市市长的杜特尔特横空出世。他在达沃市执政二十余年,以其严厉打击犯罪的铁腕手段而闻名全国。关于他组织``\textbf{达沃敢死队}''(Davao Death Squads)法外处决毒贩和罪犯的传闻甚嚣尘上。然而,这些在西方人权组织看来骇人听闻的指控,在许多菲律宾人眼中,却成了他有能力、有魄力的最佳证明。达沃市被宣传为``\textbf{菲律宾最安全的城市之一}'',成了他治理能力的活广告。

他的竞选承诺简单、粗暴且极具煽动性:``如果我当选总统,我会把所有罪犯都杀了…{}…{}把他们的尸体扔进马尼拉湾喂鱼。''他不像一个传统的政客,他更像一个民间传说中的``\textbf{惩罚者}''(The Punisher)。他满口脏话,鄙视繁文缛节,公开嘲笑人权组织是``罪犯的保护伞''。这种``不装''的姿态,与那些言辞优雅却毫无作为的马尼拉精英形成了鲜明对比,让厌倦了虚伪政治的民众感到无比``真实''和亲切。

他承诺的不是复杂的政策方案,而是一种立竿见影的``外科手术式''解决方案。他告诉选民,国家的沉疴需要猛药,而他就是那个唯一敢下猛药的医生。这场后来被称为``\textbf{禁毒战争}''的血腥风暴,在选举前就已经预告,而选民们用选票,热情地拥抱了这场即将到来的杀戮。

\subsection{印度的``政策瘫痪''与巨人苏醒的渴望}

与菲律宾街头的物理混乱不同,2014年莫迪上台前的印度,陷入的是一种政治和经济上的``慢性病''。此前由国大党领导的联合政府,被一系列惊天腐败丑闻所困扰(如``\textbf{2G频谱骗局}''、``\textbf{煤炭门}''),经济增长放缓,整个国家被一种``\textbf{政策瘫痪}''(policy paralysis)的氛围所笼罩。民众普遍认为,这个国家被一个腐败、世袭的精英家族(尼赫鲁-甘地家族)所把持,国家潜力被无休止的官僚扯皮和贪污腐败所消耗。

\textbf{纳伦德拉·莫迪},这位出身于古吉拉特邦一个普通种姓家庭的``\textbf{卖茶人}''(Chaiwala),恰好提供了另一种可能。作为古吉拉特邦的首席部长,他被塑造成一个高效、清廉、亲商的实干家。所谓的``\textbf{古吉拉特发展模式}''被宣传为印度未来的样板:基础设施飞速发展,外资涌入,行政效率高。他本人则是一个工作狂,一个为了国家而牺牲个人家庭生活的禁欲主义者。

但莫迪的形象远不止于此。他还是一个坚定的印度教民族主义组织--- \textbf{国民志愿服务团(RSS)}的长期成员。他的崛起,伴随着对\textbf{印度教特性(Hindutva)}的强调。2002年古吉拉特邦发生的大规模宗教骚乱,虽然给他的履历留下了血腥的污点,却也巩固了他在印度教民族主义者心中的``守护者''地位。

因此,莫迪向印度人民提供了双重承诺:
\begin{itemize}
\item \textbf{经济上的救赎者}:他将结束腐败和政策瘫痪,带来发展(\textbf{Vikas}),让印度这个沉睡的巨人彻底苏醒。
\item \textbf{文化上的守护者}:他将重振印度教的荣耀,洗刷数百年来被外族(穆斯林和英国人)统治以及被``伪世俗主义''精英压制的``耻辱'',让印度人在世界上昂首挺胸。
\end{itemize}
无论是杜特尔特承诺的物理秩序,还是莫迪承诺的政治经济与文化秩序,其核心逻辑是相通的:国家病了,旧的体制和精英治不好它,现在需要一个超越体制的强人,用非常规的、果断的手段来拯救一切。

\section{剧本第二幕:领袖即国家---个人崇拜的精心构建}
强人救世主的统治,不仅仅依赖于政策,更依赖于将领袖本人塑造成国家的化身。对领袖的忠诚等同于爱国,质疑领袖就是``反国家''。这种个人崇拜的构建,是一项系统性的工程。

\subsection{莫迪:从``人民公仆''到``世界导师''}

莫迪的形象塑造堪称教科书级别。他既是亲民的,又是超凡的。
\begin{itemize}
\item \textbf{人民公仆(Pradhan Sevak)}:他上任之初,便自称不是总理(Pradhan Mantri),而是``首席公仆''。他通过每月一次的广播节目``\textbf{心灵对话}''(Mann Ki Baat),直接与民众沟通,绕过主流媒体,营造出一种领袖与人民之间没有距离的亲密感。
\item \textbf{苦行僧与强人}:他的团队精心打造了一个禁欲、自律、不知疲倦的工作狂形象。他每天只睡几小时,严格吃素,没有家庭拖累,将全部生命奉献给了国家。这种形象在印度文化中具有强大的道德感召力,暗示着他不会被世俗的腐败所侵蚀。
\item \textbf{无处不在的``莫迪''}:他的头像出现在几乎所有政府福利计划的广告上,从免费煤气罐到公共厕所。新冠疫苗接种证书上印着他的照片,仿佛疫苗是他个人的赠礼。政府的成就被宣传为莫迪个人的成就,国家的荣耀就是莫迪的荣耀。
\item \textbf{神性与天命}:随着权力巩固,其形象塑造愈发大胆。他的支持者和党内同僚开始将他描绘成``天选之子'',是湿婆神的化身,是来完成神圣使命的。在他任内,备受争议的\textbf{阿约提亚罗摩神庙(Ram Mandir)}的落成,并将他作为首席主祭,完成了政治权力与宗教神性的终极合体。在奠基仪式上,他仿佛是一位现代的印度教君王,正在为他的王国奠定信仰的基石。
\end{itemize}
\subsection{杜特尔特:``父亲迪贡''与街头硬汉}

杜特尔特的个人崇拜则更具草根性和表演性。他不是高高在上的神,而是街坊里那个令人生畏却又让你感到安全的``大家长''。
\begin{itemize}
\item \textbf{``父亲迪贡''(Tatay Digong)}:支持者们亲切地称他为``父亲迪贡''。这个称呼暗示了一种威权式的父爱:他严厉、粗暴,会惩罚犯错的孩子(罪犯),但这一切都是为了整个家庭(国家)好。这种拟家庭化的政治叙事,极大地拉近了他与普通民众的心理距离。
\item \textbf{``真实''的魅力}:他那些充满咒骂和暴力威胁的深夜电视讲话,被他的支持者视为``不虚伪''、``接地气''的证据。他公开承认自己年轻时的劣迹,甚至开一些关于性侵的低俗玩笑。这些在传统政治中是不可想象的言行,反而强化了他``反精英''、``反建制''的形象。他看起来就像一个从街头酒吧里走出来的普通人,只不过恰好当上了总统。
\item \textbf{权力的表演}:杜特尔特深谙权力需要被``看见''。他会亲自去探望受伤的士兵,在镜头前怒斥毒枭,甚至威胁要驾驶水上摩托去南海插上菲律宾国旗。这些戏剧化的表演,无论能否实现,都在持续不断地强化他``说到做到''的硬汉形象。他将治国理政变成了一场充满个人英雄主义色彩的真人秀。
\end{itemize}
通过这种方式,莫迪和杜特尔特都成功地将国家议程个人化。印度的未来捆绑在了莫迪的``保证''上,菲律宾的秩序则系于杜特尔特的铁拳。任何对他们政策的批评,都很容易被转译为对国家本身、对人民希望的攻击。

\section{剧本第三幕:划分敌我---用身份政治巩固权力}
民粹主义的核心是``纯洁的人民''与``腐败的精英''的对立。而强人救世主模式则在此基础上,加入了更具爆炸性的元素:身份政治。他们不仅要打击腐败的精英,更要清除那些被定义为对国家/民族/宗教纯洁性构成威胁的``内部敌人''。

\subsection{莫迪的印度教民族主义剧本}

在莫迪的叙事中,``纯洁的人民''是印度教徒。而``敌人''的名单则很长:

\paragraph*{内部敌人:}
\begin{itemize}
\item 穆斯林:他们被描绘成历史上的侵略者后代,以及当下对印度教构成人口和文化威胁的``他者''。关于他们``不忠于国家''、``生育率过高''的阴谋论在社交媒体上泛滥。
\item ``反国家分子'':这个标签被随意贴给任何批评政府的人,包括左翼知识分子、人权活动家、独立记者和学生领袖。他们被指控为接受外国资助、意图分裂印度的``城市纳萨尔派''。
\item 世俗主义精英:以国大党为代表的传统精英,被攻击为``伪世俗主义者'',为了骗取穆斯林选票而出卖了印度教徒的利益。
\end{itemize}

\paragraph*{外部敌人:}
\begin{itemize}
\item 主要是巴基斯坦,任何对巴基斯坦的强硬姿态,都能极大地激发国内的民族主义情绪。
\end{itemize}
这套剧本通过一系列争议性的法律和行动付诸实施。例如,《公民身份修正案》(CAA),为来自邻国的非穆斯林非法移民提供了获得公民身份的快速通道,却唯独将穆斯林排除在外。这与\textbf{全国公民登记册(NRC)}相结合,在占印度总人口14\%的2亿穆斯林中制造了巨大的恐慌,他们担心自己可能在一夜之间被剥夺国籍,沦为无国籍者。

这些政策的实际效果或许有限,但其政治象征意义是巨大的。它们清晰地向印度社会传递了一个信号:在莫迪的``新印度'',印度教徒是``一等公民'',而其他人的地位则取决于他们的``忠诚度''。每一次争议,每一次抗议,都反而成了莫迪动员其核心支持者的机会,让他们相信,领袖正在为保护他们而战。

\subsection{杜特尔特的``反毒品''十字军东征}

杜特尔特的敌人划分则更为简单直接,也更为血腥。

\paragraph*{``人民'':}
\begin{itemize}
\item 是遵纪守法、辛勤工作的普通菲律宾人。
\end{itemize}

\paragraph*{``敌人'':}
\begin{itemize}
\item 是吸毒者和贩毒者。在杜特尔特的语言体系里,这些人被彻底``非人化''。他们不是犯了错的公民,而是必须被清除的``社会毒瘤''、``行尸走肉''。他曾说:``你们这些狗娘养的,我真的会杀了你们。''
\end{itemize}
这场``禁毒战争''与其说是一项公共政策,不如说是一场道德上的``十字军东征''。警方公布的官方死亡人数超过6000人,但人权组织估计的真实数字可能高达数万。在贫困社区,仅仅是被人``指认''为涉毒,就可能招来杀身之祸。

这种划分的残酷性在于其有效性。通过将一个群体定义为``非人'',针对他们的暴力就变得可以被接受,甚至值得称赞。支持杜特尔特的民众认为,这些法外处决是在``清理垃圾'',是为了保护他们的家庭和社区所必须付出的代价。而任何为这些受害者辩护的人---无论是人权律师、天主教会还是国际媒体---都会被立即打上``毒品同情者''和``国家敌人''的标签,从而在道德上被孤立。

通过这种清晰而残酷的敌我划分,莫迪和杜特尔特都成功地将复杂的社会经济问题,简化为一场正邪之间的道德决战。在这场决战中,他们是正义的化身,而他们的支持者则是正义大军的一员。

\section{比较视角:为何此模式在发展中民主国家尤为盛行?}
莫迪和杜特尔特的成功,揭示了强人模式在特定土壤中强大的生命力。这种土壤在许多发展中民主国家都存在。
\begin{itemize}
\item \textbf{制度的脆弱与信任的真空}:当司法系统缓慢而腐败,警察不可信赖,官僚体系臃肿低效时,民众对正式制度的信任就会崩塌。强人承诺的``法外正义''和``高效执行力'',成了一种诱人的替代品。他就像一个``超级CEO''或``超级警察'',能够绕过所有繁文缛节,直接解决问题。
\item \textbf{后殖民时代的身份焦虑}:许多发展中国家在摆脱殖民统治后,始终面临着``我们是谁''的身份认同问题。国家建构尚未完成,民族凝聚力脆弱。强人通过提供一个强大、统一、甚至排他的民族/宗教身份叙事,填补了这种身份焦虑。莫迪的印度教民族主义,正是要塑造一个``纯粹''的印度身份,以对抗殖民和``世俗主义''带来的``混杂''。
\item \textbf{对安全和秩序的基本渴求}:在许多西方成熟民主国家,人身安全和公共秩序在很大程度上被视为理所当然。但在犯罪率高企、社会动荡不安的地方,它们是最稀缺、最宝贵的公共产品。杜特尔特的支持者愿意用一部分民主权利,去交换街道的安全感。这种交换,在他们看来是理性的。
\item \textbf{经济不平等与尊严的丧失}:全球化和快速的经济转型,在创造财富的同时,也留下了大量感觉被抛弃的群体。强人不一定能提供更好的经济方案,但他们能提供一种心理补偿:尊严。通过将国家的强大与领袖的强大划等号,通过打击``内部敌人''和在国际上展现强硬姿态,他们让那些在经济上失落的人,在心理上重新获得了作为``胜利者''的感觉。
\end{itemize}

\section{结语:救世主的代价}
强人救世主的剧本,无疑是民粹主义中最具诱惑力也最危险的版本之一。它精准地回应了人类内心深处对秩序、安全、身份和尊严的渴望。它将复杂的治理问题简化为一场激动人心的道德斗争,为民众提供了清晰的敌人和一位值得追随的英雄。

然而,迎接救世主的代价是高昂的。为了换取强人承诺的秩序,社会往往需要牺牲民主的基石。法治被领袖的意志所取代,制衡机制被系统性地削弱,异议的声音被压制,少数群体的权利被无情地践踏。社会在``我们''与``他们''的对立中被撕裂,仇恨的种子一旦播下,便会疯狂生长。

杜特尔特的任期结束了,但他留下的血腥遗产和被毒化的政治文化,将长期困扰菲律宾。莫迪则仍在继续重塑印度,这个世界上最大的民主国家,正走在一个通往``选举式威权主义''的危险轨道上。

强人或许能带来一时的稳定和虚幻的强大感,但他们最终会侵蚀一个国家赖以长期繁荣和稳定的制度与文化根基。他们承诺拯救国家,但最终,国家可能需要从他们带来的``拯救''中被拯救。理解了他们的剧本,我们才能更清醒地认识到,当一个领袖承诺要成为你的一切问题的答案时,他很可能本身就是那个最大的问题。这,正是我们将在下一部分``后果''中深入探讨的核心议题。